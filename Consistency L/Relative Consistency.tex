\documentclass{article}

\renewcommand*\familydefault{\sfdefault} 
\newcommand{\du}{\Diamond_\uparrow}
\newcommand{\dl}{\Diamond_\leftarrow}
\newcommand{\bu}{\Box_\uparrow}
\newcommand{\bl}{\Box_\leftarrow}


\usepackage[T1]{fontenc}
\usepackage{amsmath, amssymb, amsthm}
\usepackage{bussproofs}
\usepackage{tcolorbox}
\usepackage[]{multicol}
\newtheorem{lemma}{Lemma}
\newtheorem{corollary}{Corollary}
\newtheorem{definition}{Definition}
\newtheorem{theorem}{Theorem}
\newtheorem{fact}{Fact}

\title{On the Consistency Strength of Axioms for Height and Width Potentialism}
\author{Chris Scambler}
\date{\today}
\begin{document}
\maketitle
\abstract{
    There are two parameters along which the universe of sets might 
    conceivably differ in size: it may have more or fewer ordinals, 
    and respectively may naturally be construed as taller or shorter;
    or at any given ordinal height, the initial segment of the 
    universe up to that ordinal may have more or fewer subsets, 
    and the universe may in each case naturally be constured, respectively, 
    as fatter or thinner.
    A potentialist about the height of the universe believes that, no matter what,
    there could always be further ordinals; 
    a potentialist about the width of the universe believes that, no matter what, there could 
    always be more subsets of any given infinite set.
    
    
    It is well known that natural axiom systems for height potentialism 
    are equi-consistent with axiom systems related to ZFC. 
    On the other hand, although recent literature has seen some discussion of 
    height and width potentialism combined, analogous results precisely correlating 
    these axiom systems with familiar, purely quantificational theories do not yet exist.    
    
    This paper provides such results. I show that the axioms for height and width 
    potentialism from Scambler CITE exhibit the same forms of equivalence with 
    second order arithmetic extended by topological regularity principles. 
    The results are significant because they 
    helps to round out the foundational 
    picture underlying height and width potentialism, and to bring out how different it 
    is from the standard ZFC-conception of set.
}
\section{Overview}
The purpose of this paper is to present some new results on the consistency strength of 
axiomatic systems for height and width potentialism combined, and to explain 
what I take their significance to be. The results have a rather 
technical flavor and involve elements of higher order modal logic, the theory of 
forcing, and the topology of the real line. In order to maximize intuitiveness, and 
to make the paper as accessible as possible, I have therefore decided to follow a policy 
of `double elaboration': I will begin by explaining all the result of interest 
and the methods used to attain them with a broad brush, one that won't be accurate 
enough to pass contemporary standards of logical rigor but that will suffice to get
the key ideas accross in sufficient detail to sustain some philosophical discussion. 
I will then provide a rather compressed technical appendix that shows how to substanitate 
the results cited. For the reader who has read the informal presentation, I hope, this 
more compressed formal development should be sufficient.

I will begin by discussing the sense in which axiom systems for height potentialism
are equivalent to set theories related to ZFC, and the techniques used to attain 
the equivalence theorems. I then turn to discussing the relevant axioms 
for height and width 
potentialism, and outline an argument that they share the same form of equivalence with certain extensions 
of second order arithmetic. To close, I sketch some further results that apply in the 
case of height potentialism/ZFC but fail, at least in any straightforward sense, 
in the case of height+width potentialism/second order arithmetic. 
To close the main body of the paper, I discuss the relevance of these results for our understanding 
of the height/width potentialist point of view.
Finally, a pair of appendices substantiate the technical results.
\section{Warm Up: Height Potentialism}
We begin by sketching template results regarding height potentialism. 
\subsection{Axioms}
We start with a discussion of the `natural' axiom systems for height potentialism 
that are at issue.

The systems will be directed at axiomatizing the following 
intuitive conception. Imagine there is a being who is capable of taking things 
and \emph{collecting} them together into a set, and of repeatedly performing this act 
as many times as is conceivable. Say that a \emph{set} is something that, given enough 
executions of this basic act, could be produced by such a being. 

We want to axiomatize this conception in modal logic. The modal operator, $\Diamond$, 
is to be interpreted so that $\Diamond \varphi$ means
`by repeated acts of collection, $\varphi$ can be made to hold'. The dual 
operator, $\Box \varphi$, means `no matter how many acts of collection are performed, 
$\varphi$ will hold'. The axioms governing this modal logic can be taken to be 
$\mathsf{S4.2}$ with the converse Barcan formula. More details are given in Appendix A.\footnote{(We do not want $.3$, 
since given $a \not= b$ we may form either $\{a\}$ or $\{b\}$, but we do want 
$.2$,
so from either of these we can get $\{a\}$ and $\{b\}$.) The rule of necessitation 
is assumed, along with the standardly valid inference rules:
\begin{prooftree}
    \AxiomC{$\Phi_1 \rightarrow \Box (\Phi_2 \rightarrow ... \Box ( \Phi_n \rightarrow \Box \Psi) ... )$}
    \UnaryInfC{$\Phi_1 \rightarrow \Box (\Phi_2 \rightarrow ... \Box ( \Phi_n \rightarrow \Box \forall x \Psi) ... )$}
\end{prooftree}}
 
Our conception says that the being can take any \emph{things} whatsoever and collect 
\emph{them} into a set. These are plural notions, and accordingly we shall want a plural 
langauge as well. We thus take a stock of plural variables $X$ and singular variables $x$.
The circumstance of something $x$ being one of some things $X$ will be expressed by 
the concatenation $Xx$. The precise details, once more,  are consigned to Appendix A. 

The set-theoretic axioms are more interesting. We assume all the usual 
background stuff: the axiom of extensionality for sets, rigidity for the 
membership relation, and the axiom of foundation. But in addition, we have 
distinctively potentialist existence principles, the most fundamental of which 
being:

\begin{description} 
    \item[HP] $\Box \forall X \Diamond \exists x [Set(x, X)]$
\end{description}
Here and throughout, $Set(x, X)$ is an abbreviation for 
$\forall z[z \in x \equiv Xz]$; \emph{HP} thus says: \emph{necessarily, any 
things can be the elements of a set}. Under the given 
intuitive interpretation, this corresponds to the idea that the being 
can always collect together any given things into a set in any circumstances.  
(Hence, in contrast to standard set theory, there are no 
special things that somehow can't be made to form a set.)

\emph{HP} allows us (given plural comprehension) to prove the possible existence 
of each specific hereditarily finite set. But it does not allow for a proof that 
it is possible to produce any infinite sets. 

In the spirit of transfinite 
set theory, we will want to secure this possibility axiomatically. 
One way to to this uses the following \emph{Completeability} principle. 
Let $\mathbb{N}(x)$ be the assertion that $x$ is a transitive set well-ordered 
by $\in$ which, in addition, is Dedekind finite. Then we have the following 
so-called completeability principle for the nautral numbers:
\begin{description}
    \item[Comp$_\mathbb{N}$]
    $\Diamond \exists X \Box \forall y[ Xy \leftrightarrow \mathbb{N}(x)]$
\end{description}
This says that it is possible, eventually, for the being to create \emph{all possible}
natural numbers. It can be used to prove the possibility 
e.g. of $\omega$'s existence, given the other axioms.

Indeed, in conjunction with \emph{Comp$_\mathbb{N}$}, 
\emph{HP} proves the possible existence of infinitely many
infinite sets. But there is no guarantee from the axioms stated up to now that 
the modal analogue of the powerset axiom should old: that it should be possible, 
given any set, infinite or no, eventually to produce \emph{all} its possible subsets. 
We also have no guarantee that any uncountable sets are possible.

Axiomatically, these things can be imposed by appeal to another completeability axiom:
\begin{description}
    \item[Comp$_\subseteq$]
    $\Box \forall x \Diamond \exists X \Box \forall y[ Xy \leftrightarrow y \subseteq x]$
\end{description}
This time, we assert that the subsets of any given set are always `completeable': 
under our intuitive interpretation, 
it says that one can always eventually form all possible subsets of any given set. 
The axiom can easily be shown is readily seen to have the desired consequences, as 
for example proving the possibility of uncountable sets in the strong sense that 
it is necessary that there is no function defined on the natural numbers with the set 
a subset of its range.

As a final axiom, or rather axiom scheme, we throw in a modal 
translation of replacement (the $\Diamond$-translation in the sense defined below).

These are all fairly natural axioms to adopt if one wishes to explicate the idea 
of a \emph{never ending set-construction process} in modal terms. One has an axiom 
asserting that any things possibly form a set, along with various other axioms 
that implicitly concern the lengths of iterations of set formation that are possible.

For ease of reference the axioms 
for this system are compiled in Appendix A. I will refer to them as the system $\mathsf{L}$,
for their creator, \O ystein Linnebo.

We now turn to a discussion of the proof-theoretic relations between $\mathsf{L}$
and ZFC. 
\subsection{Key Results}
We will focus for now on two ``central results''. They describe a pair of ways 
to interpret the potentialist language in terms of the actualist one and vice versa.


The first central result is that there is a translation $\varphi \mapsto \varphi^\Diamond$ 
from the language of second order set theory 
to the potentialist language $\mathcal{L}_0$ that preserves theoremhood 
from ZFC into $\mathsf{L}$ \emph{on first order formulas}. 

The key idea behind the translation, $\varphi \mapsto \varphi^\Diamond$,  
is for the potentialist to interpret the actualists quantifiers as tacitly modalized:
when the actualist says, for example, that `there is no set of all sets', the potentialist 
will interpret them as saying, `it is not possible for there to be a set that, necessarily, contains all sets',
which is a theorem from their point of view. Slightly more formally, 
the idea for $t_\diamond$ is to put a $\Diamond$ in front of every existential quantifier 
(and a $\Box$ in front of 
every universal). The variables get mapped to themselves, and the translation 
commutes with the connectives in the obvious way, so for example $t(\varphi \wedge \psi)$
is $t(\varphi) \wedge t(\psi)$. 

That the translation has the desired property is a theorem due to Linnebo. It makes 
use of the following central fact, which will be appealed to repeatedly. 

\begin{fact}
    Suppose all atomic predicates in a language $\mathcal{L}$ are rigid in some theory $T$,
    and that $\Box$ 
    is a modal operator of $\mathcal{L}$ with the axioms of S4.2 together with the 
    converse Barcan formula according to $T$. Then we have 
    $\Gamma \vdash \varphi$ iff $\Gamma^\Diamond \vdash_T \varphi^\Diamond$, where here 
    $\vdash$ represents provability in classical logic. 
\end{fact}
Given the fact, the prove that the proposed translation works amounts to showing that 
the modal translation of the first order axioms of ZFC hold in $\mathsf{L}$, which is a 
routine exercise. 

Why does the result only hold for first order formulas? Well, it is an equally easy 
exercise to prove that the $\Diamond$ translation of second order comprehension 
is inconsistent over the other axioms of $\mathsf{L}$. Thus, to the extent that 
the vocabulary is the same, this translation does not extend to the whole language. 
It is tempting to view this result as showing that, although there is a great amount 
of agreement between $\mathsf{L}$ and theories like ZFC, nevertheless there are substantive 
disagreements too. They are not `equivalent descriptions'.

Nevertheless, the translation \emph{does} have certain useful consequences. Saliently, 
it means that to the extent that we are interested in first order ZFC as a foundation 
for mathematics, $\mathsf{L}$ likewise has the `logical power' to serve as such a foundation.
Relatedly, it also means that in enquiries involving $\mathsf{L}$ we can always `import'
the modalized translations of theorems of standard set theory. Finally, I'd argue 
that it also shows a sense in which the `pictures' of the set-theoretic universe
offered by both theories are quite similar: they both agree, up to this translation, 
that every set first `appears' at some rank of the cumulative hierarchy structure; 
even if they disagree over the `modal profile' of said hierarchy, as well as on issues 
like whether there is a plurality unbounded in the (possible) ordinals.

The second central result is that there is a translation $\varphi \mapsto \varphi_\exists$ from the 
potentialist language $\mathcal{L}_0$ to the language $\mathcal{L}$ of second order 
set theory that preserves theoremhood from $\mathsf{L}$ to ZFC. Thus ZFC can 
interpret the claims of $\mathsf{L}$. This time, agreement on the full language can be 
attained, and the result is therefore a consistency proof for $\mathsf{L}$ relative to ZFC.





The guiding idea behind the translation  $\varphi \mapsto \varphi^\exists$ is 
dual to that of the previous translation.
With the $\Diamond$ translation, the potentialist injected a modal element into 
the talk of the actualist to make their claims come good. Here, with the $\exists$ translation,
we will do the opposite: the actualist will remove the modal element from 
the potentialist's talk and replace it 
with quantificational talk about `possible worlds'.
In a bit more detail,
the actualist will define the notion of a possible world, 
and then interpret a potentialist assertion 
like `necessarily, any things can be the elements of a set' as saying that, 
for any possible world 
we choose, and any things contained in that world, 
\emph{there exists} a more expansive ``possible 
world'' that contains a set with those things as its elements. And with the right 
definition of world, it turns out that in ZFC we can make all the assertions of 
$\mathsf{L}$ come true. 

How should the actualist define the notion of a possible world? That will depend a 
little bit on some details of the potentialist account we have left open. For example, 
does the potentialist allow that, given it is possible there are some sets $X$, it is also 
possible that the sets are exactly $X$ together with $\{X\}$? Or must 
all the subsets of any given set be introduced together? Different answers on the 
part of the potentialist will lead to different matching definitions of world on 
the part of the actualist. For simplicity, we take a maximally liberal conception of 
world: on this view, any transitive extension of the universe is a possibility for the 
potentialist. (Given any sets $X$, and plurality $Z$ contained in $X$, it is possible 
that $X$ and $\{Z\}$ are exactly the sets.\footnote{
    In slogan form, sets are introduced `one at a time'.}) On this conception, \emph{any}
transitive set containing everything recgonized by the potentialist is a `possible world'
in the actualist's translation.

In order to accomodate these ideas, the actualist's translation of potentialist claims 
will need to be parameterized: given a formula $\varphi \in \mathcal{L}_0$, its 
translation will be a claim in $\mathcal{L}$ with an additional free variable, $w$, 
which stands for the world relative to which the claim is being evaluated. Quantifiers 
in the translations of claims will then be bound by the free world variable; 
modal operators will be interpreted as quantifying over the world variables. So for
example the potentialists `there is an $x$ that $\varphi$s' will be interpreted as 
`there is an $x \in w$ that $\varphi_\exists$s at $w$', where $\varphi_\exists$ing `at $w$' is recursively 
defined by the other clauses. And `necessarily, $\varphi$' will be interpreted as meaning 
that `for all transitive sets $u$ containing $w$, $\varphi_\exists$ holds at $u$'.

Finally, one defines the translation $\varphi_\exists$ for a modal formula $\varphi$ 
to say: ``if $w$ is a transitive set, then $\varphi_\exists(w)$''.
One can then prove that if $\varphi$ holds in $\mathcal{L}$ implies that 
$\varphi_\exists$ holds in second order set theory.

What does this result show? Well crucially it shows that any inconsistency in 
$L$ can be converted into an inconsistency in second order set theory, 
so that's nice. And it combines with the previous result to show that the 
\emph{first order} modal theory associated with L has exactly the power 
of first order ZFC.

There are ways in which these results can be strengthened toward the idea of a 
`definitional equivalence' between the two theories. But we'll set this aside here.



\section{Height and Width Potentialism}
We now consider analogous results derivble for systems combining height potentialism 
and width potentialism.

\subsection{Axioms for the ``Simple Theory''}
Ultimately I will present two axiomatic theories: 
the \emph{simple} $\mathsf{T}_1$ and \emph{strong} $\mathsf{T}_2$.
This subsection deals with axioms for the simple theory.

Each system is directed at axiomatizing the following intuitive (if fanciful) conception.
As before, we suppose we have a being who is able to take things and \emph{collect} 
them together into a set, and to do so arbitrarily many times.
But now, we will also take them to be able to perform another kind of basic act: namely, 
that of taking some things, and \emph{counting} them, that is, 
of correlating them one-for-one
with the natural numbers. 

The axioms to be presented
will concern the possibilities for arbitrary iterations 
of \emph{collect} and \emph{count}. We will think of a set as anything 
you can get by indefinitely iterating these two basic acts in any order,
and lay down axioms designed to capture this idea.

\emph{Collect} is a height potentialist 
principle, in that the set collected together has higher rank than the collected sets.
Count on the other hand is naturally understood as a width-potentialist principle, 
as the introduction of such enumerating functions in general implies the possible 
existence of `new' sets of natural numbers over any given domain, so that there are necessarily 
always width expansions of infinite ranks. 



The language for the simple theory is just $\mathcal{L}_0$, and we
take over all the basic logical axioms from $\mathsf{T}_0$.
In addition, it contains the following two core principles:
\begin{description} 
    \item[HP] $\Box \forall X \Diamond \exists x [Set(x, X)]$
    \item[WP] $\Box \forall X \Diamond \exists f[f : \mathbb{N} \twoheadrightarrow X]$
\end{description}
The first is taken over straightforwardly from the height potentialist theory $\mathsf{T}_0$.
The second is an obvious implementation of the universal possibility of enumeration 
alluded to a moment ago. 

As before, we will also endorse Comp$_\mathbb{N}$, so we can find infinite 
sets to play with; similarly, so we do recursions and things, we will 
take $Rep^\Diamond$, the translation of the replacement schema. Let the resulting theory 
be $\mathsf{T}_1$.

Note that $\mathsf{T}_1$ does not contain Comp$_\subseteq$. In fact, it can 
do this only on pain of triviality.

\begin{lemma}
$\mathsf{T}_1 \vdash \neg$ Comp$_\subseteq$.
\end{lemma}
\begin{proof}
    It suffices to argue for the possibility of a set $x$ such that necessarily, 
    for any plurality $X$ comprising only subsets of $x$, it is possible there is a 
    subset $y$ of $x$ that is not one of the $X$. Let $x$ be $\omega$, which 
    possibly exists by Comp$_\mathbb{N}$ and HP. Suppose that there are some $X$ such 
    that the $X$ are necessarily all subsets of $\omega$. By WP, it is possible 
    there is a function enumerating the $X$; by plural comprehension, there are 
    some $C$ comprising all those $k$ for which $k \not \in f(k)$. By HP it is possible 
    there is a set $c$ whose elements are $C$. But $c$ cannot be one of the $X$: 
    otherwise, $f(k) = c$ for some $k$, so
    $$k \in c \equiv k \in f(k) \equiv \neg Ck \equiv \neg k \in c$$
    a contradiction. 
\end{proof}
Recall that Comp$_\subseteq$ was our `modal analogue' of the powerset axiom. There is thus 
a prima facie incompatibility between height + width potentialism together and 
the powerset axiom of standard set theory. 

It is also worth mentioning that this form of width potentialism is closely related 
to forcing potentialism in set theory. Let $\mathbb{P}(x)$ abbreviate the claim 
that $x$ is a partial order, let $Den(D, x)$ abbreviate the assertion that the $D$s 
are the dense subsets of $x$, and then finally let $Fmeets(x, X, y)$ abbreviate the 
claim that $y$ is a filter in $x$ that meets all the $X$. Then the assertion
\begin{description}
    \item[WPf] $\Box \forall x \mathbb{P}(x) \wedge Den(D, x) \supset \Diamond \exists y (Fmeets(x, D, y))$
\end{description}
amounts to the assertion that every partial order can be forced over. Again, we have 
the following:
\begin{lemma}
    Over $\mathsf{T}_1$ without WP, WP and WPf are equivalent.
\end{lemma}
So one can naturally think of the simple theory $\mathsf{T}_1$ as combining height 
potentialism with forcing potentailism. 

\subsection{Key Results on $\mathsf{T}_1$}
We saw previously that $\mathsf{T}_0$ stood in a close interpretational relationship 
with ZFC2. In this section we will make minor modifications to those arguments 
to establish a similar connection between the simple height and width potentiailst 
theory $\mathsf{T}_1$ and second order arithmetic.

The central results considered here use the same translations as were used in 
proving the key results about $\mathsf{T}_0$, and the proofs follow essentially the 
same pattern. 

The first central result is that the translation $\varphi \mapsto \varphi^\Diamond$ 
again preserves theoremhood from a certain quantificational theory 
to the (first order part of) $\mathsf{T}_1$: except this time, 
rather than the quantificational theory 
being ZFC, it is PA. 

The proof is straightforward. PA is known to be definitionally equivalent to 
ZFC without power + all sets are hereditarily countable. It is completely 
routine to see that the $\Diamond$ translations of all these axioms are theorems 
of $\mathsf{T}_1$. So Fact 1 implies the result. 

The second central result now takes a predictable form: namely, that the 
translation $\varphi \mapsto \varphi_\exists$ maps theorems of $\mathsf{T}_1$ 
to theorems of PA2. Again, we use the definitionally equivalent form of the latter 
stated in terms of $\in$. The translation once again invokes a notion of possible 
world, defined as a transitive set, and uses the fact that PA2 proves every set is 
contained in a transitive set, and proves every set is hereditarily countable. 
The details are provided in one of the appendices.

As before, these results admit partial strengthenings in the direction of a definitional
equivalence; as before these are set to one side. As before, there are also 
some modest gains to be made by observing the existence of these translations. 
On the on hand we may help ourselves to translations of PA2 when reasoning in $\mathsf{T}_1$;
on the other, we know that any contradictions in our simple height/width potentialist theory,
or height/forcing potentialist theory, may be translated to contradictions in 
second order arithmetic.

\subsection{Axioms for the ``Strong Theory''}

The simple theory $\mathsf{T}_1$ is mathematically weak compared to $\mathsf{T}_0$.
The business of this section is to provide (and motivate) axioms for height and width 
potentialism combined that are stronger. We will find some that are in fact 
as strong as $\mathsf{T}_0$, and that admit a reasonable conceptual motivation.

When we discussed the informal motivation for the simple theory we had recourse to 
two operations: the operation of \emph{collecting} together some things into a set and expanding 
the universe upward,
as well as the operation of \emph{counting out} some things, that is, correlating them 
one-for-one with the natural numbers, and expanding the unvierse outward.
(One could also exchange the latter with the operation of \emph{forcing over a partial order}, 
equivalently, given WPf.)
We then axiomatized the conception using 
the modal operator $\Diamond$, which was to reflect possibility by arbitrary 
iterations of either. We found that WP meant we couldn't have Comp$_\subseteq$, 
so we were left with a weak theory equivalent to set theory without powerset.

The first move toward the strong theory is to reflect the fact that we have, 
intuitively, \emph{two} ways of expanding the domain -- \emph{up} and \emph{out}, 
or \emph{collect} and \emph{count} -- in the language, taking on a modal $\du$ for vertical 
expansion of the domain and $\dl$ for horizontal. Thus, our formal language $\mathcal{L_2}$
will have operators:
\begin{itemize}
    \item $\du \varphi$, meaning $\varphi$ can be made true by iterated height expansion
    \item $\dl \varphi$, meaning $\varphi$ can be made true by iterated width expansion
    \item $\Diamond \varphi$, meaning $\varphi$ can be made true by any means
\end{itemize}
What axioms should we impose? For the modal logic, we will 
take S4.2 for each operator, and impose the natural constraint that $\du \varphi \supset \Diamond \varphi$ 
(vertical-possibility implies possibility) and similarly for $\dl$. 
We will assume essentially the same conditions on identity and existence for plurals and sets.

More substantially, the core axioms HP and WP are naturally modified:
\begin{description} 
    \item[HP$_\uparrow$] $\Box \forall X \du \exists x [Set(x, X)]$
    \item[WP$_\leftarrow$] $\Box \forall X \dl \exists f[f : \mathbb{N} \twoheadrightarrow X]$
\end{description}
these are now the central tenets of each separate modality. 

Many of the motivating thoughts behind axioms in $\mathsf{T}_0$
carry over in a straightforward way to axioms for $\mathsf{T}_2$ that concern $\du$.
For example in the case of the axiom of infinity, we thought that we could keep introducing 
sets long enough that we would eventually get all the natural numbers; and in $\mathsf{T}_2$ we 
should surely axiomatize this in terms of height expansion,

\begin{description}
    \item[Comp$_\mathbb{N}$]
    $\du \exists X \Box \forall y[ Xy \leftrightarrow \mathbb{N}(x)]$
\end{description}
Of course, this implies the more general version that begins with $\Diamond$.

The real strength of the strong theory 
lies in the fact that, beacuse it separates the modalities out in this way, 
strong axioms of possible set existence can be asserted \emph{relativized to one 
or other of these}, even if the strong axioms do not hold in general. For example, 
it is easy to modify the proof of theorem blank in $\mathsf{T}_1$ to a proof of the same result here:
we have:
\begin{theorem}
    $\mathsf{T}_2 \vdash \neg \mathsf{Comp}_\subseteq$.
\end{theorem}
However, this is clearly consistent with the weaker prinicple
\begin{description}
    \item[Comp$_\subseteq^\uparrow$] $\Box \forall x \du \exists X \bu \forall y[Xy \equiv y \subseteq x]$
\end{description}
that is, it is consistent that \emph{by height expansion} one eventually gets all the subsets 
of any given set that one can get \emph{by height expansion alone}, even if one can never get 
\emph{absolutely all} subsets of any given set.








\end{document}