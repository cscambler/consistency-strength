\documentclass{article}

\renewcommand*\familydefault{\sfdefault} 
\newcommand{\du}{\Diamond_\uparrow}
\newcommand{\dl}{\Diamond_\leftarrow}
\newcommand{\bu}{\Box_\uparrow}
\newcommand{\bl}{\Box_\leftarrow}


\usepackage[T1]{fontenc}
\usepackage{amsmath, amssymb, amsthm}
\usepackage{bussproofs}
\usepackage{tcolorbox}
\usepackage[]{multicol}
\newtheorem{lemma}{Lemma}
\newtheorem{corollary}{Corollary}
\newtheorem{definition}{Definition}
\newtheorem{theorem}{Theorem}
\newtheorem{fact}{Fact}

\title{On the Consistency Strength of Axioms for Height and Width Potentialism}
\author{Chris Scambler}
\date{\today}
\begin{document}
\maketitle
\abstract{
    There are two parameters along which the universe of sets might 
    conceivably differ in size: it may have more or fewer ordinals, 
    and respectively may naturally be construed as taller or shorter;
    or at any given ordinal height, the initial segment of the 
    universe up to that ordinal may have more or fewer subsets, 
    and the universe may in each case naturally be constured, respectively, 
    as fatter or thinner.
    A potentialist about the height of the universe believes that, no matter what,
    there could always be further ordinals; 
    a potentialist about the width of the universe believes that, no matter what, there could 
    always be more subsets of any given infinite set.
    
    
    It is well known that natural axiom systems for height potentialism 
    are equi-consistent with axiom systems related to ZFC. 
    On the other hand, although recent literature has seen some discussion of 
    height and width potentialism combined, analogous results precisely correlating 
    these axiom systems with familiar, purely quantificational theories do not yet exist.    
    
    This paper provides such results. I show that the axioms for height and width 
    potentialism from Scambler CITE exhibit the same forms of equivalence with 
    second order arithmetic extended by topological regularity principles. 
    The results are significant because they 
    helps to round out the foundational 
    picture underlying height and width potentialism, and to bring out how different it 
    is from the standard ZFC-conception of set.
}
\section{Overview}
The purpose of this paper is to present some new results on the consistency strength of 
axiomatic systems for height and width potentialism combined, and to explain 
what I take their significance to be. The results have a rather 
technical flavor and involve elements of higher order modal logic, the theory of 
forcing, and the topology of the real line. In order to maximize intuitiveness, and 
to make the paper as accessible as possible, I have therefore decided to follow a policy 
of `double elaboration': I will begin by explaining all the result of interest 
and the methods used to attain them with a broad brush, one that won't be accurate 
enough to pass contemporary standards of logical rigor but that will suffice to get
the key ideas accross in sufficient detail to sustain some philosophical discussion. 
I will then provide a rather compressed technical appendix that shows how to substanitate 
the results cited. For the reader who has read the informal presentation, I hope, this 
more compressed formal development should be sufficient.

I will begin by discussing the sense in which axiom systems for height potentialism
are equivalent to set theories related to ZFC, and the techniques used to attain 
the equivalence theorems. I then turn to discussing the relevant axioms 
for height and width 
potentialism, and outline an argument that they share the same form of equivalence with certain extensions 
of second order arithmetic. To close, I sketch some further results that apply in the 
case of height potentialism/ZFC but fail, at least in any straightforward sense, 
in the case of height+width potentialism/second order arithmetic. 
To close the main body of the paper, I discuss the relevance of these results for our understanding 
of the height/width potentialist point of view.
Finally, a pair of appendices substantiate the technical results.
\section{Warm Up: Height Potentialism}
We begin by sketching template results regarding height potentialism. 
\subsection{Axioms}
We start with a discussion of the `natural' axiom systems for height potentialism 
that are at issue.

The systems will be directed at axiomatizing the following 
intuitive conception. Imagine there is a being who is capable of taking things 
and {\bf collecting} them together into a set, and of repeatedly performing this act 
as many times as is conceivable. Say that a \emph{set} is something that, given enough 
executions of this basic act, could be produced by such a being. 

We want to axiomatize this conception in modal logic. The modal operator, $\Diamond$, 
is to be interpreted so that $\Diamond \varphi$ means
`by repeated acts of collection, $\varphi$ can be made to hold'. The dual 
operator, $\Box \varphi$, means `no matter how many acts of collection are performed, 
$\varphi$ will hold'. The axioms governing this modal logic can be taken to be 
$\mathsf{S4.2}$ with the converse Barcan formula. More details are given in Appendix A.\footnote{(We do not want $.3$, 
since given $a \not= b$ we may form either $\{a\}$ or $\{b\}$, but we do want 
$.2$,
so from either of these we can get $\{a\}$ and $\{b\}$.) The rule of necessitation 
is assumed, along with the standardly valid inference rules:
\begin{prooftree}
    \AxiomC{$\Phi_1 \rightarrow \Box (\Phi_2 \rightarrow ... \Box ( \Phi_n \rightarrow \Box \Psi) ... )$}
    \UnaryInfC{$\Phi_1 \rightarrow \Box (\Phi_2 \rightarrow ... \Box ( \Phi_n \rightarrow \Box \forall x \Psi) ... )$}
\end{prooftree}}
 
Our conception says that the being can take any \emph{things} whatsoever and collect 
\emph{them} into a set. These are plural notions, and accordingly we shall want a plural 
langauge as well. We thus take a stock of plural variables $X$ and singular variables $x$.
The circumstance of something $x$ being one of some things $X$ will be expressed by 
the concatenation $Xx$. The precise details, once more,  are consigned to Appendix A. 

The set-theoretic axioms are more interesting. We assume all the usual 
background stuff: the axiom of extensionality for sets, rigidity for the 
membership relation, and the axiom of foundation. But in addition, we have 
distinctively potentialist existence principles, the most fundamental of which 
being:

\begin{description} 
    \item[HP] $\Box \forall X \Diamond \exists x [Set(x, X)]$
\end{description}
Here and throughout, $Set(x, X)$ is an abbreviation for 
$\forall z[z \in x \equiv Xz]$; {\bf HP} thus says: \emph{necessarily, any 
things can be the elements of a set}. Under the given 
intuitive interpretation, this corresponds to the idea that the being 
can always collect together any given things into a set in any circumstances.  
(Hence, in contrast to standard set theory, there are no 
special things that somehow can't be made to form a set.)

{\bf HP} allows us (given plural comprehension) to prove the possible existence 
of each specific hereditarily finite set. But it does not allow for a proof that 
it is possible to produce any infinite sets. 

In the spirit of transfinite 
set theory, we will want to secure this possibility axiomatically. 
One way to to this uses the following {\bf Completeability} principle. 
Let $\mathbb{N}(x)$ be the assertion that $x$ is a transitive set well-ordered 
by $\in$ which, in addition, is Dedekind finite. Then we have the following 
so-called completeability principle for the nautral numbers:
\begin{description}
    \item[Comp$_\mathbb{N}$]
    $\Diamond \exists X \Box \forall y[ Xy \leftrightarrow \mathbb{N}(x)]$
\end{description}
This says that it is possible, eventually, for the being to create \emph{all possible}
natural numbers. It can be used to prove the possibility 
e.g. of $\omega$'s existence, given the other axioms.

Indeed, in conjunction with {\bf Comp$_\mathbb{N}$}, 
{\bf HP} proves the possible existence of infinitely many
infinite sets. But there is no guarantee from the axioms stated up to now that 
the modal analogue of the powerset axiom should old: that it should be possible, 
given any set, infinite or no, eventually to produce \emph{all} its possible subsets. 
We also have no guarantee that any uncountable sets are possible.

Axiomatically, these things can be imposed by appeal to another completeability axiom:
\begin{description}
    \item[Comp$_\subseteq$]
    $\Box \forall x \Diamond \exists X \Box \forall y[ Xy \leftrightarrow y \subseteq x]$
\end{description}
This time, we assert that the subsets of any given set are always `completeable': 
under our intuitive interpretation, 
it says that one can always eventually form all possible subsets of any given set. 
The axiom can easily be shown is readily seen to have the desired consequences, as 
for example proving the possibility of uncountable sets in the strong sense that 
it is necessary that there is no function defined on the natural numbers with the set 
a subset of its range.

As a final axiom, or rather axiom scheme, we throw in a modal 
translation of replacement (the $\Diamond$-translation in the sense defined below).

These are all fairly natural axioms to adopt if one wishes to explicate the idea 
of a \emph{never ending set-construction process} in modal terms. One has an axiom 
asserting that any things possibly form a set, along with various other axioms 
that implicitly concern the lengths of iterations of set formation that are possible.

For ease of reference the axioms 
for this system are compiled in Appendix A. I will refer to them as the system $\mathsf{L}$,
for their creator, \O ystein Linnebo.

We now turn to a discussion of the proof-theoretic relations between $\mathsf{L}$
and ZFC. 
\subsection{Key Results}
We will focus for now on two ``central results'': they show that the consistency 
strength of $\mathsf{L}$ is exactly that of ZFC. I will state them and indicate the methods of proof. 
Full proofs are provided in Appendix B. 


The first central result is that there is a translation $\varphi \mapsto \varphi^\Diamond$ 
from the language of second order set theory 
to the potentialist language $\mathcal{L}_0$ that preserves theoremhood 
from ZFC into $\mathsf{L}$ \emph{on first order formulas}. 

The key idea behind the translation, $\varphi \mapsto \varphi^\Diamond$,  
is for the potentialist to interpret the actualists quantifiers as tacitly modalized:
when the actualist says, for example, that `there is no set of all sets', the potentialist 
will interpret them as saying, `it is not possible for there to be a set that, necessarily, contains all sets',
which is a theorem from their point of view. Slightly more formally, 
the idea for $t_\diamond$ is to put a $\Diamond$ in front of every existential quantifier 
(and a $\Box$ in front of 
every universal). The variables get mapped to themselves, and the translation 
commutes with the connectives in the obvious way, so for example $t(\varphi \wedge \psi)$
is $t(\varphi) \wedge t(\psi)$. 

That the translation has the desired property is a theorem due to Linnebo. It makes 
use of the following central fact, which will be appealed to repeatedly. 

\begin{fact}
    Suppose all atomic predicates in a language $\mathcal{L}$ are rigid in some theory $T$,
    and that $\Box$ 
    is a modal operator of $\mathcal{L}$ with the axioms of S4.2 together with the 
    converse Barcan formula according to $T$. Then we have 
    $\Gamma \vdash \varphi$ iff $\Gamma^\Diamond \vdash_T \varphi^\Diamond$, where here 
    $\vdash$ represents provability in classical logic. 
\end{fact}
Given the fact, the prove that the proposed translation works amounts to showing that 
the modal translation of the first order axioms of ZFC hold in $\mathsf{L}$, which is a 
routine exercise. 

Why does the result only hold for first order formulas? Well, it is an equally easy 
exercise to prove that the $\Diamond$ translation of second order comprehension 
is inconsistent over the other axioms of $\mathsf{L}$. Thus, to the extent that 
the vocabulary is the same, this translation does not extend to the whole language. 
It is tempting to view this result as showing that, although there is a great amount 
of agreement between $\mathsf{L}$ and theories like ZFC, nevertheless there are substantive 
disagreements too. They are not `equivalent descriptions'.

The second central result is that there is a translation $\varphi \mapsto \varphi_\exists$ from the 
potentialist language $\mathcal{L}_0$ to the language $\mathcal{L}$ of second order 
set theory that preserves theoremhood from $\mathsf{L}$ to ZFC. Thus ZFC can 
interpret the claims of $\mathsf{L}$. This time, agreement on the full language can be 
attained, and the result is therefore a consistency proof for $\mathsf{L}$ relative to ZFC.


Let's discuss the translation $\varphi \mapsto \varphi^\exists$.
It has to translate the whole potentialist idiolect into the austere idiom of the 
second order language. How is that to be done? 

The guiding idea is in many respects a natural dual to that of the previous translation.
With the $\Diamond$ translation, the potentialist injected a modal element into 
the talk of the actualist to make their claims come good. Here, with the $\exists$ translation,
we will do the opposite: the actualist with remove the modal element and replace it 
with quantificational claims over what might naturally be thought of as `possible worlds'.
So, we define the notion of a possible world, and then interpret a potentialist assertion 
like `necessarily, any things can be the elements of a set' as saying that, for any possible world 
we choose, and any things contained in that world, there exists a more expansive possible 
world that contains a set with those things as its elements. And with the right 
definition of world, it turns out that in ZFC we can make all the assertions of 
$\mathsf{L}$ come true. 

How should the actualist define the notion of a possible world? That will depend a 
little bit on some details of the potentialist account we have left open. For example, 
does the potentialist allow that, given it is possible there are some sets $X$, it is also 
possible that the sets are exactly $X$ together with $\{X\}$? Or must 
all the subsets of any given set be introduced together? Different answers on the 
part of the potentialist will lead to different matching definitions of world on 
the part of the actualist. For simplicity, we take a maximally liberal conception of 
world: on this view, any transitive extension of the universe is a possible world 
that is reachable. (Sets are introduced `one at a time'.) Thus, our intuitive idea 
is to take possible worlds for the potentialist to be transitive sets; and 
the actualist will therfore interpret the potentialist's modal operators 
as really just quantifiers over transitive sets.

In order to accomodate these ideas, the actualist's translation of potentialist claims 
will need to be parameterized: given a formula $\varphi \in \mathcal{L}_0$, its 
translation will be a claim in $\mathcal{L}$ with an additional free variable, $w$, 
which stands for the world relative to which the claim is being evaluated. 


\subsection{Discussion}
\section{Axioms}
In this section I will give the system of axioms for height and width potentialism 
that will be the subject of the rest of the paper.

The system is directed at axiomatizing the following intuitive (if fanciful) conception.
As before, we have a being who is able to take things and {\bf Collect} them together into a set.
But now, we will also take them to be able to perform another kind of basic act: namely, 
that of taking some things, and {\bf Counting} them, that is, correlating them one-for-one
with the natural numbers. Our new theory will concern the possibilities for arbitrary iterations 
of {\bf Collect} and {\bf Count}. This time, we say that a \emph{set} is something that can be 
gotten eventually by arbitrary iterations of these operations. {\bf Collect} is a height potentialist 
principle, in that the set collected together has higher rank than the collected sets, 
while {\bf Count} is naturally understood as a width potentialist principle, 
as the introduction of an enumerating function for a given set may corresponds to the introduction 
of new sets of the same or lower rank as the sets counted in many cases--
indeed, up to isomorphism, it will correspond to introducing a subset of the naturals.

There are many natural modal operators that one can read off from this conception. 
There is for example a modal, which I will write $\du \varphi$, which corresponds to 
possibility only taking acts of {\bf Collect} into consideration. Since this is the 
height potentialist sense of possibility I shall often refer to this as {\bf vertical}
possibility. There is also a natural modal, $\dl \varphi$, which corresponds to possibility 
just by iterated enumeration. This might naturally be read as {\bf horizontal} possibility.
 And there is a general modal operator, $\Diamond \varphi$, which 
corresponds to $\varphi$s being possible by some arbitrary number of iterations and alternations 
of the two available operations. I will refer to this as {\bf general} possibility. There are 
many others besides, for example that induced by $\du\dl$, but these will play little direct role.
In fact, for our purposes, even $\dl$ will be superfluous. We will focus our attentions on 
the two operators $\du$ and $\Diamond$, where the former corresponds to possibility just by 
{\bf Collection}, and the latter to possibility by combinations of {\bf Collect} and {\bf Count}.

So we take a multi-modal language with the two modal operators $\Diamond$ and $\du$. We still 
want the capacity to talk about pluralities and so we keep the two types of variable. 
And we want to talk about sets so we include the membership relation symbol $\in$. 

What axioms should we impose? Well, we want all the same plural and first order stuff.
Both of the modal operators should have $\mathsf{S4.2}$, and
we have $\du \varphi \rightarrow \Diamond \varphi$. We take over 
all the axioms from $\mathsf{L}$ for the $\du$ modality: the idea being, nothing 
has changed from before, and if we could get something without forcing/enumeration before 
we can get it by ignoring forcing/enumeration now.
\section{Key Results}
\section{Discussion}
\section{Stuff}
Modal Set Theory:
Absolutness of various kinds.

Set Theory: 
Pi 1-1 PSP equivalent to only countably many reals in L[r].

Logic: 
Define a formula to be \emph{pseudo typed} (PT) iff 
odd numbered variables only ever occur
to the right of $\in$. 
Then in ZFC, any fomula $\varphi$ has a (weak) PT equivalent
$PT(\varphi)$ with $ZFC \vdash PT(\varphi)$ iff $ZFC \vdash \varphi$, 
and any sentence 
$\varphi$ has a (strong) PT equivalent $PT(\varphi)$ with 
$ZFC \vdash \varphi \leftrightarrow PT(\varphi)$.

Proof. Let $n(\varphi)$ be a number greater than the indices of variables in 
$\varphi$. For each $n$ let $E_n$ be an enumeration of 
the evens greater than $n$, so e.g. $E_2(0)=3$. 

We define $PT(\varphi)$ recursively. Moving from left to right in the formula 
keep looking until you find a variable out of place. Then check to see 
if it is free or bound. If it is free, replace all other free occurrences of the 
variable by $E_{n(\varphi)}(0)$; If 
it is bound, do the same for all its occurrences in the scope 
of the binding quantifier. In either case, a formula $PT_0(\varphi)$
results. Then given $PT_n(\varphi)$, do the same thing to get $PT_{n+1}(\varphi)$.
Whenever $PT_n(\varphi) = PT_{n+1}(\varphi)$, $PT_n(\varphi)$ is $PT$. We set 
such $PT_n(\varphi) := PT(\varphi)$.

$PT(\varphi)$ is provable iff $\varphi$ is. Induction on the number of out-of-place 
variables. Suppose it holds for $n$. 

\end{document}