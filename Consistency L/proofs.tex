\documentclass{article}
\renewcommand*\familydefault{\sfdefault} 
\newcommand\D{\blacklozenge}
\newcommand\B{\blacksquare}
\usepackage[T1]{fontenc}
\usepackage{amsmath, amssymb, amsthm, stmaryrd}
\usepackage{mdframed}
\usepackage{bussproofs}
\newtheorem{lemma}{Lemma}
\newtheorem{corollary}{Corollary}
\newtheorem{definition}{Definition}
\newtheorem{theorem}{Theorem}

\title{Proofs}
\author{CS}
\date{\today}
\begin{document}
\maketitle
TODO: 
Check consistency of notation: do you use $T$ for SOA$^+$? What even is the best 
name for this?

Always use mathsf for the modal theories.
\section{ZFC and L}

\subsection{Formulation of ZFC}
\subsubsection{Language $\mathcal{L}_\in$}
\paragraph{Signature}
\begin{itemize}
    \item countable infinity of first order variables $x_i$
    \item countable infinity of monadic second order variables $X_i$
    \item propositional connectives $\vee, \neg$
    \item variable binding quantifier $\exists$
    \item relation $=$
    \item relation $\in$
\end{itemize}

\paragraph{wffs}
    $$Xx | x \in y | x = y | X = Y$$
    $$\varphi \wedge \psi | \neg \varphi | \forall x \varphi | \forall X \varphi $$

\paragraph{defs}
\begin{itemize}
    \item usual defs for connectives quantifiers
    \item symbol for $\emptyset$, $\{x : \psi \}$ notation
\end{itemize}

\subsubsection{Axioms of ZFC}

Standard second order logic with full comprehension and extensional second order 
identity. For set theoretic axioms:

\begin{description}
    \item[Ext$_\forall$] $\forall x[x \in y \equiv x \in z] \supset y = z$
    \item[Fun$_\forall$] $ x \not = \emptyset \supset \exists y[y \in x \wedge y \cap x = \emptyset]$
    \item[Pair] $\exists z[z = \{x, y\}]$
    \item[Un] $\exists y [y = \bigcup{U}(x)]$
    \item[Rep] $Fun (F) \supset \forall x\exists y(y = F|x)$
\end{description}



\subsection{Formulation of L}

\subsubsection{Language $\mathcal{L}_0$}
\paragraph{Signature}
\begin{itemize}
    \item countable infinity of first order variables $x_i$
    \item countable infinity of monadic second order variables $X_i$
    \item propositional connectives $\vee, \neg$
    \item variable binding quantifier $\exists$
    \item relation $=$
    \item operator $\Diamond$
    \item relation $\in$
\end{itemize}

\paragraph{wffs}
    $$Xx | x \in y | x = y | X = Y$$
    $$\varphi \wedge \psi | \neg \varphi | \forall x \varphi | \forall X \varphi | \Diamond \varphi$$

\paragraph{defs}
\begin{itemize}
    \item usual defs for connectives quantifiers and modals
    \item $Ex$ is an abbreviation for $\exists y[y = x]$ or a more suitable alphabetic variant
    \item $Set(x, X)$ abbreviates $\forall y[Xy \equiv y \in x]$
    \item Previous abbreviations from set theory
\end{itemize}
\subsubsection{Axioms of L}
We assume all propositional tautologies, along with any standard axioms for positive free quantifier logic.

The modal logic is S4.2 with necessitation (and CBF). The rule of inference ... is also assumed.

As to the plural logic, we assume the following.
\begin{description}
    \item[pExt] $\forall X \forall Y [(\forall x[Xx \equiv Yx]) \supset X = Y]$
    \item[pR] $ \Diamond Xx \supset \Box Xx$
    \item[pBF] $\forall X[ \Diamond (\exists x[Xx \wedge x = y]) \supset \exists x [Xx \wedge x = y]]$
\end{description}
Finally, the set theoretic axioms.
\begin{description}
    \item[Ext] $\forall x \forall y [(\forall z[z \in x \equiv z \in y]) \supset x = y]$
    \item[Ele] $\Box \forall x \exists X \Box \forall y[Xy \equiv y \in x]$
    \item[Fun] $\forall x[ x \not = \emptyset \supset \exists y[y \in x \wedge y \cap x = \emptyset]]$
    \item[Set] $\Box \forall X \Diamond \exists y [Set(y, X)]$
    \item[Inf] $\Diamond \exists X\Box \forall y[Xy \equiv \mathbb{N}(y)]$
    \item[Pow] $\Box \forall x \Diamond \exists X \Box \forall y [Xy \equiv y \subseteq x]$
    \item[Rep] $Rep^\Diamond$
    \item[Min] $\forall y[ y \in x] \supset \Diamond(\forall y[y \in x] \wedge \Diamond (z \in x) \supset Ez)$ 
\end{description}

\subsection{Key Result One}
\begin{theorem} 
There is an interpretation $\exists : \mathcal{L}_0 \to \mathcal{L}_\in$ that preserves 
theoremhood from L to ZFC.
\end{theorem}
\begin{definition}[$\varphi_\exists$]
    The translation $\varphi \mapsto \varphi_\exists$ is defined by the following clauses.
    In each case, $d$ is the least variable not occuring in $\varphi$.
    \begin{itemize}
        \item $x \in y_\exists := x \in y \wedge d = d$
        \item $Xx_\exists := Xx \wedge d = d$
        \item $(\neg \varphi)_\exists := \neg \varphi_\exists(d)$
        \item $(\varphi \vee \psi)_\exists := \varphi_\exists(d) \vee \psi_\exists(d)$
        \item $(\exists x \varphi)_\exists := \exists x \in d[ \varphi_\exists(d)]$
        \item $(\exists X \varphi)_\exists := \exists X \subseteq d [ \varphi_\exists(d)]$
        \item $(\Diamond \varphi)_\exists := \exists e[e \supseteq d \wedge Tran(e) \wedge \varphi_\exists(e)]$
    \end{itemize}
    In these, $\varphi_\exists(e)$ represents the result of substituting $e$ for $d$ in
    $\varphi_\exists$.

    We then set $\exists(\varphi) := Tran(d) \supset \varphi_\exists$.
\end{definition}
\begin{proof}
    The propositional tautologies and modus ponens are obvious. For free universal 
    instantiation, we must show (under the assumptin $Tran(d)$) that
    $$\forall x \in d [ \forall y \in d (\varphi(y))_\exists (d) \supset (\varphi(x))_\exists(d)]]$$
    But this is immediate. (Note however that the unfree quantifier rule,
    which removes the initial quantifier, is not provable 
    under this interpretation.)

    As to the laws of $S4.2$ modal logic, it is completely clear that S4 will hold 
    in light of the reflexivity and transitivity of $\subseteq$. For $.2$, suppose 
    $(\Diamond \Box \varphi)_\exists$. Then there is a transitive extension $e_0$ of $d$ 
    such that every extension $f$ of $e_0$ has $\varphi_\exists(f)$. Suppose given 
    transitive extension $e$ of $d$. Then $f_1 := e \cup e_0$ is a transitive extension 
    of $e_0$ that (therefore) has $\varphi_\exists(f_1)$. 
    Hence $(\Box \Diamond \varphi)_\exists$.\footnote{
        In fact (by transitivity) we have the stronger 
        $\Diamond \Box \varphi \supset \Box \Diamond \Box \varphi$.)
    } For necessitation, we must show that if $Tran(d) \supset \varphi_\exists(d)$
    is a theorem, then so is 
    $Tran(d) \supset \forall e [e \supseteq d \wedge Tran(e) \supset \varphi_\exists(e)]$,
    which is obviously correct.

    All the plural axioms are straightforward, although it is worth 
    remarking that the interpretations become unprovable 
    (in fact demonstrably false) if the initial second order 
    quantifiers are removed. 

    On to the set-theoretic axioms. The case of extensionality reduces to the claim 
    that extensionality holds in transitive sets. Ele and Fun are equally straightforward.
    For set: suppose given a transitive extension $e$ of $d$ and a subset $X$ of $e$. Since 
    every set is an element of a transitive set (ZFC) we can extend $e$ to a transitive 
    set that contains that subset as an element, and the result follows. Inf just comes down 
    to the fact that there is a transitive set that contains the natural numbers. 
    Similarly for pow: given any set there is a transtive set that contains all its subsets.
    Each instance of $\exists(Rep^\Diamond)$ just is an instance of replacement.
\end{proof}

\subsection{Key Result Two}
\begin{theorem}
There is an interpretation $\Diamond: \mathcal{L}_\in \to \mathcal{L}_0$ that preserves theoremhood from ZFC to L \emph{on first order formulas}. 
\end{theorem}
\begin{proof}
    This is just the translation $\varphi \mapsto \varphi^\Diamond$. That the result holds 
    is a theorem of Linnebo.
\end{proof}

\subsection{Key Result Three}
\begin{theorem}
These translations yield a definitional equivalence 
between the first order fragments of ZFC and L in the following sense: for all 
$\varphi$ without second order variables, we have
\begin{enumerate}
    \item $L \vdash Univ(d) \supset (\varphi_\exists)^\Diamond \equiv \varphi$
    \item $ZFC \vdash Tran(d) \supset (\varphi^\Diamond)_\exists \equiv \varphi.$
\end{enumerate}
Here, $Univ(d)$ abbreviates 
$\forall x[x \in d] \wedge \Diamond(\exists y[y \in d \wedge y = z]) 
\supset 
\exists y[y \in d \wedge y = z]$ for 
a suitable choice of (free) $y$.
\end{theorem}

\begin{proof}
    In each case we proceed by induction on the complexity of $\varphi$.

    For 1., the base cases are all immediate, as are the propositional connectives. 
    For the quantifier, we need 
    $$L \vdash Univ(d) \supset 
    \Diamond \exists x \in d(\varphi_\exists(d)^\Diamond(x)) 
    \equiv 
    \exists x \varphi$$
    So suppose $Univ(d)$. Going right to left, our induction hypothesis 
    yields \\ $\exists x (\varphi_\exists(d)^\Diamond(x))$, and then $Univ(d)$ 
    implies $\Diamond \exists x \in d (\varphi_\exists(d)^\Diamond(x))$ as 
    required. 

    For the converse, suppose 
    $\Diamond Ex \wedge x \in d \wedge \varphi_\exists(d)^\Diamond(x)$.
    Then by $Univ(d)$ we get $Ex$. We also have $\Diamond \varphi_\exists(d)^\Diamond(x)$.
    But by a result of Linnebo this implies $\varphi_\exists(d)^\Diamond(x)$.
    By the IH we have $\varphi(x)$ and the result follows.

    Finally, for $\Diamond \varphi$, we must show that:
    $$L \vdash Univ(d) \supset 
    \Diamond \exists e [e \supseteq d \wedge Tran(e) \wedge (\varphi_\exists(e))^\Diamond]
    \equiv 
    \Diamond \varphi$$

    Suppose $Univ(d)$. 
    
    Going left to right, suppose
    $$\Diamond (Ee \wedge e \supseteq d \wedge Tran(e) \wedge (\varphi_\exists(e))^\Diamond).$$
    By rigidity, $e \supseteq d \wedge Tran(e)$. So by Min $\Diamond Univ(e)$. Since also 
    $\Diamond (\varphi_\exists(e))^\Diamond$ and the latter is rigid, we may infer 
    $\Diamond Univ(e) \wedge (\varphi_\exists(e))^\Diamond$.
    By our induction hypothesis, 
    $\Box (Univ(e) \supset (\varphi_\exists(e))^\Diamond \equiv \varphi)$.
    Hence $\Diamond \varphi$. 
    
    Going right to left,
    Suppose $\Diamond \varphi$. By comp, 
    $\Diamond \varphi \wedge EX \wedge \forall y[Xy \equiv y = y]$.
    By HP, 
    $\Diamond (\varphi \wedge \Diamond (Ee \wedge Set(e, X)))$.
    By rigidity, $\Diamond (\varphi \wedge Set(e, X))$.

    We have, as a general lemma,
    $EX \wedge \forall y[Xy \equiv y = y] \wedge Set(x, X) \supset Univ(x)$.
    For, suppose $Ez$. Then since $z = z$ we get $Xz$ and hence $z \in x$. So 
    $\forall z[z \in x]$. Then suppose $\Diamond \exists y[y \in x \wedge y = z]$. 
    Then $\Diamond \exists y[Xy \wedge y = z]$. Since $EX$ it follows that $Ez$, 
    and of course still $y \in x$ as required.

    It follows that $\Diamond (\varphi \wedge Univ(e))$. By the IH, 
    $\Box (Univ(e) \supset ((\varphi(e)_\exists)^\Diamond \equiv \varphi))$. Hence,
    since $Univ(e) \supset Tran(e)$, we have 
    $\Diamond (\exists e[e  \supseteq d \wedge Tran(e) \wedge (\varphi(e)_\exists)^\Diamond])$,
    as required.

    

\end{proof}

\section{SOA$^+$ and M}
\subsection{Formulation of SOA$^+$}
\subsubsection{Language of SOA$^+$}
The language of SOA is just $\mathcal{L}_\in$.
\subsubsection{Axioms of SOA$^+$}
The axioms of SOA$^+$ are those of ZFC without the axiom 
of power, with replacement possibly reformulated as 
collection, and with the $\Pi_1^1$ PSP.

Important lemma: $L[r] \vDash ZFC$ for every real $r$ in this theory.
\subsection{Formulation of M}
\subsubsection{Language of M}
The language $\mathcal{L}_1$ of $M$ is $\mathcal{L}_0$ closed
under the operator $\blacklozenge$.
\subsubsection{Axioms of M}
We assume all propositional tautologies, 
along with any standard axioms for positive free quantifier logic.

The modal logic is S4.2 with necessitation (and CBF) for both operators. 
The rule of inference ... is also assumed for both operators.

The only axiom on the combination of modals we need is 
\begin{description}
    \item[W] $\D \varphi \supset \Diamond \varphi$. 
\end{description}

As to the plural logic, we assume the following.
\begin{description}
    \item[pExt] $\forall X \forall Y [(\forall x[Xx \equiv Yx]) \supset X = Y]$
    \item[pR] $ \Diamond Xx \supset \Box Xx$
    \item[pBF] $\forall X[ \Diamond (\exists x[Xx \wedge x = y]) \supset \exists x [Xx \wedge x = y]]$
\end{description}
Finally, the set theoretic axioms.
\begin{description}
    \item[Ext] $\forall x \forall y [(\forall z[z \in x \equiv z \in y]) \supset x = y]$
    \item[Ele] $\Box \forall x \exists X \Box \forall y[Xy \equiv y \in x]$
    \item[Fun] $\forall x[ x \not = \emptyset \supset \exists y[y \in x \wedge y \cap x = \emptyset]]$
    \item[Set] $\Box \forall X \D \exists y [Set(y, X)]$
    \item[Inf] $\D \exists X\Box \forall y[Xy \equiv \mathbb{N}(y)]$
    \item[Pow] $\Box \forall x \D \exists X \B \forall y [Xy \equiv y \subseteq x]$
    \item[Rep] $Rep^\Diamond$, $Rep^\D$
    \item[Min2] Necessarily, either all sets are necessarily constructible, or else $\exists r[\B \forall x[x \in L[r]]$
    \item[Min1] $Univ(d_1) \wedge vUniv(d_2) \wedge Tran(e_1) \wedge d_1 \subseteq e_1 \wedge e_2 \geq d_2 \supset \Diamond Univ(e_1) \wedge vUniv(e_2)$ 
\end{description}
In the latter, $vUniv(r)$ abbreviates $\B \forall x [x \in L[r]] \wedge \Box \forall x[x \in L[r] \supset \D Ex]]$.
\subsection{Necessary Lemmas}
\begin{lemma}[Mirroring]
    Let $\Gamma 
    \cup {\phi}$ be a set of formulas in the first order language of set theory. Then
    $\Gamma \vdash \phi$ in first order logic if and only if 
    $\Gamma^\Diamond \vdash_\mathsf{M} \phi^\Diamond$. 
    The same is true for $\D$.
\end{lemma}
\begin{proof}
    See cite.
\end{proof}

\begin{lemma}[Bounded Modal Absoluteness]\label{BMA}
    Let $\phi$ be 
    a formula in the first order language of set theory
    with only bounded quantifiers. Then
    $\mathsf{M} \vdash \phi \leftrightarrow \phi^\Diamond$.
    The same is true for $\D$.
\end{lemma}
\begin{proof}
    An induction on the complexity of $\phi$. See Lemma 12.2 of Linnebo.
\end{proof}
\begin{lemma}[$\Delta_1$ Absoluteness]\label{DA}
    Let $\phi$ be a formula that is provably $\Delta_1$ over $ZFC$ without 
    powerset. Then we have: 
    \[\mathsf{M} \vdash \phi^{\Diamond} \leftrightarrow \phi^{\D}\]
\end{lemma}
\begin{proof}
    Temporarily let $T$ be the theory in question.
    Note that $T^\Diamond$ and $T^{\D}$ are each contained in $\mathsf{M}$.

    It follows from our assumption that there are $\Delta_0$ formulas $\psi, \theta$
    for which
    \[\mathsf{M} \vdash \phi^\Diamond \leftrightarrow 
    \Box \forall x \psi^\Diamond \leftrightarrow 
    \Diamond \exists x \theta^\Diamond \] 
    and similarly
    \[\mathsf{M} \vdash \phi^{\D} \leftrightarrow 
    \B \forall x \psi^{\D} \leftrightarrow 
    \D \exists x \theta^{\D} \] 
    by the previous lemma, these are each equivalent to 
    \[\mathsf{M} \vdash \phi^\Diamond \leftrightarrow 
    \Box \forall x \psi \leftrightarrow 
    \Diamond \exists x \theta \] 
    \[\mathsf{M} \vdash \phi^{\D} \leftrightarrow 
    \B \forall x \psi \leftrightarrow 
    \D \exists x \theta \] 
    respectively. But weakening implies that 
    $\D \exists x \theta \rightarrow \Diamond \exists x \theta$
    and similarly that $\Box \forall x \psi \rightarrow \B \forall x \psi$.
    The result follows.
\end{proof}

\begin{theorem}
    $\mathsf{M}$ interprets ZFC under the $\D$-translation.
\end{theorem}
\begin{proof}
    As with cite, this is a straightforward modification of 
    Linnebo's argument from cite. It is sufficient to prove the $\D$ 
    translations of axioms of ZFC. For example, for the powerset 
    axiom, use \eqref{comps2} and \eqref{hpot}.
\end{proof}

\begin{theorem}
    $\mathsf{M}$ interprets ZFC$^-$ under the $\Diamond$-translation.
\end{theorem}
\begin{proof}
    Essentially the same as the previous. 
\end{proof}
\begin{theorem}\label{hcount}
    $\mathsf{M}$ proves every set is (hereditarily) countable under the 
    $\Diamond$ translation.
\end{theorem}
\begin{proof}
    See cite.
\end{proof}
\begin{corollary}\label{soa}
    $\mathsf{M}$ interprets second order arithmetic under the $\Diamond$ translation.
\end{corollary}
\begin{proof}
    $ZFC^- + V = HC$ is definitionally equivalent to SOA. See Krapf, Simpson.
\end{proof}
\begin{lemma}[SOA]\label{sol}
    If there are only countably many reals constructible from $r$, then the $\Pi_1^1[r]$-$PSP$ holds.
\end{lemma}
\subsection{Key Result 4}
\begin{theorem} 
    There is an interpretation $\exists : \mathcal{L}_1 \to \mathcal{L}_\in$ that preserves 
    theoremhood from M to SOA.
    \end{theorem}
    \begin{definition}[$\varphi_\exists$]
        The translation $\varphi \mapsto \varphi_\exists$ is defined by the following clauses.
        In each case, $d_1, d_2$ are the least variables not occuring in $\varphi$.
        \begin{itemize}
            \item $x \in y_\exists := x \in y \wedge \bigwedge_i d_i = d_i$
            \item $Xx_\exists := Xx \wedge \wedge \bigwedge_i d_i = d_i$
            \item $(\neg \varphi)_\exists := \neg \varphi_\exists(d_1, d_2)$
            \item $(\varphi \vee \psi)_\exists := \varphi_\exists(d_1, d_2) \vee \psi_\exists(d_1, d_2)$
            \item $(\exists x \varphi)_\exists := \exists x \in d_1[ \varphi_\exists(d_1, d_2)]$
            \item $(\exists X \varphi)_\exists := \exists X \subseteq d_1 \wedge X \in L[d_2] [ \varphi_\exists(d_1, d_2)]$
            \item $(\D \varphi)_\exists := \exists e[e \supseteq d_1 \wedge Tran(e) \wedge e \in L[d_2] \wedge \varphi_\exists(e, d_2)]$
            \item $(\Diamond \varphi)_\exists := \exists e_1, e_2[d_2 \leq e_2 \wedge d_1 \subseteq e_1 \in L[e_2] \wedge Tran(e_1) \wedge \varphi_\exists(e_1, e_2)]$
        \end{itemize}
        In these, $\varphi_\exists(e_1, e_2)$ represents the result of substituting $e_1, e_2$ for $d_1, d_2$ in
        $\varphi_\exists$.
    
        We then set $\exists(\varphi) := Tran(d_1) \wedge Real(d_2) \supset \varphi_\exists$.
    \end{definition}
    \begin{proof}
        The propositional tautologies and quantifier logic stuff is straightforward.
    
        The laws of S4.2 for $\D$ hold for similar reasons to those given 
        in the proof of key result 1. 

        As for $\Diamond$, it is again easy to see the axioms of S4 come good.
        For $.2$, suppose 
        $(\Diamond \Box \varphi)_\exists$. Then there is a real $e_2$ constructible from
        $d_2$ and transitive $e_1$, a superset of $d_1$ and element of $L[e_2]$,
        such that $ (\Box \varphi)_\exists(e_1, e_2)$.
        That in turn means that for every real $e$ constructible from $e_2$, and 
        every transitive 
        set $d$ containing $e_1$ and an element of $L[e]$, we have 
        $\varphi_\exists(d, e)$.

        So suppose given arbitrary $d, e$ with $d_2$ constructible from $e$, 
        $d$ a superset of $d_1$ and an element of $L[e]$. We must find a real $r$ 
        constructible from $e$ and transitive set $t$ a superset of $d$ and element 
        of $L[r]$ such that $\varphi_\exists(t, r)$. The obvious candidates are 
        $t:= e_1 \cup d$, and $r := e_2*e$. And it is not hard to see they 
        have the required features.

        All the plural axioms are straightforward.
    
        On to the set-theoretic axioms. The case of extensionality reduces to the claim 
        that extensionality holds in transitive sets. Ele and Fun are equally straightforward.
        
        For set: suppose given a transitive extension $e$ of $d_1$ in $L[d_2]$,
        and a subset $X$ of $e$ that is also an element of $L[d_2]$. Since $L[d_2]$
        satisfies ZFC, it has that 
        every set is an element of a transitive set in $L[d_2]$.
        Thus we can extend $e$ to a transitive 
        set $e_1$ in $L[d_2]$ that contains $X$ as an element, and the result follows. 
        Inf is pretty much the same as before.
        
        For Pow, suppose given $d_4 \geq d_2$ and $d_3 \supseteq d_1$, $d_3 \in L[d_4]$,
        and $x \in d_3$. The needed result then follows from the powerset axiom in 
        $L[d_4]$.

        Replacement follows from replacement, once in $L[r]$ and once just straight up.
    \end{proof}
\subsection{Key Result 5}
\begin{theorem}
    There is an interpretation $\Diamond: \mathcal{L}_\in \to \mathcal{L}_1$ that preserves theoremhood from SOA$^+$ to M 
    \emph{on first order formulas}. 
\end{theorem}
\begin{proof}
    Using Lemma \ref{sol} and the mirroring theorem, to establish 
the desired conclusion it is sufficient to show the $\Diamond$-translation 
of the hypothesis of Lemma \ref{sol}. That is: assuming given an arbitrary real $r$, 
we must show that is possible to produce a function on the natural numbers such that, 
necessarily, if $s$ is a real constructible from $r$, 
then $s$ is in the range of $f$.

The strategy for doing so is simple: we first show that, given any real $r$, 
it is possible to produce the set of all possible reals constructible from $r$ in 
$M$ (in the sense of the modality $\Diamond$). We then invoke Lemma \ref{hcount}, which 
says that all sets are countable in $\mathsf{M}$ (with respect the $\Diamond$ translation)
to get the result.

In more detail,
we first observe that by the $\D$-translation of ZFC, 
we the $\D$ translation of the assertion that for any real $r$, the 
set of reals constructible from $r$ exists, namely:
\begin{equation}\label{dutrans}
    \D \exists x [\mathbb{R}^{L[r]}(x)^{\D}]
\end{equation}
Where in \eqref{dutrans} $\mathbb{R}^{L[r]}(x)^{\D}$ is the $\D$-translation of the first order 
formula asserting that $x$ 
is the set of reals constructible from $r$. The problem now is to derive from this 
that 
\begin{equation}\label{dtrans}
    \Diamond \exists x [\mathbb{R}^{L[r]}(x)^{\Diamond}]
\end{equation}
since it is only with respect $\Diamond$, and not $\D$, that we have Theorem \ref{hcount}.

To move from \eqref{dutrans} to \eqref{dtrans} we use the absoluteness 
lemma \ref{DA}. The formula asserting that $x$ is $L[r]_\alpha$ is $\Delta_1$ 
(in parameters $r$, $\alpha$) 
over ZFC without power. Thus, for any $x$, $\alpha$, $r$, 
\begin{equation}\label{butt}\mathsf{M} \vdash (x = L[r]_\alpha)^\Diamond \leftrightarrow
(x = L[r]_\alpha)^{\D}.\end{equation}
A simple induction shows that for any ordinal $\alpha$, $\Diamond \exists x[ x = \alpha]$ if and only if 
$\D \exists x[x = \alpha]$.
Moreover, standard set-theoretic reasoning (together with mirroring) 
implies that $\mathbb{R}^{L[r]}$ exists is equivalent to $\omega_1^{L[r]}$
exists, relative to either modality. 
We show that $(\omega_1^{L[r]} \text{ exists})^{\D}$ is equivalent to 
$(\omega_1^{L[r]} \text{ exists })^\Diamond$, 
in contrast of course to the real $\omega_1$.

For the non-trivial direction, suppose $\omega_1^{L[r]} \text{ exists}^{\D}$. 
This is equivalent to there being \emph{$L[r]$-uncountable ordinals relative to $\D$}, in the sense that
\[\D \exists \alpha \in L[r]^{\D} \B \forall f \in L[r]^{\D}[\neg(f : \alpha \twoheadrightarrow \mathbb{N})].\]
So assume the latter and instantiate such an $\alpha$ (using the rule on p $N$). 
By lemma \ref{DA}, $(\alpha \in L[r])^\Diamond$. Suppose $\alpha$ is not
$L[r]$-uncountable relative to $\Diamond$, i.e. 
$\Diamond \exists f \in L[r]^\Diamond[f : \alpha \twoheadrightarrow \mathbb{N}]$
This implies 
$\Diamond \exists \beta, f \in L[r]_\beta^\Diamond[f : \alpha \twoheadrightarrow \mathbb{N}]$. 
But then using \eqref{butt} and the fact that $\D \beta \text{ exists}$ we may infer
\[\D \exists \beta, f \in L[r]_\beta^{\D}[f : \alpha \twoheadrightarrow \mathbb{N}]\]
contradicting our assumption that $\alpha$ \emph{is $L[r]$-uncountable}$^{\D}$.

We may thus infer \eqref{dtrans} from \eqref{dutrans}: \eqref{dutrans} is equivalent to 
$(\omega_1^{L[r]} \text{ exists})^{\D}$, which is equivalent to 
$(\omega_1^{L[r]} \text{ exists })^\Diamond$,
which latter is equivalent to \eqref{dtrans} by standard set theory and mirroring. 

But now since we also have Theorem \ref{hcount}, it follows that 
\begin{equation}
    (\exists x[\mathbb{R}^{L[r]}(x) \wedge \exists f : \mathbb{N} \twoheadrightarrow x])^\Diamond
\end{equation}
which is just the $\Diamond$-translation of the claim that there are only countably many 
reals constructible from $r$. Hence, by an application of mirroring, we conclude $\Pi_1^1[r]$-$PSP$.
The result follows.

\end{proof}
\subsection{Key Result 6}
\begin{theorem}
    These translations yield a definitional equivalence 
    between the first order fragments of SOA$^+$ and $M$ in the following sense: for all 
    $\varphi$ without second order variables, we have
    \begin{enumerate}
        \item $M \vdash Univ(d_1) \wedge vUniv(d_2) \supset (\varphi_\exists)^\Diamond \equiv \varphi$
        \item $SOA^+ \vdash Tran(d_1) \wedge Real(d_2) \supset (\varphi^\Diamond)_\exists \equiv \varphi.$
    \end{enumerate}
    \end{theorem}
    \begin{proof}
        Start with 1.
        The base cases are all immediate, as are the propositional connectives. 
        For the quantifier, we need 
        $$M \vdash Univ(d_1) \wedge vUniv(d_2) \supset 
        \Diamond \exists x \in d_1(\varphi_\exists(d_1, d_2)^\Diamond(x)) 
        \equiv 
        \exists x \varphi$$
        So suppose $Univ(d_1) \wedge vUniv(d_2)$. 
        Going right to left, our induction hypothesis 
        yields $\exists x (\varphi_\exists(d_1, d_2)^\Diamond(x))$, 
        and then $Univ(d_1)$ 
        implies $\Diamond \exists x \in d_1 (\varphi_\exists(d_1, d_2)^\Diamond(x))$ as 
        required. 
    
        For the converse, suppose 
        $\Diamond Ex \wedge x \in d_1 \wedge \varphi_\exists(d_1, d_2)^\Diamond(x)$.
        Then by $Univ(d_1)$ we get $Ex$. 
        We also have $\Diamond \varphi_\exists(d_1, d_2)^\Diamond(x)$.
        But by a result of Linnebo this implies $\varphi_\exists(d_1, d_2)^\Diamond(x)$.
        By the IH we have $\varphi(x)$ and the result follows.
    
        For $\Diamond \varphi$, we must show that:
        $$M \vdash Univ(d_1) \wedge vUniv(d_2) \supset $$
        $$\Diamond \exists e_1, e_2[d_2 \leq e_2 \wedge d_1 \subseteq e_1 \in L[e_2] \wedge Tran(e_1) \wedge (\varphi_\exists(e_1, e_2))^\Diamond]$$
        $$\equiv $$
        $$\Diamond \varphi$$
    
        So suppose $Univ(d_1)$ and $vUniv(d_2)$. 
        
        Going left to right, suppose
        $$\Diamond (Ee_1 \wedge Ee_2  \wedge d_2 \leq e_2 \wedge d_1 \subseteq e_1 \in L[e_2] \wedge Tran(e_1) \wedge (\varphi_\exists(e_1, e_2))^\Diamond].$$
        By rigidity, $Tran(e_1) \wedge d_1 \subseteq e_1 \wedge e_2 \geq d_2$. 
        So by Min1 $\Diamond Univ(e_1) \wedge Univ(e_2)$.
        Since
        $\Diamond (\varphi_\exists(e))^\Diamond$ and this is rigid, we may infer 
        $\Diamond Univ(e_1) \wedge vUniv(e_2) \wedge (\varphi_\exists(e_1, e_2))^\Diamond$.
        By our induction hypothesis, 
        $$\Box (Univ(e_1) \wedge vUniv(e_2) \supset (\varphi_\exists(e_1, e_2))^\Diamond \equiv \varphi).$$
        Hence $\Diamond \varphi$. 
        
        Going right to left,
        Suppose $\Diamond \varphi$. By comp, 
        $\Diamond \varphi \wedge EX \wedge \forall y[Xy \equiv y = y]$.

        By HP and Min2, 
        $\Diamond (\varphi \wedge Ee_2 \wedge vUniv(e_2) \wedge \Diamond (Ee_1 \wedge Set(e_1, X)))$.
        By rigidity, $\Diamond (\varphi \wedge Ee_2 \wedge vUniv(e_2) \wedge Set(e_1, X))$.
    
        As before, it follows that $\Diamond (\varphi \wedge Ee_2 \wedge vUniv(e_2) \wedge Univ(e_1))$. By the IH, 
        $\Box (Univ(e_1) \wedge vUniv(e_2) \supset ((\varphi(e_1, e_2)_\exists)^\Diamond \equiv \varphi))$.  Thus 
        $\Diamond \varphi(e_1, e_2)_\exists^\Diamond$, and it is tedious but routine to show that the other 
        conditions are necessary of $e_1$ and $e_2$ using rigidity properties. ($e_1 \leq e_2$ follows 
        from the fact that $e_1$ exists when $e_2$ is vertically universal.)
    
        The most difficult case is $\B$. We need
    
        $$M \vdash Univ(d_1) \wedge vUniv(d_2) \supset $$
        $$\Diamond \exists e[e \supseteq d_1 \wedge Tran(e) \wedge e \in L[d_2] \wedge \varphi_\exists(e, d_2)^\Diamond]$$
        $$\equiv $$
        $$\D \varphi$$

        Suppose $Univ(d_1) \wedge vUniv(d_2)$.

        Going left to right, if 
        $$\Diamond (Ee \wedge e \supseteq d_1 \wedge Tran(e) \wedge e \in L[d_2] \wedge \varphi_\exists(e, d_2)^\Diamond)$$
        Then using Min1 and rigidity we may infer 
        $$\Diamond ( e \supseteq d_1 \wedge Univ(e) \wedge e \in L[d_2] \wedge \varphi_\exists(e, d_2)^\Diamond)$$
        ((HERE YOU NEED TO WRITE IN ANOTHER ANNOYING PRINCIPLE. I THINK YOU SHOULD JUST TANK THIS.
        THE PRINCIPLE YOU NEED TO WRITE IN IS THAT IF IT IS VERTICALLY POSSIBLE THERE IS SOME TRANSITIVE 
        EXTENSION OF THE UNIVERSE, THEN ITS VERTICALLY POSSIBLE THAT IT IS UNIVERSAL.
        BUT IT'S JUST GETTING RIDICULOUS.))
        the IH provides 
        $$\Box (Univ(e) \wedge vUniv(d_2) \supset \varphi_\exists(e, d_2)^\Diamond \equiv \varphi)$$
        hence 
        $$\Diamond ( e \supseteq d_1 \wedge Univ(e) \wedge e \in L[d_2] \wedge \varphi_\exists(e, d_2)^\Diamond)$$



    \end{proof}
    
\end{document}