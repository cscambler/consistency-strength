\documentclass{article}

\renewcommand*\familydefault{\sfdefault} 
\newcommand\D{\circledbar}
\newcommand\B{\boxbar}


\usepackage[T1]{fontenc}
\usepackage{amsmath, amssymb, amsthm, txfonts, stmaryrd}
\usepackage{bussproofs}
\usepackage{natbib}

\newtheorem{theorem}{Theorem}
\newtheorem{definition}[theorem]{Definition}
\newtheorem{proposition}[theorem]{Proposition}
\newtheorem{lemma}[theorem]{Lemma}
\newtheorem{fact}[theorem]{Fact}
\newtheorem{corollary}[theorem]{Corollary}

\title{On the Consistency of Height and Width Potentialism\footnote{
This article is an early draft of a paper for The Palgrave Companion to the Philosophy of Set Theory.}}
\author{Chris Scambler}
\date{\today}
\begin{document}
\maketitle
\abstract{
    Recent work in philosophy of set theory has furnished 
    some arguments that height and width potentialism 
    are inconsistent with one another. 
    One such argument can be found in this volume \citep{EB2022}; 
    the other is forthcoming \citep{SR2022}.

	At the same time, others have suggested 
    there may be some merit in the combination of 
    height  and width  potentialism. Such authors 
    have presented views that appear to manifest 
    the combination in a non-trivial way, and have 
    defended philosophical claims on their basis 
    (e.g. that all sets are ultimately countable 
    -- cf \citep{BW2022}, \citep{TM2015}, 
    \citep{CS2021}, \citep{AP2020}).  

	Clearly there is a tension here. 
    The business of this article is to explain 
    (what I take to be) its solution. 
    I will argue that height  and width  potentialism 
    are compatible, and suggest that there 
    is hope for an attractive view in the foundations of 
    mathematics that arises from their combination. 
    I will do this by explaining that view and how it 
    responds to the arguments alleging inconsistency.
    Along the way I will present some new results 
    relating height and width potentialism to 
    extensions of second order arithmetic by regularity principles.

}
\section{Overview}


Recent work in philosophy of set theory has furnished 
    some arguments that height and width potentialism 
    are inconsistent with one another. 
    One such argument can be found in this volume \citep{EB2022}; 
    the other is forthcoming \citep{SR2022}.

	At the same time, others have suggested 
    there may be some merit in the combination of 
    height  and width  potentialism. Such authors 
    have presented views that appear to manifest 
    the combination in a non-trivial way, and have 
    defended philosophical claims on their basis 
    (e.g. that all sets are ultimately countable 
    -- cf \citep{BW2022}, \citep{TM2015}, 
    \citep{CS2021}, \citep{AP2020}).  

	Clearly there is a tension here. 
    The business of this article is to explain 
    (what I take to be) its solution.  
I will argue that height  and width  potentialism 
are compatible, and hence that the programs 
cited above are in good standing. 
I will do this by developing some 
axiomatic theories that explicate height 
and width potentialism, and explaining how a proponent 
of such theories might reply to the arguments alleging inconsistency.
Along the way, we will see how the two arguments 
at issue do nevertheless serve a useful purpose, 
in helping to bring out distinctive philosophical 
commitments of height and width potentialism,
commitments in 
set theory and plural logic. I will also present some new 
results relating height and width potentialism to extensions of 
second order arithmetic by regularity principles.

The plan is as follows. 
Section 2 gives some definitional background and context. 
Section 3 then presents some axiomatic systems that purport to 
combine height  and width  potentialism and discusses
issues of relevant to consistency, citing new results
on strength and equivalence to standard `actualist' 
set theories. Section 4 presents the 
inconsistency argument of Roberts and explains 
how a proponent of height and width potentialism should reply.
Section 5 does the same for Brauer.

\section{Background}
Let me start be explaining the terms. 

For present purposes, \emph{potentialism} in set theory 
will be the idea that there could always be 
more sets than there in fact are. 
In slogan form, for the potentialist,  
the universe of sets is `indefinitely extendible'. 

\emph{Height potentialism} is the idea that the universe
is always extendible `upwards' to include 
new sets of higher rank than any given ones. 
\emph{Width potentialism}, on the other hand, 
is the idea that the universe is always extendible 
`outwards', to contain new sets of no greater rank 
than the max of any given ones.

Historically, height potentialism has been motivated 
by considerations involving the set-theoretic paradoxes. 
Russell's paradox shows there is no set of all 
non-self-membered sets, and hence that the cumulative 
hierarchy of all sets is itself (therefore) not a set. 
But the height potentialist complains that any 
`stopping point' for the cumulative hierarchy 
would be arbitrary. Surely there is no conceptual obstacle 
to any particular collection of ranks of the cumulative 
hierarchy providing `urelements' for a longer continuation.

It is natural to
explicate this idea of an indefinitely extendible 
universe of sets in modal logic. 
Indeed, modal axiom systems based around core height potentialist 
ideas are known to exhibit tight forms of 
equivalence with standard iterative set theories like ZFC.

Width potentialism has been historically 
less popular, although has been attracting 
some attention over the last few decades as a 
way understand the independence phenomenon in set 
theory. There is, in fact, a close structural 
similarity in the motivation for height  and 
width  potentialism in these terms (cf Meadows). 
Just as the height potentialist begins with the 
intuition that it is arbitrary that there should be 
some ranks of the cumulative hierarchy that somehow 
inherently can't extended to include further sets, 
the width  potentialist may begin with the idea that 
it is arbitrary that there should be some universe of 
sets that cannot be extended by forcing over its 
partial orders. Just as in the previous case, the 
mathematics of forcing seems to lead us to believe 
there is at least no conceptual obstacle to making 
sense of such `forcing extensions of the universe'.

Again, as with height potentialism, a natural way for the 
width  potentialist to formalize their view 
involves modal logic: one formulates an axiom to the 
fact that, for any partial order $\mathbb{P}$, it is possible to 
find a generic for $\mathbb{P}$. Such explicitly 
axiomatic approaches 
to width  potentialism are not very well-studied: 
the focus of most work in this area has been on model 
theory. Nevertheless, axiomatic approaches are possible 
and easy enough to formulate. 

There are, in any event, clear analogies 
in the cases for height  and width  potentialism. 
In each case one has a central inexistence result 
in first order set theory (Russell's paradox, the 
proof that some partial orders do not admit generic filters) 
and one seeks to overcome it, after a fashion, by 
implementing a modalized version -- any things can 
form a set in the first case, any partial order can 
be forced over in the second. Indeed, it is natural to think 
that going potentialist one way might give you some 
reason to consider going potentialist the other way too. 

\section{Height and Width Potentialism Combined}

In this section, I will present some axiom systems that seem to 
explicate the intuitive idea of height and width potentialism, as 
described above.

\subsection{Core Logical Principles}
All the versions of height and width potentialism we will consider will be built around
two principles: the first is the height potentialist principle that any things can be 
the elements of a set; the second is the width potentialist `forcing axiom', that 
a generic filter can be found for any partial order. 

Both potentialist principles involve modality; the first also involves plural 
quantification. Accordingly the language $\mathcal{L}_0$
we use to formulate these theories 
will have at least a modal operator $\Box$, singular variables $x_n$, plural variables 
$X_n$, the propositional connectives and quantifiers, the identity symbol $=$ 
and the symbol $\in$ for set membership. The circumstance that some thing $x$ is 
one of some things $X$ will be represented by the concatenation $Xx$. Identity 
is only well formed between terms of the same type (singular/plural). 

The following core axioms will be included in all the potentialist theories we 
will go on to discuss.
\begin{description}
    \item[FQ] Standard rules for free quantifier logic 
    with identity for each type of variable.
    \item[Mod] The modal logic $\mathsf{S4.2}$ with converse Barcan formula,
    necessitation, and standard rules for universal quantification within the scope of $\Box$.  
    \item[P-ext] $\Box \forall x [\Diamond Xx \equiv \Diamond Yx] \supset X = Y$
    \item[PR] The plural rigidity axioms 
    \begin{enumerate} 
        \item $ \Diamond Xx \rightarrow \Box Xx$ 
        \item $ \Diamond (\exists x Xx \wedge x = y) \rightarrow \exists x Xx \wedge x = y$.
    \end{enumerate}
    \item[Comp] For any formula $\varphi(x, y, Y)$, 
    $\Box \forall y \Box \forall Y \Box \exists X \forall x [Xx \leftrightarrow \varphi(x, y, Y)]$ is an axiom.
    \item[P-Choice] A plural version of the axiom of choice.
    \item[M-Rep] A modal version of the axiom scheme of replacement.
    \item[Inf] An axiom saying that the set of natural numbers exists.
    \item[Set-Ext] The axiom of extensionality in the form 
    $\Box \forall x [\Diamond x \in y \equiv \Diamond x \in z] \supset y = z $
    \item[Sets] The axiom of foundation, and an axiom 
      $\Box \forall x \exists X \Box \forall y[Xy \leftrightarrow y \in x]$ asserting the rigidity of set membership
\end{description}
These axioms provide the basic prinicples of plural logic and set theory that will 
support the exploration of various potentialist axioms of set existence. The axiom 
of infinity is assumed in its standard (rather than potentialist) form just as a simplifying 
measure; all the phenomena we are interested in occur at the level of infinite sets.

We now turn to the development of potentialist axiom systems over the core logic.

\subsection{Simple HWP}
In this section I will present what seems to me to be the simplest formal combination 
of height and width potentialism. The theory is not mathematically strong, being 
exactly equivalent in strength to second order arithmetic. But it does offer a clean 
and simple proof of concept for the combination of height and width potentialism, 
as well as a useful starting point for further extensions.

The first axiom we add is:
\begin{description}
    \item[HP] $\Box \forall X \Diamond \exists x \forall y[y \in x \leftrightarrow Xy]$
    \begin{itemize} 
    \item \emph{Any things can be the elements of a set.}
    \end{itemize}
\end{description}
HP enshrines the core idea of height potentialism, since it implies (given the
usual notion of rank) that given any sets one can find others of still higher rank.

The second requires a little more preparation. In our background logic we 
can define the notion of a partial order, the notion of being a filter on 
a partial order, and the notion of being a dense set in a partial order in 
the standard way. Let blackboard variables $\mathbb{P}, \mathbb{Q}$ range over 
partial orders. (These are singular variables.) Let $D(X, \mathbb{P})$ mean 
that all $X$s are dense in $\mathbb{P}$. Finally, let $F_\cap(x, \mathbb{P}, X)$ 
mean that $x$ is a filter on $\mathbb{P}$ that intersects every one of the $X$s.
Then our width potentialist axiom can be stated as:
\begin{description}
    \item[WP] $\Box \forall \mathbb{P} \forall X D(X, \mathbb{P}) \rightarrow \Diamond \exists g(F_\cap(g, \mathbb{P}, X)) $   \begin{itemize} 
    \item \emph{Any partial order can be used in forcing.}
    \end{itemize}
\end{description}
WP is one way to flesh out width potentialism, since it implies that any 
sufficiently rich pluarlity of sets
sets may be extended to include ones of rank no greater than their max by 
forcing. 

The following will be useful to us going forward. In it, $f : \mathbb{N} \twoheadrightarrow X$
is an abbreviation for the assertion that $f$ is a function on $\mathbb{N}$ with every $X$ in its range.

\begin{proposition}\label{SHWP0}
    Let Count  be the principle:
    $$\Box \forall X \Diamond \exists f[f : \mathbb{N} \twoheadrightarrow X] $$ 
    then Count is equivalent to WP over the core logic + HP.
\end{proposition}
\begin{proof}
    See \cite{CS2021}.
\end{proof}
Let us call the result of adding these HP and either WP or Count
 to the core logic 
 Simple Height and Width Potentialism, or SHWP.

Turning now to questions of consistency, 
one can prove that SHWP is 
consistent relative to Second Order Arithmetic (SOA), and that 
in fact SHWP itself interprets SOA back under a natural translation.
The theories are (close to) `mutually interpretable', as the jargon goes.

(Here and below, we understand SOA under the guise of 
ZFC without power + all sets are countable. This is 
definitionally equivalent to the more standard
arithmetical formulations; see e.g. Simpson.)

Let $\mathcal{L}_\in$ be the first order 
language of set theory. Then:
\begin{theorem}\label{SHWP1}
    There is a map $\cdot^\Diamond : \mathcal{L}_\in \to \mathcal{L}_0$ that preserves 
    theoremhood from SOA to SHWP.
\end{theorem}
\begin{theorem}\label{SHWP2}
    There is a map $\cdot^\exists : \mathcal{L}_0 \to \mathcal{L}_\in$ that preserves 
    theoremhood from SHWP to
    SOA.
\end{theorem}
Theorem \ref{SHWP1} says that SHWP interprets SOA.
The translation $\varphi \mapsto \varphi^\Diamond$ proceeds by prepending every 
universal quantifier in $\varphi$ by a $\Box$ and every existential by a $\Diamond$.
A result due to Linnebo says that for $\varphi \in \mathcal{L}_\in$, 
\[\Gamma \vdash \varphi \Leftrightarrow \Gamma^\Diamond \vdash_{Core} \varphi^\Diamond\]
where $\Gamma^\Diamond$ has the obvious meaning of $\{\gamma^\Diamond : \gamma \in \Gamma\}$.
It thus suffices to prove $\varphi^\Diamond$ for each axiom of SOA in SHWP, 
which (using Proposition \ref{SHWP0}) is not difficult. 

It is worth emphasizing 
that SOA is formulated here as a \emph{first order set theory}, somewhat confusingly; 
and if we extend that first order set theory to a second order language in the usual way, 
the result then fails to hold. The reason for this is that second order set set theories 
will typically have comprehension axioms for the second order variables
$$\exists X \forall y [Xy \equiv \varphi(y)] $$
while the instance 
$$ \Diamond \exists X \Box \forall y[Xy \equiv y = y]$$
of the translation of this schema is demonstrably false in SHWP and its extensions.

This incompatibility suggests that the equivalence 
between these theories is not deep. Ultimately, they disagree on questions about 
second order existence, 
such as whether there can be a plurality of all (possible) sets. Nevertheless, 
it is useful in that it means we can help ourselves to the modal translations 
of arithmetic truths (and the corresponding truths of the fragment of set theory)
when reasoning in height and width potentialist theories.

Theorem \ref{SHWP2} is of more direct significance to us. 
It says that SOA interprets SHWP and hence
secures the consistency of the latter relative to the former. 
Full details of the 
ideal proof are a little fiddly and would go beyond our needs here. 
But some idea of how things go down will be useful.

The key idea behind the translation $\varphi \mapsto \varphi^\exists$ is to factor 
out use of modal operators in favor of quantification over possible worlds. 
Ultimately what the actualist aims to do is to define a notion of a world, and of 
a formula's holding at a world, in such a way that when the potentialist says 
something is possible the actualist has a corresponding truth at a possible world.

The introduction of talk of worlds raises some questions about the potentialist 
view that are not settled by the axioms up to now, but that are settled under 
translation, in various ways, depending on the choice of definition for `possible 
world'. For example, supposing the $X$s are everything and the $Y$s are some arbitrary things, 
is it possible that everything is either one of the $X$s or the set whose elements are the $Y$s?

Various answers to this question and others like it might be entertained. One might, for example, 
require that whenever some set is introduced, so too are all its subsets: a possible world 
will then be a $V_\alpha$, and the above question will be answered negatively in general. Another, 
more liberal option would be to answer in the affirmative in all cases. This 
corresponds to allowing set introduction in a maximally `piecemeal' fashion: given 
any things, you can introduce just the set of them and nothing else.

However this question is answered by the potentialist, the actualist may introduce a corresponding 
formula, $World(x)$, defined so as to pick out those sets that the potentialist thinks of as \emph{possible worlds},
or in this context simply as sets that the potentialist might allow to be all the sets. 
The actualist will then use that definition, along with quantification over such worlds,
to interpret the potentialist's claims of possibility and necessity. 

Formally, this means our translation will have to 
carry formulas $\varphi$ in the modal language (which may be sentences) to formulas 
$\varphi^\exists(w)$ 
in the first order language with a free `world' variable $w$. Key clauses of the 
translation are things like
\[ (\forall x \varphi)^\exists(w) := \forall x \in w \varphi^\exists(w) \]
\[ (\forall X \varphi)^\exists(w) := \forall x \subseteq w \varphi^\exists(w) \]
\[ (\Box \varphi)^\exists(w) := \forall u [World(u) \wedge w \subseteq u \rightarrow \varphi^\exists(u)]\]
where $World(u)$ is the definition of world at issue. 

For the sake of concreteness, let us take a maximally liberal appraoch to set 
formation where given any things one is allowed to introduce the set of those 
things and nothing else. A `possible world' will then be any transitive set.
The potentialist's axioms for SHWP are then readily seen to be true, 
under the translation, 
in SOA. For HP, this is because every set is an element of a transitive set. 
For WP, this is because SOA proves all sets are countable. So for any 
given partial order, we may move to a transitive set that witnesses its countability, 
and then proceed to introduce a generic for it if need be.

\subsection{Strong HWP}
The theory I have just cited combines height and width potentialism in a 
straightforward way. The result is a theory that is (up to something like 
mutual interpretability) 
essentially just second order arithmetic. This is a pretty weak theory, although as we know 
from the reverse mathematics literature it is plenty strong enough to develop 
much of the mathematics needed in applications. Can the height and with potentialist 
do better?

There is indeed a reasonable way to proceed here. Our potentialist is interested 
with expansions of the universe along two `directions'. One has the ability 
to extend the universe `upwards' to create sets of higher rank; and one has the ability 
to extend it `outwards' by forcing. By separating out these two possible methods of 
expansion, and by asserting strong axioms regarding what can be attained along the 
vertical dimension of expansion alone, much stronger HWP systems can be derived,
indeed ones with all the power of ZFC and more. This extension makes the general 
picture of height and width potentialism much more philosophically interesting, since 
such versions have at least the consistency strength to recover all standard mathematics.
But they also bring new conceptual problems with them, problems that will be exploited 
in one of the inconsistency arguments we will consider. 

Let us first discuss in a little more detail how to implement this idea. We expand 
the language $\mathcal{L}_0$ to $\mathcal{L}_1$ by adding in two new modal operators, 
$\circledbar$ and $\circledbslash$. $\circledbar$ is to reflect possibility by \emph{only} vertical expansion: 
one can think of this as possibility by \emph{only} iteratively introducing new sets 
for given pluarlities. $\circledbslash$, on the other hand, is to reflect possibility by \emph{only}
horizontal expansion: one can think of this, indifferently, as just by adding 
new generic filters, or adding new enumerating functions. ($\boxbar$ and $\boxbslash$
are the duals.) $\Diamond$ remains in the 
language, and represents `absolute' possibility, that is, possibility by either 
domain expansion method. 

Here, in outline, is a way to axiomatize a `strong' system of this kind.
(More details can be found in \citep{CS2021}.)

We begin by modifying HP and WP. 
\begin{description}
    \item[HP] $\Box \forall X \circledbar \exists x \forall y[y \in x \leftrightarrow Xy]$
\end{description}

\begin{description}
    \item[WP] $\Box \forall \mathbb{P} \forall X D(X, \mathbb{P}) \rightarrow \circledbslash \exists g(F_\cap(g, \mathbb{P}, X)) $
\end{description}

Next, we add rules to reflect the generality of $\Diamond$.
\begin{description}
    \item[Gh] $\circledbar \varphi \rightarrow \Diamond \varphi$
    \item[Gw] $\circledbslash \varphi \rightarrow \Diamond \varphi$
\end{description}
Finally, add the following restricted version of the 
powerset axiom to vertical possibility.
\begin{description}
    \item[r-Pow] $\Box \forall x \circledbar \exists y \boxbar \forall z[ z \in y \leftrightarrow z \subseteq x]$
\end{description}
r-Pow says that it is always possible by vertical expansion to get all the subsets of 
any given set, \emph{so long as one disregards the subsets you can get by forcing}. 
This is a kind of local powerset axiom: in intuitive (procedural) terms,
it says that you can always eventually get 
all the subsets of a given set you can get without forcing, if you go on introducing 
sets long enough.

It is useful to compare r-Pow with the corresponding `unrestricted' version 
\begin{description}
    \item[Pow] $\Box \forall x \circledbar \exists y \Box \forall z[ z \in y \leftrightarrow z \subseteq x]$
\end{description}
which is provably inconsistent with {\bf WP} over the rest of the theory. Pow says that 
it is possible for there to such things as \emph{all possible} subsets of any given 
set; but given the existence of an infinite set (as we are guaranteed), and the 
universal possibility of forcing, enshrined in {\bf WP}, this cannot be, since we can always force 
to add new subsets to any given set.

Let the extension of the core theory by the above principles (but not, of course, Pow)
be called HWP. The following facts come easily.
\begin{theorem}\label{HWP0}
    There is a translation $\cdot^\circledbar : \mathcal{L}_\in \to \mathcal{L}_1$
    that preserves theoremhood from ZFC to HWP.
\end{theorem}
\begin{theorem}\label{HWP1}
    There is a translation $\cdot^\Diamond : \mathcal{L}_\in \to \mathcal{L}_1$
    that preserves theoremhood from SOA to HWP.
\end{theorem}
In each case, the result follows much as it did in the previous case. In fact 
Theorem \ref{HWP1} really just is Theorem \ref{SHWP1}. The first uses the same translation
but with $\circledbar$ in place of $\Diamond$ and $\boxbar$ in place of $\Box$ 
everywhere. That the theorem goes through is a known theorem of Linnebo in a slightly 
different key.

The resulting theory has some intriguing features. If one restricts one's attention 
to vertical possibility and necessity, one can construct objects that 
satisfy anything you can get in ZFC, for example the existence of uncountable cardinals, 
beth fixed points, and so on. But these large cardinals (to describe them a little provocatively) are always mirages: 
and the mirage can always be revealed 
by appeal to $\circledbslash$-possibility. Anything that might in some possible world 
satisfy the formula saying that it is (say) $\omega_1$ will also not satisfy that 
description in some other possible world. There are no \emph{absolute} uncountables, only 
relative pretenders.

The chief advantage HWP has over SHWP is interpretative power: it can interpret all 
of ZFC, notwithstanding its ultimate commitment to the countability of all things.
This gives HWP the capacity, at least once motivational details are filled in,
to potentially offer a countabilist foundation for all mathematics. This is I think
one of the principle interests of the combination of height and width potentialism.

I have said that the chief advantage of HWP over SHWP is interpretative power. But 
how much of that does it have? Does it have too much? We now turn to addressing 
these issues of consistency in HWP.

It turns out that the methods employed in the previous section generalize fairly 
naturally to cater to systems like HWP and even further extensions. The generalization
involves extend second order arithmetic to include `topological regularity axioms',
and then giving a more nuanced definition of possible world in terms of such an extension.

Let's start with the extension of second order arithmetic involved, 
to include so-called topological regularity axioms. 
What are those?

Well, there are certain `nice' topological properties of sets of reals -- 
things like being Lebesgue measurable, or having the perfect set property -- 
that cannot hold everywhere, at least given the axiom of choice. The issue (repeatedly)
is that the axiom of choice allows you to well-order the reals and then construct
barbaric sets of various kinds, ones that don't have the nice features, by exploiting 
the well-order. 

There's a general feeling in set theory that the `nice' properties should all 
hold of easily definable sets of reals, and that the `nasty' counterexample sets 
should all be pretty complicated to define (in terms of the usual analytical hierarchy).
For example, if $V = L$, then there are easily definable nasty sets of reals, and 
this is generally taken to be a mark against the principle.

Since 
$V = L$ implies there are easily definable but nasty sets, 
it follows that principles asserting 
that easily definable sets are nice must in some cases go beyond SOA (which is 
of course consistent with $V = L$). Principles of this kind 
are what I'm calling \emph{topological 
regularity axioms}. They assert that good-behavior properties hold for 
certain easily definable classes of reals, even when this is not 
provable in SOA (or ZFC). 

The weakest such axiom is something called the
$\Pi_1^1$-Perfect Set Property (PSP), which says that every uncountable set of reals 
definable by a $\Pi_1^1$ formula is either countable or has a perfect subset.\footnote{
    A perfect set is a closed set with no isolated points.
} It turns out that this minimal extension of SOA is enough to secure 
the interpretability of HWP.

\begin{theorem}
    There is a map $\cdot^\exists : \mathcal{L}_1 \to \mathcal{L}_\in$
    that preserves theoremhood from HWP to SOA + $\Pi_1^1$-PSP.
\end{theorem}
The proof uses similar ideas to those employed for SOA and SHWP in the previous 
section, but as I said one has to take a more nuanced definition of `possible world'. 
In particular, it turns out that here our definition of a world will 
need to be `doubly parameterized': that is, we will need \emph{two} free 
variables in $\varphi^\exists$. 

Why?
Let us think about how our possible worlds will have 
to look to get an interpretation going. In order for r-Pow to be true, 
there will have to be worlds $w$ at which the set $x$ 
of all $\circledbar$-possible sets of numbers 
exist. These worlds are worlds where 
any plurality of numbers, in the sense of $w$, forms a set. 
And yet, since we can always force to add subsets to any infinite set, 
there must still be further possible sets of numbers. These sets of numbers 
have numbers as their members. But we said that all pluralities of numbers at 
$w$ formed sets at $w$!

The only way out of this tangle for the HWP proponent is to accept indefinite 
extendability both with respect the singular domain, and with respect the plural 
domain (holding the singular domain fixed). The introduction of new generics 
witnesses, on this view, the extension of the plural domain (over a fixed) singular 
domain. Indefinite extendability `runs in two dimensions', according to the present 
picture. Hence, our possible worlds will be doubly parameterized, with one 
parameter representing the singular domain, and the other parameter representing the
plural. Expansions will then be possible, under the proposed interpretation, in 
both directions. 

Let's now look in more detail 
how this works. We will make use of the following fact, whose proof 
is originally due to Solovay.
\begin{fact}\label{sol}
    Over SOA, the $\Pi_1^1$-PSP is equivalent to $L[r]$ containing only 
    countably many reals for every $r$.
\end{fact}
Thus, in effect, the Fact says that $L[r]$ always falls badly short of containing 
all reals if we have $\Pi_1^1$-PSP. It is thus a strong form of $V \not = L$.

It is fairly easy to see that the fact has the following corollary.
\begin{corollary}
    In  SOA + $\Pi_1^1$-PSP, $L[r]$ is a (class) model of ZFC for every real $r$.
\end{corollary}
The interested reader can find a short proof of this fact, by Dmytro Taranovsky, online.\footnote{
    See https://web.mit.edu/dmytro/www/other/PerfectSubsetsAndZFC.htm.
}

This now gives us the tools we need to implement our parametrized possible worlds 
strategy. We define, in SOA + $\Pi_1^1$-PSP, our `possible worlds' to comprise
a transitive set $t$ (representing the first order domain) and a real number $r$
(representing the second order domain). We require that $t \in L[r]$, 
and will say that plural quantification in a possible world is always 
restricted to subsets of $L[r]$. A possible world $t, r$ may then be extended 
along two paramters. `Vertical' expansion expands the $t$ parameter, staying within 
$L[r]$. `Horizontal' expansion expands the $r$ parameter, and insodoing 
accommodates the needed growth in the plural domain even holding the first 
order domain fixed. $t$ will in general have different subsets in $L[r]$ than 
$L[s]$, and \emph{these} are the first order theorist's interpretation for talk of 
plurals in HWP. 

To implement this formally, 
our mapping $\varphi \mapsto \varphi^\exists$ will therefore 
need to carry $\varphi \in \mathcal{L}_1$ to a formula in $\mathcal{L}_\in$ with 
not just one but two free variables: $\varphi \mapsto \varphi^\exists(t, r)$. It will 
contain clauses like 

\[ (\forall x \varphi)^\exists(t, r) := \forall x \in t \varphi^\exists(t, r) \]
\[ (\forall X \varphi)^\exists(t, r) := \forall x \subseteq t [x \in L[r] \rightarrow \varphi^\exists(t, r)] \]
\[ (\boxbar \varphi)^\exists(t, r) := \forall u \in L[r] [Tran(u) \wedge t \subseteq u \wedge u \in L[r] \rightarrow \varphi^\exists(u, r)]\]
\[ (\Box \varphi)^\exists(t, r) := \forall s \geq r \forall u \in L[s] [Tran(u) \wedge t \subseteq u  \rightarrow \varphi^\exists(u, s)]\]
(In the final expression, $s \geq r$ means that $r$ is constructible from $s$.)

With the full translation in hand 
it is a fairly straightforward matter to prove that every theorem of HWP 
comes good in SOA + $\Pi_1^1$-PSP under the translation: 
$\varphi \mapsto \forall t, r Tran(t) \wedge t \in L[r][ \varphi^\exists(t, r)]$.
(The fact that $L[r]$ models ZFC
in SOA + $\Pi_1^1$-PSP is needed to get the translations of the $\circledbar$ 
axioms to come good.)
In fact, one can show that HWP has $\Pi_1^1$-PSP$^\Diamond$ as a theorem, and 
prove the same kind of tight proof-theoretic equivalence obtains here 
between HWP and SOA + $\Pi_1^1$-PSP as did between SHWP and SOA simpliciter. (Proofs 
of these claims are provided in the appendix.)

That concludes the (outline of the) formal arguments 
in favor of the consistency of height and width potentialism.
We have seen, in outline, that there are arguments for the consistency of 
height and width potentialism in both a strong and a weak form, relative to 
extensions of SOA by topological regularity axioms.

Let us 
now turn to the arguments alleging inconsistency between height and width 
potentialism, and to see how they relate to the approaches sketched up to now.

\section{Roberts}

Right from the off it is perhaps worth mentioning 
that Roberts' argument applies to HWP only. This is not a big deal: 
he is assuming that the height potentialist will be committed at least 
to the recovery of ZFC in terms of pure height potentialism, so that 
any commitment to width potentialism will require a bimodal treatment 
of the sort described in the last section. But it is worth mentioning 
that a less grand form of height and width potentialism can still be viable,
in the form of SHWP.

Here then is the argument.\footnote{
    This is not exactly the argument I saw in the most recent draft of Roberts'
    paper. Roberts gives a version of the argument that does not straightforwardly 
    use the free variable version of comprehension that plays a starring role 
    in this argument, but instead uses versions of the principles that are closed 
    but in which all the modal operators are replaced by actualized counterparts -- 
    Roberts adds an actuality opeartor $@$ to the language, and uses comprehension 
    in the form $@\Box \forall \vec{x} @\Box \exists X \forall y[Xy \equiv \varphi(y, \vec{x})]$.
    However, comprehension in this form is not derivable from the straightforward 
    combination of an actuality operator to the language without using instances 
    of free variable comprehension. Consequently even in that more complicated setting 
    I think the issues turn on the question of which version of comperhension to accept,
    and I have therefore opted to consider this simpler version of the argument.
} Let HWP$^+$ be the 
results of allowing all instances of comp, 
\[ \exists X \forall x[Xx \equiv \varphi]\]
where $\varphi$ may include free variables. Then:
\begin{theorem}
    HWP$^+$ is inconsistent.
\end{theorem}
\begin{proof}
    HWP can be used to derive\footnote{This is a little misleading, since the last line 
    of the derivation is an application of existential instantiation. So the below is not 
    a theorem, and neither (of course) is its universal generalization.}
    \begin{equation}\label{prev}
        \Diamond \exists x[\Diamond (\exists y[ y \subseteq x \wedge y = z]) \wedge \boxbar(\neg \exists y[y = z])]
    \end{equation}
    by letting $x$ be any infinite set, and $y$ a possible enumeration of the 
    set of $\circledbar$-possible subsets of $x$.

    Abbreviate \eqref{prev} as $\Diamond \exists x \Psi(x, z)$. By 
    existential instantiation and the comprehension instance
    \begin{equation}\label{exa}
        \exists X \forall w[Xw \equiv \Diamond w \in z]
    \end{equation}
    in parameter $z$ we get:
    \begin{equation}\label{exb}
        \Diamond (Ex \wedge \Psi(x, z) \wedge \exists X \forall w[Xw \equiv \Diamond w \in z])
    \end{equation}
    Then by HP, 
    \begin{equation}\label{exc}
        \Diamond (Ex \wedge \Psi(x, z) \wedge \circledbar \exists u \forall w[w \in u \equiv \Diamond w \in z])
    \end{equation}
    one can then instantiate on $u$ and prove $u = z$, using SetExt.

    But this is a contradiction.

\end{proof}

Naturally, given this argument, we should expect the proof of consistency implicit in the 
previous section to break down with strong comprehension. And this is precisely 
what happens in the interpretation offered for HWP in SOA $+ \Pi_1^1$-PSP.
Consider for example the following simple case. Let $w$ be the world that has $\omega$
for its first order domain and $P(\omega)^L$ for its plural domain. 
The interpretation for closed plural comprehension at this world amounts to the claim 
that for any $y$ and $Y$ \emph{in $L$}, the subset of $\omega$ for which 
$\varphi(n, y, Y)$ holds is again in $P(\omega)^L$, which is obviously true. But 
if we remove the restriction to closed formulas, we are saying that for any $\varphi$, $y$ and $Y$ 
we choose (whether in $L$ or not), the set of $n$ with $\varphi(n, y, Y)$ is in $L$. 
But this is provably false in SOA + $\Pi_1^1$-PSP.

The question about the consistency of HWP that arises from this argument concerns, then,
the status of the `strong' open version of plural comprehension. Formally speaking 
the path toward saving HWP is clear: accept only the closed instances of comprehension.

Roberts' claim is that the open instances are mandatory:
any reasonable plural logic will contain them. But I see no good reason 
for thinking this is so. Indeed, I would argue that the intuitive validity 
of the instances of comprehension that are used in 
the cited argument is legitimately (and independently) questionable, 
whether or not one is a potentialist. For example, 
those of a Stalnakerian actualist bent in modal metaphysics 
might find the open plural comprehension schema doubtful, by analogy with things they 
already think about property comprehension.

Let me explain. Stalnakerian actualism involves contingentism about 
(among other things) \emph{properties}. For example, 
in the event that I should not exist, the property of not being 
identical to me would also fail to exist according to Stalnaker. Of course, 
given that I exist, it makes sense to look upon some 
other circumstances in which I don't exist, and truly affirm 
of various things that they are not identical to me. 
But this does not mean that the circumstances in question would 
be ones in which various things had the property of not being 
identical to me, since in those circumstances that property 
does not exist. 

Whatever the merits of this view in general -- I myself 
happen to be sympathetic to it -- it is clearly coherent and 
a reasonable idea to consider in modal metaphysics. At its 
core, from a formal point of view, is precisely a denial of 
arbitrary instances of comprehension for properties. It is 
not true, on this picture, that arbitrary worlds will satisfy comprehension for 
properties on formulas like `being identical to $x$', though 
they will have the closed analogue (for every $x$ there is a property 
of being identical to $x$.) Similar points go for `higher order' instances 
of comprehension, like `being identical to the property $P$'.

For a Stalnakerian, the problem with open instances of property comprehension
is that they allow worlds illict access to things in \emph{other} 
worlds in the model.
In our formal model, we have a bunch of worlds with separate domains,
and we can name all the things in the domain of any model and ask 
questions about whether one thing that exists according to one world 
exists according to another. But these questions are only askable with 
our external, model-theoretic perspective. 
When $z$ gets assigned something that doesn't exist at $w$, 
inhabitants of $w$ do not have the resources to refer to $z$, 
and there in a certain sense simply are no comprehension instances 
involving $z$ at $w$.
Such formulas, 
under interpretation in the model,
just don't correspond to real propositions that exist at the 
worlds at which truth is to be evaluated.
There is no thought there to be thought there.

The view naturally goes along with an analogous one that
denies the relevant instances of comprehension for plurals.
Suppose for example that we think of a plurality as fundmentally 
tied to an individuating property, roughly along the lines that 
a plurality just is the extension of a property (the things instantiating it).
We still hold -- since pluralities are \emph{extensions} -- that plural membership 
is rigid, extensional, and all the other things involved in HWP. Only now, 
our basic comprehension principle for plurals is of the form: for any property 
$P$, there are some things such that $x$ is one of them if and only if $Px$. 
The plurality of $P$s are, on this view, `rigidly' 
the things that happen to instantiate $P$. 

Such an account of pluralities would support
failures of comprehension on non-existent parameters of 
the kind needed to defend HWP, given the corresponding 
failures for property comprehension.
For example, on this view, 
in circumstances where 
I did not exist there would be no guarantee there 
were such things as \emph{my} molecules,
unless some other property (not involving me) happened 
to have them for its extension;
and, this is so even if each individual molecule 
that would comprise me nevertheless existed.\footnote{
    In this case it is plausible that there would be properties
    not involving me that would fit the bill, 
    since for each molecule $m_i$ in me there 
    would be the property of being $m_i$, and 
    hence the disjunctive property of being either 
    $m_1$ or $m_2$ or ... or $m_k$, where $k$ 
    is the number of molecules comprising me. But this sort of thing argument 
    is not available in cases, like the next one I mention and like those 
    relevant to HWP,
    where infinitely many things are involved. 
}
The existence of each thing, as it were, is not 
enough for the existence of the many: there must 
in addition be `resources' to single them out.
Similarly, 
in circumstances where a generic filter
for a partial order (is possible but) does not exist there would be no guarantee 
that there exist such things 
as `its' elements, whether or not each element we will eventually 
recognize `it' to have in fact exists: for there may be no property, like 
`being an element of \emph{it}', that singles `its' elements out. 
(Note that denying precisely this instance of comprehension 
would suffice to block the argument above.)

This seems to me to be a clear enough rationale for holding only closed comprehension 
to be valid, compatibly with the other tenets of height (and width) potentialism.
We still have for example the pluralities we'd like to secure the translations 
of separation in set theory; and we still have reasons for holding membership in 
a plurality to be strongly extensional. But we have reason to reject 
comprehension on parameters that don't exist in our formal theory, since 
in fact in a certain sense 
we believe there \emph{are} no such instances, when it comes to the semantics proper,
in line with Stalnakerian actualism.

I conclude that Roberts' argument does not show HWP is inconsistent.
Instead, it serves to empahsize the importance of the 
restriction to closed comprehension
in the formulation of HWP, 
and highlights the need for a corresponding conception of 
plurality according to which 
only the closed instances are supported, such as the one just given. 
(There are in fact others besides, salient among them a predicativst 
conception of pluralities, but the one above is simpler to explain.)

I think that a useful way to understand Roberts' argument is as showing
that HWP is inconsistent \emph{given the 
`nothing over and above' conception of plurals}. This is the popular conception 
according to which pluralities are the sorts of things you get `for free'
when you have each individual comprising them.\footnote{This must not be confused with 
the view, accepted on all sides, that a plurality is not a single thing 
separate from the many comprising it.} It is crystal 
clear that this take on plurals is incompatible with HWP: for example, 
the nothing over and above conception would clearly mandate acceptance of 
\begin{description}
    \item[BF] $\Box \forall Y (\Diamond \exists X \sqsubseteq Y [X = Z] \supset \exists X \sqsubseteq Y[X = Z])$
\end{description}
where $X \sqsubseteq Y$ abbreviates $\forall x[Xx \supset Yx]$. 
This says that given pluralities cannot gain sub-pluralities, which 
is clearly true if the existence of each component of a plurality 
implies the existence of the plurality; moreover, it is violated 
in HWP, since there even the sub-pluralities of the natural numbers are indefinitely 
extendible.

Roberts and I agree here:
this `nothing over and above' conception and HWP cannot live together.
We disagree, however, because Roberts believes the nothing over and above 
conception is the only way (for a potentialist) to think about plurals.
On the other hand I think something like the account sketched above is 
just as good (for potentialist purposes). Thus, in my view, 
the message is that the proponent of HWP must reject the nothing over and 
above conception, not that their account is inconsistent.

There is, I would argue, a general moral in the vicinity.
HWP uses a cluster of ideas: modality, set membership, quantification, plurality, 
and so on. None of them is an absolute fixed point, and there are few 
conceptual fixed points within them. It is a matter of finding the right configuration 
of these things to do justice to the guiding picture, and then assessing 
the resulting tableau on its own merit. That is why, in my view,
it is highly unlikely that any serious inconsistency argument for the combination of
height and width potentialism will be forthcoming: 
pretty much any derived inconsistency  in a formal system would as naturally 
be taken to show 
the formalization was not up to the task.\footnote{This 
generally seems to be the right response to these sorts of inconsistency proofs.
Compare e.g. Russell's paradox and the concept of set, or Berkeley's 
critique of the notion of the infinitesimal.}

\section{Brauer}
Brauer's argument is quite closely related to Roberts', and in a sense the 
same key logical observations are at work in each. But Brauer's argument brings 
out some interesting issues distinctive to the foundations of set theory 
that are not really brought up by Roberts, and for that reason 
warrants seprate discussion.

The core argument is simple.
\begin{description}
    \item[P1] Height potentialism, as motivated by the iterative concept of set,
    sanctions only convergent extensions of the universe: if it could be that 
    the set $x$ existed, and it could be that some other set $y$ existed, 
    then it could be that $x$ and $y$ 
    both exist together.  
    \item[P2] Width potentialism, as motivated by the idea of forcing potentialism,
    requires allowing for divergent extensions of the universe: 
    for there are (or at least could be) partial orders 
    $P$ and possible generic filters $G_1$ and $G_2$ such that $G_1$ and $G_2$ 
    could not exist together.
    \item[C] Height and width potentialism (so understood) are inconsistent.
\end{description}
I'm not completely sold on either premise. In the case of P1,
Brauer gives an argument which he glosses informally as follows:
\begin{quote}
    The basic idea of the argument is straightforward: 
    if you have a universe $V_0$, you can get different extensions 
    $V_1$ and $V_2$ when $V_1$ 
    is formed by bringing together some collections $X$ into sets and 
    $V_2$ is formed by bringing some other collections $Y$ into sets; 
    but nothing in the process of forming sets out of the 
    $X$ collections would then prevent you from 
    later forming sets out of the $Y$ collections.
\end{quote}
So, we are to conclude, that according to height potentialism under the guise of
the iterative concept of set, only convergent extensions of the universe are possible.

My doubts about this form of argument concern whether it is OK to conclude 
that quite generally the iterative concept of set is incompatible with 
divergent extensions of the universe. After all, set theory is not everything, 
and although there may be nothing in the simple process of forming sets 
from given things that prevents 
re-convergence, nevertheless other things may do so. 

For example, suppose I have a handle, two blades, some permanently 
binding superglue, and a universe of sets $V$. I can then form the knife $K_1$
and form its singleton getting a universe of sets $V_1$. Or I can form 
the knife $K_2$ and form \emph{its} singleton getting $V_2$. But these 
can never be unified. 

Or, for a more pointed\footnote{...and yet, less pointed...} example (in the spirit of the `second approach' to width 
potentialism discussed by Brauer on p7), suppose I have two machines. 
One takes in some things and gives back a set with them as its elements.
The other takes in a 
partial order, some dense sets in it, and gives back a plurality that 
intersects all the given dense sets. Then, given suitable background assumptions 
(including, but not by any means restricted to P2),
we will be in a structurally analogous situation to the blade-handle 
case: we might find it is possible to produce a set whose elements are $G_1$,
and also possible to find a set whose elements are $G_2$,
without these addmiting a unification.
But this is no different from the previous case, which is (hopefully?) completely 
unproblematic.

But discussion of P1 is really a red herring, since at least so far as the 
brands of height and width potentialism here go P2 is false:
one can argue in (S)HWP that if it is possible that some generic 
$g_1$ exists, and it is possible that some generic $g_2$ exists, then it is 
possible that they exist together. (One can see this most easily by using 
the $\Diamond$-translation and reasoning in SOA.) 
Thus the modal logic of forcing, in this 
setting, \emph{is} convergent. 

Brauer's argument on the other hand is based on the familiar result from ZFC saying 
that, in any given countable transitive model for ZFC, one can find a partial 
order and a pair of generics for it that can't exist together in a model of ZFC.
The argument uses the meta-mathematical fact that any countable model can be 
enumerated in the metatheory, and one can then encode said enumeration into a 
real number in an absolute way. That real number then cannot live the relevant 
countable model, since if it did the model would be able to furnish a bijection 
between its ordinals and $\omega$. But that of course cannot hold in any model of ZFC.

This is clearly correct, and is indeed a fact that the proponent of HWP can 
recognize: they can see for example that it is vertically possible that there
are models for (fragments of) ZFC, and partial orders they contain, and such that no 
possible model for the same theory contains both generics. But from the point of 
view of HWP, this is just because you are ignoring some universes of sets, namely those 
wherein everything under discussion is in fact countable. More generally, according 
to them, it is always possible for generics to co-exist, even if 
sometimes they never cohabitate in a model of ZFC.

The point is that HWP is associated with a radically different coneption of 
the mathematical universe than that associated with ZFC. 
Effectively, it locates all mathematics in second order arithmetic,
taking all sets to be countable, and 
interprets ZFC only in restricted inner models. As a result the 
whole meta-mathematical landscape looks somewhat different, and standard 
results about models of ZFC now have quite a different significance. 
In particular,
the Woodin argument no longer shows in any deep sense that the modal logic 
of forcing fails to converge, at least in the most general sense at issue.
It only shows it holds for a certain restricted class of models.

I should say that, as with Roberts previously, there is a sense in which 
I completely agree with Brauer. Insofar as we view height potentialism 
and the iterative concept of set as essentially associated with principles related 
to ZFC, then what he says is correct: width (or forcing) potentialism is incompatible with these things.
For example, if our height potentialist requires (as Brauer suggests they should)
that \emph{all possible} sets over a given domain get introduced when any set 
at all gets introduced, then (as with the discussion of Pow on pn) we will 
be able to prove the negation of width potentialism. But I take it that this 
is clearly not a deep obstacle to a height and width potentialist theory, 
since this commitment of height potentialism is optional (cf Linnebo).
One could, for example, think of sets as the sort of thing that could be introduced 
`one at a time', so that 
any particular things can be  made to form a set, even if it is impossible ever 
to get absolutely all the sets you can ever get out of a given domain. 

Analogous things stand to be said about the iterative concept of set, as Brauer 
notes. There 
are really two ideas, a weak version which says that a set is something obtainable 
from already given things, and  a strong version which says that in addition 
the powerset is always obtainable from the set. HWP is compatible with the iterative concept 
in the first form, but not the second. 

It is also true that the stronger form of the iterative concept of 
set, as associated with ZFC, is at once standardly accepted as basic in 
set theory
and rejected in HWP. As Bruaer emphasizes, there therefore remain significant questions 
of about the extent to which this revisionary positions is defensible, 
and questions about what the underlying motivation for the view is (in 
analogy to the familiar iterative concept); these are questions 
on which Brauer himself has illuminating ideas, although I think ideas 
along the lines of the theories set out here (of a more radical 
kind, to be sure) are also possible. One such account, which I am currently 
working on with Neil Barton, is a foundational picture according to which the notion 
of real number is basic, and the concept of set (like other matheamtical concepts)
is held to be reducible to it, rather than the other way around as we standardly 
think. Formally, you can think of this as an account where 
the true axioms for set theory are really just extensions of 
SOA by topological regularity, rather than ZFC by large cardinals. There is 
no loss of mathematical power, though, because in effect you can have 
all the things you usually want in `inner models'.
But the details, and defense, are for another time. 

So as before I think there is value in Brauer's argument: not just in bringing the 
interesting considerations about the modal logic of forcing to the fore,
but also in really emphasizing the need to understand set theory quite differently 
from usual in order to support HWP.
But I do not think that in either case there 
is anything that presents a severe impediment to the combination of height and width 
potentialism, or to the combination of the iterative concept of set and forcing potentialism,
in anything other than unreasonably restricted readings of these terms. 

With that said,
I understand that to some extent this is a terminological point: others may 
feel differently about what is essential to `the iterative concept of set', 
or forcing potentialism, or whatever.
But the terminological points are not important: what is important,
and I hope clear 
from this discussion, is that there is 
at least a coherent 
position that incorporates a lot of the traditional ideas in the 
iterative conception, understood in height potentialist terms, but that amalgamates 
width potentialism as well. 

The foundations of set theory is at present rather like the foundations 
of quantum mechanics. The discipline itself has thrown up results that our 
ordinary mathematical tools are incapable -- or seem incapable -- of resolving.
Foundational researchers have therefore been led to look for different ways 
to understand what's going on, different ways of organizing the data presented 
in matheamtical research into a conceptual scheme. Height and width potentialism 
combined seems to me to be a reasonable player here that has clear benefits when it 
comes to understanding the centrality of forcing practice to contemporary 
set theory, one of the big open problems in the area. As the considerations 
here attest, it is also logically and metaphysically subtle, 
and conceptually revisionary. But let 
us not discard it outright for that.

\section*{Appendix}
In this appendix I will refer to the theory HWP as $H$, and SOA + $\Pi_1^1$-PSP as $SOA^+$.
I will construe the latter theory as ZFC without powerset toegether with the axiom that 
all sets are hereditarily countable, and that any $\Pi_1^1$ subset of $\omega$ 
has a perfect subset.
\subsection*{Necessary Lemmas}
\begin{lemma}[Mirroring]
    Let $\Gamma 
    \cup {\phi}$ be a set of formulas in the first order language of set theory. Then
    $\Gamma \vdash \phi$ in first order logic if and only if 
    $\Gamma^\Diamond \vdash_{H}\phi^\Diamond$. 
    The same is true for $\D$.
\end{lemma}
\begin{proof}
    See for example \cite[Chapter 12]{OL2018}.
\end{proof}

\begin{lemma}[Bounded Modal Absoluteness]\label{BMA}
    Let $\phi$ be 
    a formula in the first order language of set theory
    with only bounded quantifiers. Then
    $H\vdash \phi \leftrightarrow \phi^\Diamond$.
    The same is true for $\D$.
\end{lemma}
\begin{proof}
    An induction on the complexity of $\phi$. See Lemma 12.2 of \cite{OL2018}.
\end{proof}
\begin{lemma}[$\Delta_1$ Absoluteness]\label{DA}
    Let $\phi$ be a formula that is provably $\Delta_1$ over $ZFC$ without 
    powerset. Then we have: 
    \[H\vdash \phi^{\Diamond} \leftrightarrow \phi^{\D}\]
\end{lemma}
\begin{proof}
    Using the previous two lemmas, we see our assumption implies that there are $\Delta_0$ formulas $\psi, \theta$
    for which
    \[H\vdash \phi^\Diamond \leftrightarrow 
    \Box \forall x \psi^\Diamond \leftrightarrow 
    \Diamond \exists x \theta^\Diamond \] 
    and similarly
    \[H\vdash \phi^{\D} \leftrightarrow 
    \B \forall x \psi^{\D} \leftrightarrow 
    \D \exists x \theta^{\D} \] 
    by the previous lemma, these are each equivalent to 
    \[H\vdash \phi^\Diamond \leftrightarrow 
    \Box \forall x \psi \leftrightarrow 
    \Diamond \exists x \theta \] 
    \[H\vdash \phi^{\D} \leftrightarrow 
    \B \forall x \psi \leftrightarrow 
    \D \exists x \theta \] 
    respectively. But weakening implies that 
    $\D \exists x \theta \rightarrow \Diamond \exists x \theta$
    and similarly that $\Box \forall x \psi \rightarrow \B \forall x \psi$.
    The result follows.
\end{proof}
\subsection*{Theorems}
\begin{theorem} 
    There is an interpretation $\exists : \mathcal{L}_1 \to \mathcal{L}_\in$ that preserves 
    theoremhood from $H$ to SOA$^+$.
    \end{theorem}
    \begin{definition}[$\varphi_\exists$]
        The translation $\varphi \mapsto \varphi_\exists$ is defined by the following clauses.
        In each case, $d_1, d_2$ are the least variables not occuring in $\varphi$, and $\leq$ is the relation of relative constructibility for 
        reals.
        \begin{itemize}
            \item $x \in y_\exists := x \in y \wedge \bigwedge_i d_i = d_i$
            \item $Xx_\exists := Xx \wedge \bigwedge_i d_i = d_i$
            \item $(\neg \varphi)_\exists := \neg \varphi_\exists(d_1, d_2)$
            \item $(\varphi \vee \psi)_\exists := \varphi_\exists(d_1, d_2) \vee \psi_\exists(d_1, d_2)$
            \item $(\exists x \varphi)_\exists := \exists x \in d_1[ \varphi_\exists(d_1, d_2)]$
            \item $(\exists X \varphi)_\exists := \exists x \subseteq d_1 \wedge x \in L[d_2] [ \varphi_\exists(d_1, d_2)]$
            \item $(\D \varphi)_\exists := \exists e[e \supseteq d_1 \wedge Tran(e) \wedge e \in L[d_2] \wedge \varphi_\exists(e, d_2)]$
            \item $(\Diamond \varphi)_\exists := \exists e_1, e_2[d_2 \leq e_2 \wedge d_1 \subseteq e_1 \in L[e_2] \wedge Tran(e_1) \wedge \varphi_\exists(e_1, e_2)]$
            \item $(\circledbslash \varphi)_\exists := (\Diamond \varphi)_\exists$
        \end{itemize}
        In these, $\varphi_\exists(e_1, e_2)$ represents the result of substituting $e_1, e_2$ for $d_1, d_2$ in
        $\varphi_\exists$.
    
        We then set $\exists(\varphi) := Tran(d_1) \wedge Real(d_2) \supset \varphi_\exists$.
    \end{definition}
    \begin{proof}
        The propositional tautologies and quantifier logic stuff is straightforward.
    
        The laws of S4.2 for $\circledbar$ are easy.

        As for $\Diamond$, it is again easy to see the axioms of S4 come good.
        For $.2$, suppose 
        $(\Diamond \Box \varphi)_\exists$. Then there is a real $e_2$ with $d_2$ constructible from
        $e_2$, and transitive $e_1$ a superset of $d_1$ and element of $L[e_2]$,
        such that $ (\Box \varphi)_\exists(e_1, e_2)$.
        That in turn means that for every real $e$ with $e_2$ constructible from $e$, and 
        every transitive 
        set $d$ containing $e_1$ and an element of $L[e]$, we have 
        $\varphi_\exists(d, e)$.

        So suppose given arbitrary $d, e$ with $d_2$ constructible from $e$, 
        $d$ a superset of $d_1$ and an element of $L[e]$. We must find a real $r$ 
        with $e$ constructible from $r$, and transitive set $t$ a superset of $d$ and element 
        of $L[r]$ such that $\varphi_\exists(t, r)$. The obvious candidates are 
        $t:= e_1 \cup d$, and $r := e_2*e$. (Here $*$ can be any function taking a pair of 
        reals to a real they are both constructible from.) It is not hard to see they 
        have the required features.

        All the plural axioms are straightforward. For comprehension, we use 
        separation in $L[r]$. Note that only the closed version is valid, 
        since the open version might involve us with parameters from outside 
        $L[r]$.
    
        On to the set-theoretic axioms. 
        Extensionality is a straightforward consequence of extensionality for 
        sets. The axiom of choice similarly is a consequence of choice in $SOA^+$.
        The existence of an infinite set is also given.
        
        For HP: suppose given a transitive extension $e$ of $d_1$ in $L[d_2]$,
        and a subset $X$ of $e$ that is also an element of $L[d_2]$. Since $L[d_2]$
        satisfies ZFC, it has that 
        every set is an element of a transitive set in $L[d_2]$.
        Thus we can extend $e$ to a transitive 
        set $e_1$ in $L[d_2]$ that contains $X$ as an element, and the result follows. 
        Inf is pretty much the same as before.
        
        For r-Pow, suppose given $d_4 \geq d_2$ and $d_3 \supseteq d_1$, $d_3 \in L[d_4]$,
        and $x \in d_3$. The needed result then follows from the powerset axiom in 
        $L[d_4]$.

        As to WP, by propsition 1 it is enough to show that any set in any 
        $L[r]$ is countable in some extension to $L[s]$ where $s$ is a real 
        from which $r$ is constructible. This follows immediately by 
        winding in a real enumerating the set in question to $r$.

        The modal version of replacement is just the $\Diamond$ and $\circledbar$ 
        translations of replacement; these follow again from replacement in the metatheory.
    \end{proof}

\begin{theorem}
    There is an interpretation $\Diamond: \mathcal{L}_\in \to \mathcal{L}_1$ that preserves theoremhood from SOA$^+$ to $H$
    \emph{on first order formulas}. 
\end{theorem}
\begin{proof}
    The only non-trivial part is to show that $\Pi_1^1$-PSP$^\Diamond$ holds.

    Using Fact \ref{sol} and the mirroring theorem, to establish 
the desired conclusion it is sufficient to show the $\Diamond$-translation 
of the right hand side of the biconditional in Fact \ref{sol}. That is: assuming given an arbitrary real $r$, 
we must show that is possible to produce a function on the natural numbers such that, 
necessarily, if $s$ is a real constructible from $r$, 
then $s$ is in the range of $f$.

The strategy for doing so is simple: we first show that, given any real $r$, 
it is possible to produce the set of all possible reals constructible from $r$ in 
$H$ (in the sense of the modality $\Diamond$). Since $H$ proves all sets are countable (with respect the $\Diamond$ translation)
this implies the desired result.

In more detail,
we first observe that by the $\D$-translation of ZFC, 
we have the $\D$ translation of the assertion that for any real $r$, the 
set of reals constructible from $r$ exists, namely:
\begin{equation}\label{dutrans}
    \D \exists x [\mathbb{R}^{L[r]}(x)^{\D}]
\end{equation}
Where in \eqref{dutrans} $\mathbb{R}^{L[r]}(x)^{\D}$ is the $\D$-translation of the first order 
formula asserting that $x$ 
is the set of reals constructible from $r$. The problem now is to derive from this 
that 
\begin{equation}\label{dtrans}
    \Diamond \exists x [\mathbb{R}^{L[r]}(x)^{\Diamond}]
\end{equation}
since it is only with respect $\Diamond$, and not $\D$, that we have the countability 
of all sets in $H$.

To move from \eqref{dutrans} to \eqref{dtrans} we use the absoluteness 
lemma \ref{DA}. The formula asserting that $x$ is $L[r]_\alpha$ is $\Delta_1$ 
(in parameters $r$, $\alpha$) 
over ZFC without power. Thus, for any $x$, $\alpha$, $r$, 
\begin{equation}\label{butt}H \vdash (x = L[r]_\alpha)^\Diamond \leftrightarrow
(x = L[r]_\alpha)^{\D}.\end{equation}
A simple induction shows that for any ordinal $\alpha$, $\Diamond \exists x[ x = \alpha]$ if and only if 
$\D \exists x[x = \alpha]$.
Moreover, standard set-theoretic reasoning (together with mirroring) 
implies that $\mathbb{R}^{L[r]}$ exists is equivalent to $\omega_1^{L[r]}$
exists, relative to either modality. 
We show that $(\omega_1^{L[r]} \text{ exists})^{\D}$ is equivalent to 
$(\omega_1^{L[r]} \text{ exists })^\Diamond$, 
in contrast of course to the real $\omega_1$.

For the non-trivial direction, suppose $\omega_1^{L[r]} \text{ exists}^{\D}$. 
This is equivalent to there being \emph{$L[r]$-uncountable ordinals relative to $\D$}, in the sense that
\[\D \exists \alpha \in L[r]^{\D} \B \forall f \in L[r]^{\D}[\neg(f : \alpha \twoheadrightarrow \mathbb{N})].\]
So assume the latter and instantiate such an $\alpha$. 
By lemma \ref{DA}, $(\alpha \in L[r])^\Diamond$. Suppose $\alpha$ is not
$L[r]$-uncountable relative to $\Diamond$, i.e. 
$\Diamond \exists f \in L[r]^\Diamond[f : \alpha \twoheadrightarrow \mathbb{N}]$
This implies 
$\Diamond \exists \beta, f \in L[r]_\beta^\Diamond[f : \alpha \twoheadrightarrow \mathbb{N}]$. 
But then using \eqref{butt} and the fact that $\D \beta \text{ exists}$ we may infer
\[\D \exists \beta, f \in L[r]_\beta^{\D}[f : \alpha \twoheadrightarrow \mathbb{N}]\]
contradicting our assumption that $\alpha$ \emph{is $L[r]$-uncountable}$^{\D}$.

We may thus infer \eqref{dtrans} from \eqref{dutrans}: \eqref{dutrans} is equivalent to 
$(\omega_1^{L[r]} \text{ exists})^{\D}$, which is equivalent to 
$(\omega_1^{L[r]} \text{ exists })^\Diamond$,
which latter is equivalent to \eqref{dtrans} by standard set theory and mirroring. 

But now since $H$ proves all sets are countable in the sense of $\Diamond$, it follows that 
\begin{equation}
    (\exists x[\mathbb{R}^{L[r]}(x) \wedge \exists f : \mathbb{N} \twoheadrightarrow x])^\Diamond
\end{equation}
which is just the $\Diamond$-translation of the claim that there are only countably many 
reals constructible from $r$. Hence, by an application of mirroring, we conclude $\Pi_1^1[r]$-$PSP^\Diamond$.
The result follows.

\end{proof}
\bibliographystyle{kluwer}
\bibliography{cons.bib}
\end{document}