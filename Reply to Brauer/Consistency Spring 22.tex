\documentclass{article}

\renewcommand*\familydefault{\sfdefault} 
\newcommand{\du}{\Diamond_\uparrow}
\newcommand{\dl}{\Diamond_\leftarrow}
\newcommand{\bu}{\Box_\uparrow}
\newcommand{\bl}{\Box_\leftarrow}

\usepackage[T1]{fontenc}
\usepackage{amsmath, amssymb}
\usepackage{bussproofs}

\title{On the Consistency of Height and Width Potentialism}
\author{Chris Scambler}
\date{\today}
\begin{document}
\maketitle
\abstract{
    Recent work in philosophy of set theory has furnished 
    some arguments that height and width potentialism 
    are inconsistent with one another. 
    One such argument can be found in this volume (Brauer); 
    the other is forthcoming (Roberts).

	At the same time, others have suggested 
    there may be some merit in the combination of 
    height  and width  potentialism. Such authors 
    have presented views that appear to manifest 
    the combination in a non-trivial way, and have 
    defended philosophical claims on their basis 
    (e.g. that all sets are ultimately countable 
    -- cf Builes \& Wilson, Meadows, 
    Scambler, Pruss).  

	Clearly there is a tension here. 
    The business of this article is to explain 
    (what I take to be) its solution. 
    I will argue that height  and width  potentialism 
    are compatible, and that there is  
    even an attractive view in the foundations of 
    mathematics that arises from their combination. 
    I will do this by explaining that view and how it 
    responds to the arguments alleging inconsistency.

}
\section{Overview}
Recent work in philosophy of set theory has furnished 
some arguments that height  and width  potentialism 
are inconsistent with one another. 
One such argument can be found in this volume (Brauer); 
the other is forthcoming (Roberts).

At the same time, others have suggested 
there may be some merit in the combination of 
height  and width  potentialism. Such authors 
have presented views that appear to manifest 
the combination in a non-trivial way, and have 
defended philosophical claims on their basis 
(e.g. that all sets are ultimately countable 
-- cf Builes \& Wilson, Meadows, 
Scambler, Pruss).  

Clearly there is a tension here. 
The business of this article is to explain 
(what I take to be) its solution. 
I will argue that height  and width  potentialism 
are compatible, and that there is  
even an attractive view in the foundations of 
mathematics that arises from their combination. 
I will do this by explaining that view and how it 
responds to the arguments alleging inconsistency.

The plan is as follows. 
Section 1 gives some definitional background and context. 
Section 2 then presents a system that purports to 
combine height  and width  potentialism and cites 
some new results about its precise consistency 
strength and equivalence to standard actualist 
systems of set theory. Section 3 presents the 
inconsistency argument of Roberts and explains 
how the a proponent of the theory from Section 2 
would reply. Section 4 does the same for Brauer.

\section{Background}
Let me start be explaining the terms. 

For present purposes, \emph{potentialism} in set theory 
will be the idea that there could always be 
more sets than there in fact are. 
In slogan form, for the potentialist,  
the universe of sets is `indefinitely extendible'. 

\emph{Height potentialism} is the idea that the universe
is always extendible `upwards' to include 
new sets of higher rank than any given ones. 
\emph{Width potentialism}, on the other hand, 
is the idea that the universe is always extendible 
`outwards', to contain new sets of no greater rank 
than the max of any given ones.

Historically, height potentialism has been motivated 
by considerations involving the set-theoretic paradoxes. 
Russell's paradox shows there is no set of all 
non-self-membered sets, and hence that the cumulative 
hierarchy of all sets is itself (therefore) not a set. 
But the height potentialist complains that any 
`stopping point' for the cumulative hierarchy 
would be arbitrary. Surely there is no conceptual obstacle 
to any particular collection of ranks of the cumulative 
hierarchy providing `urelements' for a longer continuation.

It is natural to
explicate this idea of an indefinitely extendible 
universe of sets in modal logic. 
Indeed, modal axiom systems based around core height potentialist 
ideas are known to exhibit tight forms of 
equivalence with standard iterative set theories like ZFC.

Width potentialism has been historically 
less popular, although has been attracting 
some attention over the last few decades as a 
way understand the independence phenomenon in set 
theory. There is, in fact, a close structural 
similarity in the motivation for height  and 
width  potentialism in these terms (cf Meadows). 
Just as the height potentialist begins with the 
intuition that it is arbitrary that there should be 
some ranks of the cumulative hierarchy that somehow 
inherently can't extended to include further sets, 
the width  potentialist may begin with the idea that 
it is arbitrary that there should be some universe of 
sets that cannot be extended by forcing over its 
partial orders. Just as in the previous case, the 
mathematics of forcing seems to lead us to believe 
there is at least no conceptual obstacle to making 
sense of such `forcing extensions of the universe'.

Again, as with height potentialism, a natural way for the 
width  potentialist to formalize their view 
involves modal logic: one formulates an axiom to the 
fact that, for any partial order $\mathbb{P}$, it is possible to 
find a generic for $\mathbb{P}$. Such explicitly 
axiomatic approaches 
to width  potentialism are in fact less well-studied: 
the focus of most work in this area has been on model 
theory. Nevertheless, axiomatic approaches are possible 
and easy enough to formulate. 

There are, in any event, clear analogies 
in the cases for height  and width  potentialism. 
In each case one has a central inexistence result 
in first order set theory (Russell's paradox, the 
proof that some filters do not admit partial orders) 
and one seeks to overcome it, after a fashion, by 
implementing a modalized version -- any things can 
form a set in the first case, any partial order can 
be forced over in the second. It is natural to think 
that going potentialist one way might give you some 
reason to consider going potentialist the other way too. 
But then, as I have said, there are those who think 
that this is impossible on grounds of a logical 
inconsistency between the two ideas.

\section{Height and Width Potentialism Combined}

In this section we will present and begin to motivate 
axioms for a theory designed to combine height and width 
potentialism. 

Simple case: take a modal operator, $\Diamond$. 
Height potentialism says: any things possibly are the elements 
of a set. 
Width potentialism: any partial order and any dense sets,
possible to find a filter meeting all of them. 
What first order theory do we get? It turns out that 
with not very much extra we get second order arithmetic 
up to near synonymy. Interpret possible worlds as transitive 
sets. Then all the axioms will come out true (everything 
is countable in our background theory).

If we want more mathematical strength, factor the contributions 
into a vertical and a horizontal dimension. We can assert 
ZFC for the vertical but retain the horizontal. What we 
get is lots of ZFC models living inside our overal universe, 
which is one for SOA. 

These get determinacy axioms in the translation.

\section{Roberts}













\end{document}