\documentclass{article}

\renewcommand*\familydefault{\sfdefault} 


\usepackage[T1]{fontenc}
\usepackage{amsmath, amssymb, amsthm, txfonts, stmaryrd}
\usepackage{bussproofs}

\newtheorem{theorem}{Theorem}
\newtheorem{proposition}[theorem]{Proposition}
\newtheorem{lemma}[theorem]{Lemma}

\title{On the Consistency of Height and Width Potentialism}
\author{Chris Scambler}
\date{\today}
\begin{document}
\maketitle
\abstract{
    Recent work in philosophy of set theory has furnished 
    some arguments that height and width potentialism 
    are inconsistent with one another. 
    One such argument can be found in this volume (Brauer); 
    the other is forthcoming (Roberts).

	At the same time, others have suggested 
    there may be some merit in the combination of 
    height  and width  potentialism. Such authors 
    have presented views that appear to manifest 
    the combination in a non-trivial way, and have 
    defended philosophical claims on their basis 
    (e.g. that all sets are ultimately countable 
    -- cf Builes \& Wilson, Meadows, 
    Scambler, Pruss).  

	Clearly there is a tension here. 
    The business of this article is to explain 
    (what I take to be) its solution. 
    I will argue that height  and width  potentialism 
    are compatible, and that there is  
    even an attractive view in the foundations of 
    mathematics that arises from their combination. 
    I will do this by explaining that view and how it 
    responds to the arguments alleging inconsistency.

}
\section{Overview}
Recent work in philosophy of set theory has furnished 
some arguments that height  and width  potentialism 
are inconsistent with one another. 
One such argument can be found in this volume (Brauer); 
the other is forthcoming (Roberts).

At the same time, others have suggested 
there may be some merit in the combination of 
height  and width  potentialism. Such authors 
have presented views that appear to manifest 
the combination in a non-trivial way, and have 
defended philosophical claims on their basis 
(e.g. that all sets are ultimately countable 
-- cf Builes \& Wilson, Meadows, 
Scambler, Pruss).  

Clearly there is a tension here. 
The business of this article is to explain 
(what I take to be) its solution. 
I will argue that height  and width  potentialism 
are compatible, and that there is  
even an attractive view in the foundations of 
mathematics that arises from their combination. 
I will do this by explaining that view and how it 
responds to the arguments alleging inconsistency.

The plan is as follows. 
Section 1 gives some definitional background and context. 
Section 2 then presents a system that purports to 
combine height  and width  potentialism and cites 
some new results about its precise consistency 
strength and equivalence to standard actualist 
systems of set theory. Section 3 presents the 
inconsistency argument of Roberts and explains 
how the a proponent of the theory from Section 2 
would reply. Section 4 does the same for Brauer.

\section{Background}
Let me start be explaining the terms. 

For present purposes, \emph{potentialism} in set theory 
will be the idea that there could always be 
more sets than there in fact are. 
In slogan form, for the potentialist,  
the universe of sets is `indefinitely extendible'. 

\emph{Height potentialism} is the idea that the universe
is always extendible `upwards' to include 
new sets of higher rank than any given ones. 
\emph{Width potentialism}, on the other hand, 
is the idea that the universe is always extendible 
`outwards', to contain new sets of no greater rank 
than the max of any given ones.

Historically, height potentialism has been motivated 
by considerations involving the set-theoretic paradoxes. 
Russell's paradox shows there is no set of all 
non-self-membered sets, and hence that the cumulative 
hierarchy of all sets is itself (therefore) not a set. 
But the height potentialist complains that any 
`stopping point' for the cumulative hierarchy 
would be arbitrary. Surely there is no conceptual obstacle 
to any particular collection of ranks of the cumulative 
hierarchy providing `urelements' for a longer continuation.

It is natural to
explicate this idea of an indefinitely extendible 
universe of sets in modal logic. 
Indeed, modal axiom systems based around core height potentialist 
ideas are known to exhibit tight forms of 
equivalence with standard iterative set theories like ZFC.

Width potentialism has been historically 
less popular, although has been attracting 
some attention over the last few decades as a 
way understand the independence phenomenon in set 
theory. There is, in fact, a close structural 
similarity in the motivation for height  and 
width  potentialism in these terms (cf Meadows). 
Just as the height potentialist begins with the 
intuition that it is arbitrary that there should be 
some ranks of the cumulative hierarchy that somehow 
inherently can't extended to include further sets, 
the width  potentialist may begin with the idea that 
it is arbitrary that there should be some universe of 
sets that cannot be extended by forcing over its 
partial orders. Just as in the previous case, the 
mathematics of forcing seems to lead us to believe 
there is at least no conceptual obstacle to making 
sense of such `forcing extensions of the universe'.

Again, as with height potentialism, a natural way for the 
width  potentialist to fromalize their view 
involves modal logic: one formulates an axiom to the 
fact that, for any partial order $\mathbb{P}$, it is possible to 
find a generic for $\mathbb{P}$. Such explicitly 
axiomatic approaches 
to width  potentialism are in fact less well-studied: 
the focus of most work in this area has been on model 
theory. Nevertheless, axiomatic approaches are possible 
and easy enough to formulate. 

There are, in any event, clear analogies 
in the cases for height  and width  potentialism. 
In each case one has a central inexistence result 
in first order set theory (Russell's paradox, the 
proof that some filters do not admit partial orders) 
and one seeks to overcome it, after a fashion, by 
implementing a modalized version -- any things can 
form a set in the first case, any partial order can 
be forced over in the second. It is natural to think 
that going potentialist one way might give you some 
reason to consider going potentialist the other way too. 
But then, as I have said, there are those who think 
that this is impossible on grounds of a logical 
inconsistency between the two ideas.

\section{Height and Width Potentialism Combined}

In this section, I will present some axiom systems that seem to 
explicate the intuitive idea of height and width potentialism, as 
described above.

\subsection{Core Logical Principles}
All the versions of height and width potentialism we will consider will be built around
two principles: the first is the height potentialist principle that any things can be 
the elements of a set; the second is the width potentialist `forcing axiom', that 
a generic filter can be found for any partial order. 

Both potentialist principles involve modality; the first also involves plural 
quantification. Accordingly the language $\mathcal{L}_0$
we use to formulate these theories 
will have at least a modal operator $\Box$, singular variables $x_n$, plural variables 
$X_n$, the propositional connectives and quantifiers, the identity symbol $=$ 
and the symbol $\in$ for set membership. The circumstance that some thing $x$ is 
one of some things $X$ will be represented by the concatenation $Xx$. Identity 
is only well formed between terms of the same type (singular/plural). 

The following core axioms will be included in all the potentialist theories we 
will go on to discuss.
\begin{description}
    \item[FQ] Standard rules for free quantifier logic 
    with identity for each type of variable.
    \item[Mod] The modal logic $\mathsf{S4.2}$ with converse Barcan formula,
    necessitation, and standard rules for universal quantification within the scope of $\Box$.  
    \item[PR] The plural rigidity axioms 
    \begin{enumerate} 
        \item $\Box \forall X \Diamond Xx \rightarrow \Box Xx$ 
        \item $\Box \forall X \Diamond (\exists x Xx \wedge x = y) \rightarrow \exists x Xx \wedge x = y$.
    \end{enumerate}
    \item[Comp] For any formula $\varphi(x, y, Y)$, 
    $\Box \forall y \Box \forall Y \Box \exists X \forall x [Xx \leftrightarrow \varphi(x, y, Y)]$ is an axiom.
    \item[P-Choice] A plural version of the axiom of choice.
    \item[M-Rep] A modal version of the axiom scheme of replacement.
    \item[Inf] An axiom saying that the set of natural numbers exists.
    \item[Sets] The axioms of extensionality and the axiom of foundation, and an axiom 
      $\Box \forall x \exists X \Box \forall y[Xy \leftrightarrow y \in x]$ asserting the rigidity of set membership
\end{description}
The precise details of many of these won't be too relevant below, but they are cited 
here for something like completeness' sake. More details can be found in CITE. The 
odd one out, of course, is {\bf Inf}. It is included here for simplicity, since 
many of the 
phenomena we will be interested in occur only at the level of infinite sets.

The basic 
idea behind the core axioms is to get a modal and plural logic, combined with fundamental 
principles of set theory, to allow for the development of set theory based on potentialist 
principles of set existence. Let us now turn to that development.

\subsection{Simple HWP}
In this section I will present what seems to me to be the simplest formal combination 
of height and width potentialism. The theory is not mathematically strong, being 
exactly equivalent in strength to second order arithmetic. But it does offer a clean 
and simple proof of concept for the combination of height and width potentialism, 
as well as a useful starting point for further extensions.

The first axiom we add is:
\begin{description}
    \item[HP] $\Box \forall X \Diamond \exists x \forall y[y \in x \leftrightarrow Xy]$
    \begin{itemize} 
    \item \emph{Any things can be the elements of a set.}
    \end{itemize}
\end{description}
HP enshrines the core idea of height potentialism, since it implies (given the
usual notion of rank) that given any sets one can find others of still higher rank.

The second requires a little more preparation. In our background logic we 
can define the notion of a partial order, the notion of being a filter on 
a partial order, and the notion of being a dense set in a partial order in 
the standard way. Let blackboard variables $\mathbb{P}, \mathbb{Q}$ range over 
partial orders. (These are singular variables.) Let $D(X, \mathbb{P})$ mean 
that all $X$s are dense in $\mathbb{P}$. Finally, let $F_\cap(x, \mathbb{P}, X)$ 
mean that $x$ is a filter on $\mathbb{P}$ that intersects every one of the $X$s.
Then our width potentialist axiom can be stated as:
\begin{description}
    \item[WP] $\Box \forall \mathbb{P} \forall X D(X, \mathbb{P}) \rightarrow \Diamond \exists g(F_\cap(g, \mathbb{P}, X)) $   \begin{itemize} 
    \item \emph{Any partial order can be used in forcing.}
    \end{itemize}
\end{description}
WP is one way to flesh out width potentialism, since it implies that any 
given sets may be extended to include ones of rank no greater than their max by 
forcing. 

The following will be useful to us going forward. In it, $f : \mathbb{N} \twoheadrightarrow X$
is an abbreviation for the assertion that $f$ is a function on $\mathbb{N}$ with every $X$ in its range.

\begin{proposition}\label{SHWP0}
    Let {\bf Count} be the principle:
    $$\Box \forall X \Diamond \exists f[f : \mathbb{N} \twoheadrightarrow X] $$ 
    then {\bf Count} is equivalent to WP over the core logic + HP.
\end{proposition}
\begin{proof}
    See CATBC.
\end{proof}
Let us call the result of adding these HP and either WP or {\bf Count}
 to the core logic 
{\bf Simple Height and Width Potentialism}, or SHWP.

Turning now to questions of consistency, 
it turns out that SHWP is 
demonstrably consistent relative to Second Order Arithmetic (SOA). 
In fact it turns out that SHWP exhibits a tight form of equivalence with SOA,
something Tim Button has called `near synonymy' in the 
recent literature. 

(Here and below, we understand SOA under the guise of 
ZFC without power + all sets are countable. This is 
definitionally equivalent to the more standard
arithmetical formulations; see e.g. Simpson.)

A detailed account of the relationship between SOA and SHWP would be overkill here.
(Details can be found in CITE.) But certain features will be important 
to our discussion of consistency below, and having some idea of how the interpretation 
of the modal theory SHWP in the non-modal SOA goes will be very 
helpful to us going forward. 
So we will spend some time discussing certain aspects of the `tight equivalence'
just mentioned.

Let $\mathcal{L}_\in$ be the first order 
language of set theory. Then:
\begin{theorem}\label{SHWP1}
    There is a map $\cdot^\Diamond : \mathcal{L}_\in \to \mathcal{L}_0$ that preserves 
    theoremhood from SOA to SHWP.
\end{theorem}
\begin{theorem}\label{SHWP2}
    There is a map $\cdot^\exists : \mathcal{L}_0 \to \mathcal{L}_\in$ that preserves 
    theoremhood from SHWP to
    SOA.
\end{theorem}
Theorem \ref{SHWP1} says that SHWP interprets SOA; this 
direction does not really concern us directly, since it does not 
imply anything about the consistency of SHWP, but some 
idea of how things go will be useful.
The translation $\varphi \mapsto \varphi^\Diamond$ proceeds by prepending every 
universal quantifier in $\varphi$ by a $\Box$ and every existential by a $\Diamond$.
A result due to Linnebo says that for $\varphi \in \mathcal{L}_\in$, 
\[\Gamma \vdash \varphi \Leftrightarrow \Gamma^\Diamond \vdash_{Core} \varphi^\Diamond\]
where $\Gamma^\Diamond$ has the obvious meaning of $\{\gamma^\Diamond : \gamma \in \Gamma\}$.
It thus suffices to prove $\varphi^\Diamond$ for each axiom of SOA in SHWP, 
which (using Proposition \ref{SHWP0}) is not difficult. 

Theorem \ref{SHWP2} is directly on-topic. It says that SOA interprets SHWP and hence
secures the consistency of the latter relative to the former. Full details of the 
ideal proof are a little fiddly and would go beyond our needs here. But some idea, 
again, of how things go down will be useful.

The key idea behind the translation $\varphi \mapsto \varphi^\exists$ is to factor 
out use of modal operators in favor of quantification over possible worlds. 

In a bit more detail, the idea is that we define the notion of a possible world 
in SOA to be a transitive set. That means we interpret the language of SHWP so that ordinary 
quantifiers are always restricted to such worlds, and the modal operators induce 
quantifiers over such worlds. 

Formally, this means our translation will have to 
carry formulas $\varphi$ in the modal language (which may be sentences) to formulas 
$\varphi^\exists(w)$ 
in the first order language with a free `world' variable $w$. Key clauses of the 
translation are things like
\[ (\forall x \varphi)^\exists(w) := \forall x \in w \varphi^\exists(w) \]
\[ (\forall X \varphi)^\exists(w) := \forall x \subseteq w \varphi^\exists(w) \]
\[ (\Box \varphi)^\exists(w) := \forall u [Tran(u) \wedge w \subseteq u \rightarrow \varphi^\exists(u)]\]
where $Tran(u)$ of course is the assertion that $u$ is transitive. (The case of 
plural containment $Xx$ is a little fiddly, but can be made to work quite nicely.)
The potentialist axioms are then readily seen to be true, under the translation, 
in SOA. For HP, this is because every set is an element of a transitive set. 
For WP, this is because SOA proves all sets are countable. So for any 
given partial order, we may move to a transitive set that witnesses its countability, 
and then proceed to introduce a generic for it if need be.

A final point is that these results can be strengthened to attain something 
approximating definitional equivalence between the two theories.  ...
\subsection{Strong HWP}
The theory I have just cited combines height and width potentialism in a 
straightforward way. The result is a theory that is (up to a near-synonymy) 
second order arithmetic. This is a pretty weak theory, although as we know 
from the reverse mathematics literature it is plenty strong enough to develop 
much of the mathematics needed in applications. Can the height and with potentialist 
do better?

There is indeed a reasonable way to proceed here. Our potentialist is interested 
with expansions of the universe along two `directions'. One has the ability 
to extend the universe `upwards' to create sets of higher rank; and one has the ability 
to extend it `outwards' by forcing. By separating out these two possible methods of 
expansion, and by asserting strong axioms regarding what can be attained along the 
vertical dimension of expansion alone, much stronger HWP systems can be derived,
indeed ones with all the power of ZFC and more. This extension makes the general 
picture of height and width potentialism much more philosophically interesting, since 
such versions have at least the consistency strength to recover all standard mathematics.
But they also bring new conceptual problems with them, problems that will be exploited 
in one of the inconsistency arguments we will consider. 

Let us first discuss in a little more detail how to implement this idea. We expand 
the language $\mathcal{L}_0$ to $\mathcal{L}_1$ by adding in two new modal operators, 
$\circledbar$ and $\circledbslash$. $\circledbar$ is to reflect possibility by \emph{only} vertical expansion: 
one can think of this as possibility by \emph{only} iteratively introducing new sets 
for given pluarlities. $\circledbslash$, on the other hand, is to reflect possibility by \emph{only}
horizontal expansion: one can think of this, indifferently, as just by adding 
new generic filters, or adding new enumerating functions. ($\boxbar$ and $\boxbslash$
are the duals.) $\Diamond$ remains in the 
language, and represents `absolute' possibility, that is, possibility by either 
domain expansion method. 

Here is a way to axiomatize a `strong' system of this kind. 

We begin by modifying HP and WP. 
\begin{description}
    \item[HP] $\Box \forall X \circledbar \exists x \forall y[y \in x \leftrightarrow Xy]$
\end{description}

\begin{description}
    \item[WP] $\Box \forall \mathbb{P} \forall X D(X, \mathbb{P}) \rightarrow \circledbslash \exists g(F_\cap(g, \mathbb{P}, X)) $
\end{description}

Next, we add rules to reflect the generality of $\Diamond$.
\begin{description}
    \item[Gh] $\circledbar \varphi \rightarrow \Diamond \varphi$
    \item[Gw] $\circledbslash \varphi \rightarrow \Diamond \varphi$
\end{description}
Finally, add the following restricted version of the 
powerset axiom to vertical possibility.
\begin{description}
    \item[r-Pow] $\Box \forall x \circledbar \exists y \boxbar \forall z[ z \in y \leftrightarrow z \subseteq x]$
\end{description}
Pow says that it is always possible by vertical expansion to get all the subsets of 
any given set, \emph{so long as one disregards the subsets you can get by forcing}.

It is useful to compare r-Pow with the corresponding `unrestricted' version 
\begin{description}
    \item[Pow] $\Box \forall x \circledbar \exists y \Box \forall z[ z \in y \leftrightarrow z \subseteq x]$
\end{description}
which is provably inconsistent with {\bf WP} over the rest of the theory. Pow says that 
it is possible for there to such things as \emph{all possible} subsets of any given 
set; but given the existence of an infinite set (as we are guaranteed), and the 
universal possibility of forcing, enshrined in {\bf WP}, this cannot be, since we can always force 
to add new subsets to any given set.

Let the extension of the core theory by the above principles (but not, of course, Pow)
be called HWP. The following facts come easily.
\begin{theorem}\label{HWP0}
    There is a translation $\cdot^\circledbar : \mathcal{L}_\in \to \mathcal{L}_1$
    that preserves theoremhood from ZFC to HWP.
\end{theorem}
\begin{theorem}\label{HWP1}
    There is a translation $\cdot^\Diamond : \mathcal{L}_\in \to \mathcal{L}_1$
    that preserves theoremhood from SOA to HWP.
\end{theorem}
In each case, the result follows much as it did in the previous case. In fact 
Theorem \ref{HWP1} really just is Theorem \ref{SHWP1}. The first uses the same translation
but with $\circledbar$ in place of $\Diamond$ and $\boxbar$ in place of $\Box$ 
everywhere. That the theorem goes through is a known theorem of Linnebo in a slightly 
different key.

\section{Roberts}













\end{document}