\documentclass{article}
\usepackage{amsmath, amsthm, amssymb, bussproofs}
%
\title{On the Consistency Strength of Axioms for Height and Width Potentialism}
%
\author{Chris Scambler 
\\
All Souls College\\
University of Oxford}
%
% These will be typeset in italics
\newtheorem{Theorem}{Theorem}[section]
\newtheorem{Proposition}[Theorem]{Proposition}
\newtheorem{Lemma}[Theorem]{Lemma}
\newtheorem{Corollary}[Theorem]{Corollary}
% These will be typeset in Roman
\theoremstyle{definition}
\newtheorem{Definition}[Theorem]{Definition}
\newtheorem{Fact}[Theorem]{Fact}
\newtheorem{Conjecture}[Theorem]{Conjecture}
\newtheorem{Remark}[Theorem]{Remark}
%
\newcommand{\du}{\Diamond_\uparrow}
\newcommand{\dl}{\Diamond_\leftarrow}
\newcommand{\bu}{\Box_\uparrow}
\newcommand{\bl}{\Box_\leftarrow}

\begin{document} 
\maketitle
%\begin{abstract} \end{abstract}

\section{Introduction}
In CITE, I presented an axiom system for height and width potentialism combined. 
Roughly speaking the axiom system is motivated by the idea of an iterative set 
construction process in which two `acts' are possible at each stage -- 
firstly, that of collecting some things 
into a set, and secondly, that of enumerating some things in terms of the natural 
numbers (or, equivalently, forcing). I proved the system consistent on the assumption 
a Mahlo cardinal existed, and showed that the system interpreted ZFC under an `inner'
modal translation, and that second order arithmetic was interpretable under a 
more general modal translation. This article improves on the consistency result given
there. I will show that in fact the theory is equi-consistent with second order 
arithmetic extended with the $\Pi_1^1$ perfect set property, and hence also with ZFC.
I will also explore the question of bi-interpretability.

\section{The Axiom System}
\subsection{Language}
The language $\mathcal{L}_\diamond$ is multi-modal and in fact contains three modal 
operators, $\du$, $\dl$, and $\Diamond$. $\du \varphi$ should be read as: `by repeated 
acts of collection, $\varphi$ can be made true'; that is, by repeatedly taking some things 
and producing the set with them as its elements, $\varphi$ can be made true. 
(We allow any number of repetations, from $0$ into the transfinite.) 
$\dl$ can be interpreted in one of two ways: either as `by repeated acts of enumeration, 
$\varphi$ can be made true', where here by enumeration we mean the act of correlating 
some given things with the natural numbers; or alternatively `enumeration' 
may be replaced by `forcing'. The results are equivalent in the sense demonstrated in 
CITE. $\Diamond$ is the `most general' modality, and represents possibility by 
arbitrary iterations of either domain expansion technique. 

$\mathcal{L}_\diamond$ is also a monadic second order language, 
with standard, singular `objectual' variables $x$ 
and second order `plural' variables $X$; these latter range over things taken many at 
a time (e.g. the members of an orchestra) rather than individuals (e.g. the conductor). 
In addition, we will of course use the membership symbol $\in$ and identity relation
$=$. Atomic formulas are of the form $x = y$, $X = Y$, $x \in y$, and $Xx$. Compound 
formulas are formed from these in the usual way, and we assume all the usual definitions,
e.g. of $\Box$ in terms of $\Diamond$ and $\neg$.
\subsection{Logic}
The logical (non-set-theoretic) axioms can reasonably naturally be separated 
into those concerning the first-order part of the language, those that concern 
the modals, those that govern the second order variables, 
and those that concern identity.

For the first order part, we take the axioms to be any standard system of first
order quantification logic, with universal instantiation weakened to its `free'
version, that is, with the universal instantiation axiom written in the form: 
\begin{equation}
    \forall x [\forall y [ \varphi y] \rightarrow \varphi x]
\end{equation}
With regards the modal logic, we assume $\mathsf{S4.2}$ for each modal operator,
which (given any standard axiom system) will imply the converse Barcan formula. 
We also take necessitation to be part of the system; and, in 
accordance with the idea that $\Diamond$ is the most general modal at issue, we take 
each of the `weakening' principles
\begin{equation}
    \du \varphi \rightarrow \Diamond \varphi
\end{equation}
\begin{equation}
    \dl \varphi \rightarrow \Diamond \varphi
\end{equation}
as further axioms.
The standardly valid inference rule 
\begin{prooftree}
    \AxiomC{$\Phi_1 \rightarrow \Box (\Phi_2 \rightarrow ... \Box ( \Phi_n \rightarrow \Box \Psi) ... )$}
    \UnaryInfC{$\Phi_1 \rightarrow \Box (\Phi_2 \rightarrow ... \Box ( \Phi_n \rightarrow \Box \forall x \Psi) ... )$}
\end{prooftree}

will also be assumed.

As to the second order logic, we assume full comprehension \emph{in closed form},
so all expressions of the form 
\begin{equation}
    \Box \forall \vec{z} \Box \forall \vec{Z} \exists X \forall x[Xx \leftrightarrow \Phi(x, z, Z)]
\end{equation}
are axioms.
Here of course $X$ may not appear free in $\Phi$, but we assume all variables 
other than 
$x$ that are free in $\Phi$ occur in the lists $\vec{z}, \vec{Z}$. We also assume 
a version of the axiom of choice, to the effect that if $X$ comprises disjoint 
non-empty sets then there is $Y$ containing exactly one element of 
each component of $X$. (Formalizing this in our language is easy enough, using the 
usual definitions of non-empty and disjoint in terms of $\in$ and $=$.)

Finally we turn to the axioms concerning identity, which also tend to involve 
all the other components of the language. As usual, we assume reflexivity and 
Leibniz law for first and second order identity; and as usual we are able to 
derive the necessity of identity from these in the form 
$\forall x \forall y [ x = y \rightarrow \Box x = y]$, analogously for $X$ and $Y$.\footnote{
    Note however that this implies nothing about inexistent values of variables,
    which may be contingently identical given our axioms up to now. That is, 
    $x = y \wedge \Diamond x \not = y$ is consistent.
} However, since our modal logic is not symmetric, we must add the necessity of 
distinctness ($\forall x \forall y [x \not = y \rightarrow \Box [x \not = y]]$) 
`by hand'. 

Our conception of identity for `plural' variables $X$ is in addition \emph{strongly
extensional}, meaning that plurals comprise the same things at all possible worlds. 
We can enforce this conception axiomatically using the following principles.
\begin{equation}
    \forall x [Xx \leftrightarrow Yx] \rightarrow X = Y
\end{equation}
\begin{equation}
    \Diamond Xx \rightarrow \Box Xx
\end{equation}
\begin{equation}
    \Diamond \exists Xx \wedge x = y \rightarrow \exists x Xx \wedge x = y
\end{equation}
Given the inference rule mentioned above, the latter implies a version of the 
Barcan formula for $Xx$, and hence that pluralities do not `pick up' new components 
from world to world. Overall, the effect is to ensure (speaking model-theoretically 
for a moment) that $Xx$ holds at some possible world iff it holds at all 
possible worlds, and that some things are the same things as some others when and only
when they are composed of the same individuals. 






\end{document}