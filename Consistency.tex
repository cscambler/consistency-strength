\documentclass{article}
\usepackage{amsmath, amsthm, amssymb, bussproofs}
%
\title{On the Consistency Strength of Axioms for Height and Width Potentialism}
%
\author{Chris Scambler 
\\
All Souls College\\
University of Oxford}
%
% These will be typeset in italics
\newtheorem{Theorem}{Theorem}[section]
\newtheorem{Proposition}[Theorem]{Proposition}
\newtheorem{Lemma}[Theorem]{Lemma}
\newtheorem{Corollary}[Theorem]{Corollary}
% These will be typeset in Roman
\theoremstyle{definition}
\newtheorem{Definition}[Theorem]{Definition}
\newtheorem{Fact}[Theorem]{Fact}
\newtheorem{Conjecture}[Theorem]{Conjecture}
\newtheorem{Remark}[Theorem]{Remark}
%
\newcommand{\du}{\Diamond_\uparrow}
\newcommand{\dl}{\Diamond_\leftarrow}
\newcommand{\bu}{\Box_\uparrow}
\newcommand{\bl}{\Box_\leftarrow}

\begin{document} 
\maketitle
%\begin{abstract} \end{abstract}

\section{Introduction}
ADD CONVENTION FOR TRANSITIVE SETS AND REALS 
DISCUSS THE S5 CASE

In CITE, I presented an axiom system for height and width potentialism combined. 
Roughly speaking the axiom system is motivated by the idea of an iterative set 
construction process in which two `acts' are possible at each stage -- 
firstly, that of collecting some things 
into a set, and secondly, that of enumerating some things in terms of the natural 
numbers (or, equivalently, forcing). I proved the system consistent relative 
to ZFC + the existence of a Mahlo, showed that the system interpreted ZFC under a restricted
modal translation, and that second order arithmetic was interpretable under a 
more general, `full' modal translation. This article improves on these results in various ways.
I will show that in fact the modal theory in question is equi-consistent with second order 
arithmetic extended with the $\Pi_1^1$ perfect set property, and hence also with ZFC.
I will also explore the question of bi-interpretability.

\section{The Axiom System}
\subsection{Language}
The language $\mathcal{L}_\Diamond$ is multi-modal and in fact contains three modal 
operators, $\du$, $\dl$, and $\Diamond$. $\du \varphi$ should be read as: `by repeated 
acts of collection, $\varphi$ can be made true'; that is, by repeatedly taking some things 
and collecting them together into a set, $\varphi$ can be made true. 
(We allow any number of repetitions, from $0$ into the transfinite.) 
$\dl$ can be interpreted in one of two ways: either as `by repeated acts of enumeration, 
$\varphi$ can be made true', where here by enumeration we mean the act of correlating 
some given things with the natural numbers; or alternatively `enumeration' 
may be replaced by `forcing', where this is understood as the act of 
introducing a filter meeting all the given dense sets of some partial order.
(The resulting interpretations are equivalent in the sense demonstrated in 
CITE.) The remaining modal, $\Diamond$, is the `most general' modality, 
and represents possibility by arbitrary iterations of either domain expansion technique. 

$\mathcal{L}_\Diamond$ is also a monadic second order language, 
with standard, singular `objectual' variables $x$ 
and second order `plural' variables $X$; these latter range over things taken many at 
a time rather than individuals; so for example the members of an orchestra and the string 
section are each possible values of the monadic second order variables, while the conductor 
and the first chair violinist would be possible values for the first order variables. 
In addition, we will of course use the membership symbol $\in$ and identity relation
$=$. Atomic formulas are of the form $x = y$, $X = Y$, $x \in y$, and $Xx$. Compound 
formulas are formed from these in the usual way, and we assume all the usual definitions,
e.g. of $\Box$ in terms of $\Diamond$ and $\neg$.
\subsection{Logic}
The logical (non-set-theoretic) axioms can reasonably naturally be separated 
into those concerning the first-order part of the language, those that concern 
the modals, those that govern the second order variables, 
and those that concern identity.

For the first order part, we take the axioms to be any standard system of first
order quantification logic, with universal instantiation weakened to its `free'
version, that is, with the universal instantiation axiom written in the form: 
\begin{equation}
    \forall x [\forall y [ \varphi y] \rightarrow \varphi x]
\end{equation}
With regards the modal logic, we assume $\mathsf{S4.2}$ for each modal operator,
which (given any standard axiom system) will imply the converse Barcan formula. 
We also take necessitation to be part of the system; and, in 
accordance with the idea that $\Diamond$ is the most general modal at issue, we take 
each of the `weakening' principles
\begin{equation}
    \du \varphi \rightarrow \Diamond \varphi
\end{equation}
\begin{equation}
    \dl \varphi \rightarrow \Diamond \varphi
\end{equation}
as further axioms.
The standardly valid inference rule 
\begin{prooftree}
    \AxiomC{$\Phi_1 \rightarrow \Box (\Phi_2 \rightarrow ... \Box ( \Phi_n \rightarrow \Box \Psi) ... )$}
    \UnaryInfC{$\Phi_1 \rightarrow \Box (\Phi_2 \rightarrow ... \Box ( \Phi_n \rightarrow \Box \forall x \Psi) ... )$}
\end{prooftree}

will also be assumed. 

As to the second order logic, we assume full comprehension \emph{in closed form},
so all expressions of the form 
\begin{equation}\label{pcomp}
    \Box \forall \vec{z} \Box \forall \vec{Z} \exists X \forall x[Xx \leftrightarrow \Phi(x, z, Z)]
\end{equation}
are axioms.
Here of course $X$ may not appear free in $\Phi$, but we assume all variables 
other than 
$x$ that are free in $\Phi$ occur in the lists $\vec{z}, \vec{Z}$. We also assume 
a version of the axiom of choice, to the effect that if $X$ comprises disjoint 
non-empty sets then there is $Y$ containing exactly one element of 
each component of $X$. (Formalizing this in our language is easy enough, using the 
usual definitions of non-empty and disjoint in terms of $\in$.)

Finally we turn to the axioms concerning identity, which also tend to involve 
all the other components of the language. As usual, we assume reflexivity and 
Leibniz law for first and second order identity; and as usual we are able to 
derive the necessity of identity from these in the form 
$\forall x \forall y [ x = y \rightarrow \Box x = y]$, analogously for $X$ and $Y$.\footnote{
    Note however that this implies nothing about inexistent values of variables,
    which may be contingently identical given our axioms up to now. That is, 
    $x = y \wedge \Diamond x \not = y$ is consistent.
} However, since our modal logic is not symmetric, we must add the necessity of 
distinctness ($\forall x \forall y [x \not = y \rightarrow \Box [x \not = y]]$) 
`by hand'. 

Our conception of identity for `plural' variables $X$ is in addition \emph{strongly
extensional}, meaning that plurals comprise the same things at all possible worlds. 
We can enforce this conception axiomatically using the following principles.
\begin{equation}
    \forall x [Xx \leftrightarrow Yx] \rightarrow X = Y
\end{equation}
\begin{equation}\label{prig}
    \Box \forall X, x [\Diamond Xx \rightarrow \Box Xx]
\end{equation}
\begin{equation}\label{pbf}
    \Box \forall X[\Diamond \exists x [Xx \wedge x = y] \rightarrow \exists x [Xx \wedge x = y]]
\end{equation}
Given the inference rule mentioned above, the latter implies a version of the 
Barcan formula for $Xx$, and hence that pluralities do not `pick up' new components 
from world to world. Overall, the effect is to ensure (speaking model-theoretically 
for a moment) that $Xx$ holds at some possible world iff it holds at all 
possible worlds, and that some things are the same things as some others when and only
when they are composed of the same individuals. 
\subsection{Set Theory}
With the background logic in tow we can formalize the axiomatic system of height
and width potentialism that will be the target of our investigations.

The set-theoretic axioms can themselves be reasonably naturally divided into two categories.
First, there are those that concern the identity conditions for sets. Second, there are those 
that concern possible set existence, and that make assertions about the kinds of sets it is 
possible to produce by iterating our various construction procedures.

First, on the side of identity conditions. We would like sets to have the members they have as a matter of 
necessity, so that (like plurals) they have exactly the members they have at any world in all worlds
(to slip into model-theoretic talk again). This can be imposed by the following pair of axioms:
\begin{equation}
   \Box \forall x \Box \forall y[\forall z[z \in x \leftrightarrow z \in y] \rightarrow x = y]
\end{equation}
\begin{equation}\label{ecompl}
    \Box \forall x\exists X\Box\forall z[Xx \leftrightarrow z \in x].
\end{equation} 
Analogs of \eqref{prig} and \eqref{pbf} for $\in$ can be derived using \eqref{prig}, \eqref{pbf}
and \eqref{ecompl}. As a final constraint of sorts on the identity conditions for sets, we 
impose the standard axiom of foundation, which says that every non-empty set has a member with 
which it shares no members. 

As to the set-theoretic axioms, we begin with a discussion 
of the distinctive axioms for height potentialism, for which we will 
follow the development of \O ystein Linnebo in CITE.
Here the central axiom is 
\begin{equation}\label{hpot}
    \Box \forall X \du \exists x[Set(x, X)]
\end{equation}
where $Set(x, X)$ is an abbreviation for $\forall y[ y \in x \leftrightarrow Xy]$. This 
axiom can intuitively be read as saying that any possible things are possibly the elements of a set;
it leads to a form of indefinite extendibility of the universe of sets in light of the modal logical 
derivability of $\Box \exists X \neg \exists x[Set(x, X)]$ (Russell's paradox).

This axiom by itself guarantees the possible existence of each hereditarily finite set, but 
in the spirit of pursuing transfinite set theory in the potentialist system we will want to 
push things further. As a first step, consider the axiom 
\begin{equation}\label{compn}
    \du \exists X \Box \forall x [ Xx \leftrightarrow Nat(x)]
\end{equation}
where here $Nat(x)$ can be any of your favorite non-recursive definitions of natural number 
(e.g. finite von Neumann ordinal).\footnote{One could also 
use a recursive definition, but it makes the exposition slightly more complicated.} 
\eqref{compn} says that by repeated acts of collection one can eventually 
produce all possible natural numbers,
and given this a further application of \eqref{hpot} secures the possible 
existence of an infinite set.

\eqref{compn} is a natural analogue of the axiom of infinity in the potentialist setting. The 
natural analogue of the powerset axiom would be:
\begin{equation}\label{comps}
    \Box \forall y \du \exists X \Box \forall x [ Xx \leftrightarrow x \subseteq y]
\end{equation}
However this 
axiom will be false on the intended interpretation, since even just setting $y = \omega$ we will have 
that it is always possible to introduce new subsets of $y$ in the form of enumerating functions for sets /
generic filters for partial orders on sequences of naturals. But instead of giving up altogether on the infinitary mathematics that goes 
along with powerset, we will instead adopt the `restriction' of the principle to the modality $\bu$.
The idea will be that, given any arbitrary set $y$, by repeatedly introducing sets one will 
eventually get all the subsets of it that one can ever get \emph{without using forcing}. The 
axiom thus reads:
\begin{equation}\label{comps2}
    \du \exists X \bu \forall x [ Xx \leftrightarrow x \subseteq y]
\end{equation}
 This is 
a kind of local powerset axiom for `inner models'; the precise sense in which this is true will 
be made apparent in more detail below.

We will also adopt the following axiom, distinctive of width potentialism. In it, we let 
$D(x, X)$ abbreviate the claim that $x$ is a partial order and $X$ contains all the dense subsets 
in $x$; and $Fmeets(x, X)$ will abbreviate the claim that $x$ is a filter that 
meets all the sets in $X$.
\begin{equation}\label{wpot}
    \Box \forall X, x [D(x, X) \rightarrow \dl \exists g[Fmeets(g, X)]]
\end{equation}
One can show, using the other axioms, that \eqref{wpot} implies the negation of \eqref{comps}.

Finally, we adopt certain instances of the replacement axiom (really the 
collection principle.) To state the relevant instances, we first need a definition.

\begin{Definition}[$\blacklozenge$-translation]
    Let $\blacksquare$ be a modal operator, let $\blacklozenge$ be its dual, and let 
    $\varphi$ be a formula in the first order language of set theory. 
    Then the $\blacksquare$-translation of $\varphi$, written $\varphi^\blacklozenge$, 
    is the result of prepending each atomic formula with a $\blacklozenge$, 
    each universal quantifier in $\varphi$ with $\blacksquare$,
    and each existential quantifier in $\varphi$ with a $\blacklozenge$.
\end{Definition}

The relevant instance of replacement, then, are all the $\du$ and $\Diamond$-translations 
of first-order instances of replacement. In effect, we are asserting that replacement 
is valid both with respect the upwards modality alone, and with repsect the modality 
that combines height and width potentialism.

\section{Basic Facts}
In this section we articulate some basic facts about the axiom system just presented, 
along with some general results that will prove useful later on. 

\begin{Lemma}[Mirroring]
    Let $\blacksquare$ be any modal operator in $\mathcal{L}$, and let $\Gamma 
    \cup {\phi}$ be a set of formulas in the first order language of set theory. Then
    $\Gamma \vdash \phi$ in first order logic if and only if 
    $\Gamma^\blacklozenge \vdash_\mathsf{M} \phi^\blacklozenge$.
\end{Lemma}
\begin{proof}
    See cite.
\end{proof}

\begin{Lemma}[Bounded Modal Absoluteness]\label{BMA}
    Let $\blacksquare$ be any modal operator in $\mathcal{L}$, and let $\phi$ be 
    a formula in the first order language of set theory
    with only bounded quantifiers. Then
    $\mathsf{M} \vdash \phi \leftrightarrow \phi^\blacklozenge$.
\end{Lemma}
\begin{proof}
    An induction on the complexity of $\phi$. See Lemma 12.2 of Linnebo.
\end{proof}
\begin{Lemma}[$\Delta_1^T$ Absoluteness]\label{DA}
    Let $\phi$ be a formula that is provably $\Delta_1$ over $ZFC$ without 
    powerset. Then we have: 
    \[\mathsf{M} \vdash \phi^{\Diamond} \leftrightarrow \phi^{\du}\]
\end{Lemma}
\begin{proof}
    Temporarily let $T$ be the theory in question.
    Note that $T^\Diamond$ and $T^{\du}$ are each contained in $\mathsf{M}$.
    It follows from our assumption that there are $\Delta_0$ formulas $\psi, \theta$
    for which
    \[\mathsf{M} \vdash \phi^\Diamond \leftrightarrow 
    \Box \forall x \psi^\Diamond \leftrightarrow 
    \Diamond \exists x \theta^\Diamond \] 
    and similarly
    \[\mathsf{M} \vdash \phi^{\du} \leftrightarrow 
    \bu \forall x \psi^{\du} \leftrightarrow 
    \du \exists x \theta^{\du} \] 
    by the previous lemma, these are each equivalent to 
    \[\mathsf{M} \vdash \phi^\Diamond \leftrightarrow 
    \Box \forall x \psi \leftrightarrow 
    \Diamond \exists x \theta \] 
    \[\mathsf{M} \vdash \phi^{\du} \leftrightarrow 
    \bu \forall x \psi \leftrightarrow 
    \du \exists x \theta \] 
    respectively. But weakening implies that 
    $\du \exists x \theta \rightarrow \Diamond \exists x \theta$
    and similarly that $\Box \forall x \psi \rightarrow \bu \forall x \psi$.
    The result follows.
\end{proof}

\begin{Theorem}
    $\mathsf{M}$ interprets ZFC under the $\du$-translation.
\end{Theorem}
\begin{proof}
    As with cite, this is a straightforward modification of 
    Linnebo's argument from cite. It is sufficient to prove the $\du$ 
    translations of axioms of ZFC. For example, for the powerset 
    axiom, use \eqref{comps2} and \eqref{hpot}.
\end{proof}

\begin{Theorem}
    $\mathsf{M}$ interprets ZFC$^-$ under the $\Diamond$-translation.
\end{Theorem}
\begin{proof}
    Essentially the same as the previous. 
\end{proof}
Note that we can't interpret power in the previous theorem because we don't 
have \eqref{comps}, only \eqref{comps2}, in $\mathsf{M}$. Indeed, more
can be said:
\begin{Theorem}\label{hcount}
    $\mathsf{M}$ proves every set is hereditarily countable under the 
    $\Diamond$ translation.
\end{Theorem}
\begin{proof}
    See cite.
\end{proof}
\begin{Corollary}\label{soa}
    $\mathsf{M}$ interprets second order arithmetic under the $\Diamond$ translation.
\end{Corollary}
\begin{proof}
    $ZFC^- + V = HC$ is definitionally equivalent to SOA. See Krapf, Simpson.
\end{proof}

\section{The Results}
\subsection{The Correlated Theory}
The first order theory with which we will show $\mathsf{M}$ is (at least) 
mutually interpretable is the theory of second order arithmetic together with the 
perfect set property for all $\mathbf{\Pi}_1^1$ classes of reals. 
Here, a class of reals is $\mathbf{\Pi}_1^1$ if there is a second order parameter 
(real) $R$ and a $\Pi_1^1$ formula $\Phi$ in the language of second order arithmetic
that defines the class $R$; and such a class $R$ has the perfect set property iff 
it is either countable or has a perfect subclass, that is, a subclass that is closed 
in the standard topology on $\mathbb{R}$ and contains no isolated points. 

The development of these notions (e.g. the topology on $\mathbb{R}$) \emph{can} 
be done in SOA, and the study of perfet set properties and their consequences 
can similarly; see Simpson, cite. But it makes life slightly easier to proceed through 
a definitionally equivalent theory that is just first-order. We can do so 
by using the following modification of standard set theory: take ordinary $\mathsf{ZFC}$,
with replacement formulated as collection; remove the powerset axiom and replace it 
with an axiom saying that every set is (hereditarily) countable. The result is 
definitionally equivalent to second order arithmetic since sets in set theory 
can be coded as well-founded trees in SOA. One can now define a \emph{real} to be a 
function from $\omega$ to $\omega$, and a definable class of reals to be $\mathbf{\Pi}_1^1$
if there is a formula in normal form with a single unbounded quantifier at the front
that defines it. In this setting one defines the notion of perfection of a class 
in terms of trees. A closed set always has a natural representation as the class of 
paths through a tree on $\omega \times \omega$; such a closed set is perfect when 
the resulting tree is such that any of its elements has two incompatible extensions.
See citations.

Let $\mathsf{T}$ then be the theory of ZFC without power together with the assertion 
that all sets are (hereditarily) countable, and that the schema asserting the $\Pi_1^1$ 
perfect set property holds. We are interested in $\mathsf{T}$ because while on the one hand 
it proves all sets are countable (and hence that any set sized partial order 
admits a generic filter), nevetheless it interprets ZFC in lots of inner models, and 
more exactly, we have that $\mathsf{T}$ proves ZFC is true in $L[r]$ for every 
real number $r$. For a detailed proof, see Krapf cite. 

These two properties suggest hopes of correlating $\mathsf{T}$ with $\mathsf{M}$ proof-theoretically.
Intuitively, an arbitrary possible world in $\mathsf{M}$ is a transitive set, 
representing all the sets we've built up to whatever point we are it in construction,
and a real $r$ representing all the subsets of $\omega$ we can `see' at the corresponding 
point. 
We can give a parameterized interpretation of formulas in the modal language 
by first order formulas in the language of set theory: 
we interpret $\du \phi$ at parameters $T, r$ as meaning: 
there is a transitive extension of $T$ \emph{contained in $L[r]$} in which 
the translation of $\phi$ holds; and $\Diamond \phi$ as simply there is a transitive 
extension of $T$ in which the translation of $\phi$ holds. The two properties 
cited of $\mathsf{T}$ might then lead us to hope this will do the job: We can always 
climb $L[r]$ to get a model of ZFC, but we can also always move to a `wider' universe 
where anything in $L[r]$ turns out to be countable. And it turns out that 
a converse inteprertation is possible as well, by combining a technique of Linnebo's 
and a theorem of Solovay. 

So much for the intuitions. On to the proofs.

\subsection{An interpretation of $\mathsf{M}$ in $\mathsf{T}$}
 Let $M \vDash SOA + \Pi_1^1 PSP$.

$t : \mathcal{L}_1 \times M \times \mathbb{R}^M \to \mathcal{L}_\in, (\varphi, T, r) \mapsto \psi(T, r)$
\begin{itemize}
    \item assign plural variables odd numbered variables $t(X)$, and singular variables even 
            numbered variables $t(x)$.
    \item $t(x \in y) := t(x) \in t(y) \wedge t(y) \subseteq T$
    \item commutes with propositional connectives 
    \item $t(Xx)(T, r) := t(x) \in t(X) \wedge t(X) \subseteq T$
    \item $t(\forall x \varphi)(T, r) := \forall x \in T [t(\varphi)(T, r)]$
    \item $t(\forall X \varphi)(T, r) := \forall x \subseteq T [x \in L[r] \rightarrow t(\varphi)(T, r)]$
    \item $t(\bu \varphi)(T, r) := \forall S \supseteq T [Tran(S) \wedge S \in L[r] \rightarrow t(\varphi)(S, r)]$
    \item $t(\bl \varphi)(T, r) := \forall s[ r \in L[s] \rightarrow \forall S[ S \supseteq T \wedge rank(S) = rank(T) \wedge S \in L[s] \rightarrow t(\varphi)(S, s)]]$
    \item $t(\Box \varphi)(T, r) := \forall s, S[r \in L[s] \wedge T \subseteq Tran(S) \in L[s]  \rightarrow t(\varphi)(S, s) ]$
\end{itemize}


\begin{Theorem}
  $\mathsf{M} \vdash \varphi$ implies $\mathsf{T} \vdash \forall r, T [r \in \mathbb{R} \wedge T \in L[r] \wedge Tran(T) \rightarrow t(\varphi)(T, r)]$  
\end{Theorem}
\begin{proof}
    Induction on the complexity of proofs. Axioms. Consider for example 
    Free instantiation: $\forall x[\forall y \varphi y \rightarrow \varphi x]$.
    Suppose $T$ and $r$ given; this translates to $\forall x \in T [ \forall y \in T t(\varphi y)(T, r) \rightarrow t(\varphi x) (T, r)]$.
    But this is trivial. Note that unfree instantiation fails since we might have counterexamples 
    outside $T$.

    The plural rigidity axioms are tedious but routine. As an example, consider \eqref{pbf}.
    By way of interpreting the outer $\Box$, let $s \in L[s]$ and $T\subseteq S \in L[s]$. 
    Then for the plural quantifier we assume $t(X) \subseteq S$, and for the next $\Diamond$ 
    we take a transitive extension $S'$ of $S$ in $L[s']$ ($s \in L[s']$). 
    We get as a hypothesis that there exists $t(x) \in S'$ with $t(x) \in t(X)$. 
    But since $t(X) \subseteq S$ we must already have had $t(x) \in S$, 
    and clearly $t(Xx)(S, s)$, since this just amounts 
    to the given $t(x) \in t(X)$, as required.

    For plural comprehension 
    we can use separation in $L[r]$. Let $\Phi(x, z, Z)$ be a formula with the exhibited 
    free variables. It is readily seen 
    (applying the clauses of the translation to \eqref{pcomp}) 
    to suffice to show that 
    in any transitive extension $S$ of $T$ contained in some $L[s]$ ($r \in L[s]$) 
    containing $t(z)$ and $t(Z)$, the set $\{x \in S : \Phi(x, t(z), t(Z))\}$ is a member of $L[s]$. 
    But this follows immediately from the definition of $L[s]$.
    (Note that this only works because we took \eqref{pcomp} in closed form, with quantifiers 
    bounded and necessitated. This has the effect of restricting us to cases where the parameters
    are only allowed to come from $S$ and hence in $L[s]$; if this restriction is lifted, it is 
    easy to find counterexamples to comprehension under the translation. For example, if $S$ contains 
    all reals in $L[s]$, the instance for $\Phi := x \in s'$ for any real $s'$ not in $L[s]$ will do.) 

    On to the set-theoretic axioms. First, \eqref{hpot}. Given any transitive $S \in L[s]$, and any 
    subset $X$ of $S$ also in $L[s]$, the translation follows by considering the transitive set 
    $S \cup \{X\} \in L[s]$. For \eqref{compn} one can use the set of natural numbers itself as the 
    relevant transitive set, which is an element of any $L[s]$. 
    As to power, given $S \in L[s]$ of the relevant kind and any set $x \in S$, we can use the 
    fact that ZFC hold in $L[s]$ to guarantee that $\mathcal{P}(S)^{L[s]}$ is in $L[s]$, and hence 
    the transitive set $S \cup \mathcal{P}(S)^{L[s]}$ will do the trick.

    As for replacement, the $\du$ translation follows from the fact that $M$ proves replacement for 
    each $L[r]$, and the $\Diamond$ translation follows from the same for the model at large.
    

\end{proof}


\subsection{An interpretation of $\mathsf{T}$ in $\mathsf{M}$}
For the converse, we already have lemma \ref{soa}; 
it remains to show $\mathbf{\Pi^1_1}$-$PSP^\Diamond$. 

Work in $\mathsf{M}$, and let $r$ be an arbitrary real. It suffices to show
$\Pi_1^1[r]$-$PSP^\Diamond$. 

We will use the following lemma, which is a slight modification of a theorem of Solovay.
\begin{Lemma}[SOA]\label{sol}
    If there are only countably many reals constructible from $r$, then the $\Pi_1^1[r]$-$PSP$ holds.
\end{Lemma}
The proof is a minor modification of Solovay's argument.\footnote{The modification is that 
here we must use the fact that in SOA one can show 
that any $\Pi_1^1$ set admits a decomposition into $Ord$-many Borel sets; Solovay used $\aleph_1$ 
in place of $Ord$, but the proof works just fine without assuming $\aleph_1$ exists.} 

Using Lemma \ref{sol} and the mirroring theorem, to establish 
the desired conclusion it is sufficient to show the $\Diamond$-translation 
of the hypothesis of Lemma \ref{sol}. That is: assuming given an arbitrary real $r$, 
we must show that is possible to produce a function on the natural numbers such that, 
necessarily, if $s$ is a real constructible from $r$, 
then $s$ is in the range of $f$.

The strategy for doing so is simple: we first show that, given any real $r$, 
it is possible to produce the set of all possible reals constructible from $r$ in 
$M$ (in the sense of the modality $\Diamond$). We then invoke Lemma \ref{hcount}, which 
says that all sets are countable in $\mathsf{M}$ (with respect the $\Diamond$ translation)
to get the result.

In more detail,
we first observe that by the $\du$-translation of ZFC, 
we the $\du$ translation of the assertion that for any real $r$, the 
set of reals constructible from $r$ exists, namely:
\begin{equation}\label{dutrans}
    \du \exists x [\mathbb{R}^{L[r]}(x)^{\du}]
\end{equation}
Where in \eqref{dutrans} $\mathbb{R}^{L[r]}(x)^{\du}$ is the $\du$-translation of the first order 
formula asserting that $x$ 
is the set of reals constructible from $r$. The problem now is to derive from this 
that 
\begin{equation}\label{dtrans}
    \Diamond \exists x [\mathbb{R}^{L[r]}(x)^{\Diamond}]
\end{equation}
since it is only with respect $\Diamond$, and not $\du$, that we have Theorem \ref{hcount}.

To move from \eqref{dutrans} to \eqref{dtrans} we use the absoluteness 
lemma \ref{DA}. The formula asserting that $x$ is $L[r]_\alpha$ is $\Delta_1$ 
(in parameters $r$, $\alpha$) 
over ZFC without power. Thus, for any $x$, $\alpha$, $r$, 
\begin{equation}\label{butt}\mathsf{M} \vdash (x = L[r]_\alpha)^\Diamond \leftrightarrow
(x = L[r]_\alpha)^{\du}.\end{equation}
Note also that for any ordinal $\alpha$, $\Diamond \exists x[ x = \alpha]$ if and only if 
$\du \exists x[x = \alpha]$ (prove).
It is easy enough to see that $\mathbb{R}^{L[r]}$ exists is equivalent to $\omega_1^{L[r]}$
exists, relative to either modality. 
We show that $\omega_1^{L[r]} \text{ exists}^{\du}$ is equivalent to 
$\omega_1^{L[r]} \text{ exists }^\Diamond$, 
in contrast of course to the real $\omega_1$.

For the non-trivial direction, suppose $\omega_1^{L[r]} \text{ exists}^{\du}$. 
This is equivalent to there being \emph{$L[r]$-uncountable ordinals relative to $\du$}, in the sense that
\[\du \exists \alpha \in L[r]^{\du} \bu \forall f \in L[r]^{\du}[\neg(f : \alpha \twoheadrightarrow \mathbb{N})].\]
So assume the latter and instantiate such an $\alpha$ (using the rule on p $N$). 
By absoluteness, $(\alpha \in L[r])^\Diamond$. Suppose $\alpha$ is not
$L[r]$-uncountable relative to $\Diamond$, i.e. 
$\Diamond \exists f \in L[r]^\Diamond[f : \alpha \twoheadrightarrow \mathbb{N}]$
This implies 
$\Diamond \exists \beta, f \in L[r]_\beta^\Diamond[f : \alpha \twoheadrightarrow \mathbb{N}]$. 
But then using absoluteness and the fact that $\du \beta \text{ exists}$ we may infer
\[\du \exists \beta, f \in L[r]_\beta^{\du}[f : \alpha \twoheadrightarrow \mathbb{N}]\]
contradicting our assumption that $\alpha$ \emph{is $L[r]$-uncountable}$^{\du}$.

We may thus infer \eqref{dtrans} from \eqref{dutrans}: since it is vertically possible 
$\omega_1^L[r]$ exists, it is also generally possible, and the general possibility of 
$L[r]$ exists implies the general possibility of the reals of $L[r]$ existing 
by standard set theory and mirroring. 

But now since we also have Theorem \ref{hcount}, it follows that 
\begin{equation}
    (\exists x[\mathbb{R}^{L[r]}(x) \wedge \exists f : \mathbb{N} \twoheadrightarrow x])^\Diamond
\end{equation}
which is just the $\Diamond$-translation of the claim that there are only countably many 
reals constructible from $r$. Hence, by an application of mirroring, we conclude $\Pi_1^1[r]$-$PSP$.
The result follows.


\subsection{Bi-interpretation?}
In order to establish bi-interpretability, we have to be a little bit devious.
This is to account for the shifting behavior of variables in the translation, 
and the fact that some variables intuitively `represent' plurals  in the first order language.

To accommodate this, first we redefine the formulae of $\mathcal{L}_{ST}$ so 
that $x \in y$ is only well formed if $x$ and $y$ both have even indices, 
or $x$ is even and $y$ is odd. It is easy that, up to change in variables, theoremhood 
in $\mathsf{T}$ is the same between the two languages. On the one hand if somethin is a theorem in the
modified language it is a theorem in the old one. On the other, if it is a theorem in the old one,
then the result of relettering any odd variables that come on the left of a membership symbol 
by the variable indexed by the next available even is a theorem of the new system, and the old 
system proves them equivalent.

We then modify the $\blacklozenge$ 
translation by setting: 

\[(y_{2n} \in y_{2k})^\blacklozenge:= \blacklozenge x_n \in x_k \]
\[ (y_{2n} \in y_{2k+1})^\blacklozenge := \blacklozenge X_k x_n \]

Under these definitions, we can now prove:
\begin{Theorem}
    $T \vdash t(\varphi^\Diamond)(T, r) \leftrightarrow \varphi$
\end{Theorem}
\begin{proof}
    An induction on complexity. This is all routine,
    but we discuss the atomic case to bring out 
    the point of the new definitions. 
    Consider then $\varphi := y_{2n} \in y_{2k+1}$.
    Then
    $\varphi^\Diamond$ is $\Diamond X_k x_n$. But now applying $t$ 
    to this, with any $T$, gets us $y_{2n} \in y_{2k+1}$, along with 
    some vacuous quantification. The argument for the case where the 
    second $y$ is $y_{2k}$ is similar. But these are the only cases in the 
    new language. (Note that any formula of the form $y_{2n+1} \in y_{j}$ would only 
    be translatable into a different formula, since the first variable 
    gets assigned a plural variable by $t$.)
\end{proof}

For the converse, we have:
\begin{Theorem}
    $\mathsf{M} \vdash t((\varphi)(T, r))^\Diamond \leftrightarrow \varphi$
\end{Theorem}
\begin{proof}
    An induction on complexity. For the base case, suppose $\varphi$ is $X_k x_n$. 
    Under the $t$ translation this yields $y_{2n} \in y_{2k+1}$, a well formed 
    formula of $\mathcal{L}$. The $\Diamond$ translation of this then takes us 
    to $\Diamond X_k x_n$, which is equivalent to $\varphi$ in $\mathsf{M}$. 
    The case for $x_k \in x_n$ is similar.
    
    The quantifiers and connectives are routine.
    
    For the modal operators, consider first $\varphi := \bu \psi$. The claim then unpacks to
    \[ \Box \forall S [T \subseteq S \in L[r] \rightarrow t(\psi)(S)^\Diamond] \leftrightarrow \bu \psi
    \]
    Going from right to left, assume the antecdent. 
    By weakening
    we get  
    \[ \bu \forall S [T \subseteq S \in L[r] \rightarrow t(\psi)(S, r)^\Diamond]
    \]
    by the IH, this  entails
    \[ \bu \forall S [T \subseteq S \in L[r] \rightarrow \psi] 
    \]
    but since the quantification is now vacuous this in turn reduces to 
    \[ \bu  \psi 
    \]
    securing the result. 

    For the converse, suppose $\bu \psi$, but $\Diamond \exists S \in L[r] \wedge T \subseteq S \wedge \neg t(\psi)(S)^\Diamond$.
    By absoluteness style reasoning\footnote{
        This does work and it is quite similar to an argument above. Perhaps a general lemma?
        The idea is this. Suppose it is genearlly possible some $x$ in $L[r]$ has some 
        absolute property (i.e. $\Diamond \varphi \rightarrow \Box \varphi$). 
        Then it is generally possible it is in some $L[r]_\alpha$. It is vertically possible 
        this $\alpha$ exists and hence that the $L[r]_\alpha$ exists (by absoulteness). 
        Since the property is absolute, it still has it there. So it is vertically possible 
        some $x$ in $L[r]^{\du}$ has the property. 
    } this implies $\du \exists S \in L[r] \wedge T \subseteq S \wedge \neg t(\psi)(S)^\Diamond$.
    Now we can invoke the IH to get 
    $\du \exists S \in L[r] \wedge T \subseteq S \wedge \neg \psi$.
    Here again the quantifiers drop out and we get $\du \neg \psi$. 
    But this is a contradiction.

    The argument for $\varphi := \Box \psi$ is easy. 

    It doesn't go through for $\dl$, so we need to use the restricted language. 
\end{proof}

\end{document}