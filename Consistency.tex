\documentclass{article}
\usepackage{amsmath, amsthm, amssymb, bussproofs}
%
\title{On the Consistency Strength of Axioms for Height and Width Potentialism}
%
\author{Chris Scambler 
\\
All Souls College\\
University of Oxford}
%
% These will be typeset in italics
\newtheorem{Theorem}{Theorem}[section]
\newtheorem{Proposition}[Theorem]{Proposition}
\newtheorem{Lemma}[Theorem]{Lemma}
\newtheorem{Corollary}[Theorem]{Corollary}
% These will be typeset in Roman
\theoremstyle{definition}
\newtheorem{Definition}[Theorem]{Definition}
\newtheorem{Fact}[Theorem]{Fact}
\newtheorem{Conjecture}[Theorem]{Conjecture}
\newtheorem{Remark}[Theorem]{Remark}
%
\newcommand{\du}{\Diamond_\uparrow}
\newcommand{\dl}{\Diamond_\leftarrow}
\newcommand{\bu}{\Box_\uparrow}
\newcommand{\bl}{\Box_\leftarrow}

\begin{document} 
\maketitle
%\begin{abstract} \end{abstract}

\section{Introduction}
In CITE, I presented an axiom system for height and width potentialism combined. 
Roughly speaking the axiom system is motivated by the idea of an iterative set 
construction process in which two `acts' are possible at each stage -- 
firstly, that of collecting some things 
into a set, and secondly, that of enumerating some things in terms of the natural 
numbers (or, equivalently, forcing). I proved the system consistent relative 
to ZFC + the existence of a Mahlo, showed that the system interpreted ZFC under a restricted
modal translation, and that second order arithmetic was interpretable under a 
more general, `full' modal translation. This article improves on these results in various ways.
I will show that in fact the modal theory in question is equi-consistent with second order 
arithmetic extended with the $\Pi_1^1$ perfect set property, and hence also with ZFC.
I will also explore the question of bi-interpretability.

\section{The Axiom System}
\subsection{Language}
The language $\mathcal{L}_\diamond$ is multi-modal and in fact contains three modal 
operators, $\du$, $\dl$, and $\Diamond$. $\du \varphi$ should be read as: `by repeated 
acts of collection, $\varphi$ can be made true'; that is, by repeatedly taking some things 
and collecting them together into a set, $\varphi$ can be made true. 
(We allow any number of repetitions, from $0$ into the transfinite.) 
$\dl$ can be interpreted in one of two ways: either as `by repeated acts of enumeration, 
$\varphi$ can be made true', where here by enumeration we mean the act of correlating 
some given things with the natural numbers; or alternatively `enumeration' 
may be replaced by `forcing', where this is understood as the act of 
introducing a filter meeting all the given dense sets of some partial order.
(The resulting interpretations are equivalent in the sense demonstrated in 
CITE.) The remaining modal, $\Diamond$, is the `most general' modality, 
and represents possibility by arbitrary iterations of either domain expansion technique. 

$\mathcal{L}_\diamond$ is also a monadic second order language, 
with standard, singular `objectual' variables $x$ 
and second order `plural' variables $X$; these latter range over things taken many at 
a time rather than individuals; so for example the members of an orchestra and the string 
section are each possible values of the monadic second order variables, while the conductor 
and the first chair violinist would be possible values for the first order variables. 
In addition, we will of course use the membership symbol $\in$ and identity relation
$=$. Atomic formulas are of the form $x = y$, $X = Y$, $x \in y$, and $Xx$. Compound 
formulas are formed from these in the usual way, and we assume all the usual definitions,
e.g. of $\Box$ in terms of $\Diamond$ and $\neg$.
\subsection{Logic}
The logical (non-set-theoretic) axioms can reasonably naturally be separated 
into those concerning the first-order part of the language, those that concern 
the modals, those that govern the second order variables, 
and those that concern identity.

For the first order part, we take the axioms to be any standard system of first
order quantification logic, with universal instantiation weakened to its `free'
version, that is, with the universal instantiation axiom written in the form: 
\begin{equation}
    \forall x [\forall y [ \varphi y] \rightarrow \varphi x]
\end{equation}
With regards the modal logic, we assume $\mathsf{S4.2}$ for each modal operator,
which (given any standard axiom system) will imply the converse Barcan formula. 
We also take necessitation to be part of the system; and, in 
accordance with the idea that $\Diamond$ is the most general modal at issue, we take 
each of the `weakening' principles
\begin{equation}
    \du \varphi \rightarrow \Diamond \varphi
\end{equation}
\begin{equation}
    \dl \varphi \rightarrow \Diamond \varphi
\end{equation}
as further axioms.
The standardly valid inference rule 
\begin{prooftree}
    \AxiomC{$\Phi_1 \rightarrow \Box (\Phi_2 \rightarrow ... \Box ( \Phi_n \rightarrow \Box \Psi) ... )$}
    \UnaryInfC{$\Phi_1 \rightarrow \Box (\Phi_2 \rightarrow ... \Box ( \Phi_n \rightarrow \Box \forall x \Psi) ... )$}
\end{prooftree}

will also be assumed.

As to the second order logic, we assume full comprehension \emph{in closed form},
so all expressions of the form 
\begin{equation}
    \Box \forall \vec{z} \Box \forall \vec{Z} \exists X \forall x[Xx \leftrightarrow \Phi(x, z, Z)]
\end{equation}
are axioms.
Here of course $X$ may not appear free in $\Phi$, but we assume all variables 
other than 
$x$ that are free in $\Phi$ occur in the lists $\vec{z}, \vec{Z}$. We also assume 
a version of the axiom of choice, to the effect that if $X$ comprises disjoint 
non-empty sets then there is $Y$ containing exactly one element of 
each component of $X$. (Formalizing this in our language is easy enough, using the 
usual definitions of non-empty and disjoint in terms of $\in$.)

Finally we turn to the axioms concerning identity, which also tend to involve 
all the other components of the language. As usual, we assume reflexivity and 
Leibniz law for first and second order identity; and as usual we are able to 
derive the necessity of identity from these in the form 
$\forall x \forall y [ x = y \rightarrow \Box x = y]$, analogously for $X$ and $Y$.\footnote{
    Note however that this implies nothing about inexistent values of variables,
    which may be contingently identical given our axioms up to now. That is, 
    $x = y \wedge \Diamond x \not = y$ is consistent.
} However, since our modal logic is not symmetric, we must add the necessity of 
distinctness ($\forall x \forall y [x \not = y \rightarrow \Box [x \not = y]]$) 
`by hand'. 

Our conception of identity for `plural' variables $X$ is in addition \emph{strongly
extensional}, meaning that plurals comprise the same things at all possible worlds. 
We can enforce this conception axiomatically using the following principles.
\begin{equation}
    \forall x [Xx \leftrightarrow Yx] \rightarrow X = Y
\end{equation}
\begin{equation}\label{prig}
    \Diamond Xx \rightarrow \Box Xx
\end{equation}
\begin{equation}\label{pbf}
    \Diamond \exists x [Xx \wedge x = y] \rightarrow \exists x [Xx \wedge x = y]
\end{equation}
Given the inference rule mentioned above, the latter implies a version of the 
Barcan formula for $Xx$, and hence that pluralities do not `pick up' new components 
from world to world. Overall, the effect is to ensure (speaking model-theoretically 
for a moment) that $Xx$ holds at some possible world iff it holds at all 
possible worlds, and that some things are the same things as some others when and only
when they are composed of the same individuals. 
\subsection{Set Theory}
With the background logic in tow we can formalize the axiomatic system of height
and width potentialism that will be the target of our investigations.

The set-theoretic axioms can themselves be reasonably naturally divided into two categories.
First, there are those that concern the identity conditions for sets. Second, there are those 
that concern possible set existence, and that make assertions about the kinds of sets it is 
possible to produce by iterating our various construction procedures.

First, on the side of identity conditions. We would like sets to have the members they have as a matter of 
necessity, so that (like plurals) they have exactly the members they have at any world in all worlds
(to slip into model-theoretic talk again). This can be imposed by the following pair of axioms:
\begin{equation}
    \forall z[z \in x \leftrightarrow z \in y] \rightarrow x = y
\end{equation}
\begin{equation}\label{ecompl}
    \exists X\Box\forall z[Xx \leftrightarrow z \in x]
\end{equation}
where $x$, in the latter, is an arbitrary set parameter. 
Analogs of \eqref{prig} and \eqref{pbf} for $\in$ can be derived using \eqref{prig}, \eqref{pbf}
and \eqref{ecompl}. As a final constraint of sorts on the identity conditions for sets, we 
impose the standard axiom of foundation, which says that every non-empty set has a member with 
which it shares no members. 

As to the set-theoretic axioms, we begin with a discussion 
of the distinctive axioms for height potentialism, for which we will 
follow the development of \O ystein Linnebo in CITE.
Here the central axiom is 
\begin{equation}\label{hpot}
    \Box \forall X \du \exists x[Set(x, X)]
\end{equation}
where $Set(x, X)$ is an abbreviation for $\forall y[ y \in x \leftrightarrow Xy]$. This 
axiom can intuitively be read as saying that any possible things are possibly the elements of a set;
it leads to a form of indefinite extendibility of the universe of sets in light of the modal logical 
derivability of $\Box \exists X \neg \exists x[Set(x, X)]$ (Russell's paradox).

This axiom by itself guarantees the possible existence of each hereditarily finite set, but 
in the spirit of pursuing transfinite set theory in the potentialist system we will want to 
push things further. As a first step, consider the axiom 
\begin{equation}\label{compn}
    \du \exists X \Box \forall x [ Xx \leftrightarrow Nat(x)]
\end{equation}
where here $Nat(x)$ can be any of your favorite definitions of natural number 
(e.g. finite von Neumann ordinal.) 
\eqref{compn} says that by repeated acts of collection one can eventually 
produce all possible natural numbers,
and given this a further application of \eqref{hpot} secures the possible 
existence of an infinite set.

\eqref{compn} is a natural analogue of the axiom of infinity in the potentialist setting. The 
natural analogue of the powerset axiom would be 
\begin{equation}\label{comps}
    \du \exists X \Box \forall x [ Xx \leftrightarrow x \subseteq y]
\end{equation}
where $\subseteq$ is defined as usual and $y$ is an arbitrary set parameter. However this 
axiom will be false on the intended interpretation, since given any infinite $y$ we will have 
that it is always possible to introduce new subsets of $y$ in the form of enumerating functions /
generic filters. But instead of giving up altogether on the infinitary mathematics that goes 
along with powerset, we will instead adopt the `restriction' of the principle to the modality $\bu$.
The idea will be that, given any arbitrary set $y$, by repeatedly introducing sets one will 
eventually get all the subsets of it that one can ever get \emph{without using forcing}. The 
axiom thus reads:
\begin{equation}\label{comps2}
    \du \exists X \bu \forall x [ Xx \leftrightarrow x \subseteq y]
\end{equation}
 This is 
a kind of local powerset axiom for `inner models'; the precise sense in which this is true will 
be made apparent in more detail below.

We will also adopt the following axiom, distinctive of width potentialism. In it, we let 
$D(x, X)$ abbreviate the claim that $x$ is a partial order and $X$ contains all the dense subsets 
in $x$; and $Fmeets(x, X)$ will abbreviate the claim that $x$ is a filter that 
meets all the sets in $X$.
\begin{equation}\label{wpot}
    \Box \forall X, x [D(x, X) \rightarrow \dl \exists g[Fmeets(g, X)]]
\end{equation}
One can show, using the other axioms, that \eqref{wpot} implies the negation of \eqref{comps}.

Finally, we adopt a version of the axiom of replacement (in fact collection). Say $\varphi$ is \emph{chaotically modalized}
iff every existential quantifier is prefixed by one of $\du, \dl,$ or $\Diamond$, and similarly for 
universal quantifiers and the boxes. Then 
\[\Box \forall x \in a \Diamond \exists y \varphi (x, y) 
\rightarrow \du \exists b \forall x \in a \exists y \in b \varphi(x, y)\]
is an axiom for every chaotically modalzed $\varphi$. The intuition here, as will be borne out 
below, is that chaotically modalized $\varphi$ are a broad and natural class of `rigid' formulas,
that is formulas that satisfy $\Diamond \varphi \rightarrow \Box \varphi$, and hence the 
relation $\varphi(x, y)$ is defined stably enough that collection is appropriate over given sets $a$.
The fact the consequent begins with $\du$ reflects that replacement is a principle of height expansion.

\section{Basic Facts}
In this section we articulate some basic facts about the axiom system just presented, 
along with some general results that will prove useful later on.






\end{document}