\documentclass{article}
\usepackage{amsmath, amsthm, amssymb}
%
\title{On the Consistency Strength of Axioms for Height and Width Potentialism}
%
\author{Chris Scambler 
\\
All Souls College\\
University of Oxford}
%
% These will be typeset in italics
\newtheorem{Theorem}{Theorem}[section]
\newtheorem{Proposition}[Theorem]{Proposition}
\newtheorem{Lemma}[Theorem]{Lemma}
\newtheorem{Corollary}[Theorem]{Corollary}
\usepackage{times}
% These will be typeset in Roman
\theoremstyle{definition}
\newtheorem{Definition}[Theorem]{Definition}
\newtheorem{Fact}[Theorem]{Fact}
\newtheorem{Conjecture}[Theorem]{Conjecture}
\newtheorem{Remark}[Theorem]{Remark}
%
\begin{document} 
\maketitle
%\begin{abstract} \end{abstract}

\section{Introduction}
Potentialism in the foundations of mathematics is a nebulous collection of ideas
with links in logic and metaphysics. All the various approaches to potentialism 
are united by a common theme, namely that modal logic is somehow useful in studying 
mathematical domains that have traditionally only been approached in the setting 
of quantification theory. But the ways in which modal logic is applied,
and the underlying motivations for its application, vary from instance to instance
quite considerably.


\end{document}