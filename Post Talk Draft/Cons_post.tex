\documentclass{article}

\renewcommand*\familydefault{\sfdefault} 
\newcommand{\du}{\Diamond_\uparrow}
\newcommand{\dl}{\Diamond_\leftarrow}
\newcommand{\bu}{\Box_\uparrow}
\newcommand{\bl}{\Box_\leftarrow}

\usepackage[T1]{fontenc}
\usepackage{amsmath, amssymb}
\usepackage{bussproofs}

\title{On the Consistency Strength of Axioms for Height and Width Potentialism}
\author{Chris Scambler}
\date{\today}
\begin{document}
\maketitle
\abstract{
    There are two parameters along which the universe of sets might 
    conceivably differ in size: it may have more or fewer ordinals, 
    and respectively may naturally be construed as taller or shorter;
    or at any given ordinal height, the initial segment of the 
    universe up to that ordinal may have more or fewer subsets, 
    and the universe may in each case naturally be constured, respectively, 
    as fatter or thinner.
    A potentialist about the height of the universe believes that, no matter what,
    there could always be further ordinals; 
    a potentialist about the width of the universe believes that, no matter what, there could 
    always be more subsets of any given infinite set.
    
    
    It is well known that natural axiom systems for height potentialism 
    exhibit tight forms of proof-theoretic equivalence with standard set theories 
    associated with ZFC (and its extension by large cardinals). 
    The recent literature has seen some discussion of 
    height and width potentialism combined. But analogous results precisely correlating 
    these axiom systems with familiar, purely quantificational theories do not yet exist.    
    
    This paper provides such results. I show that the axioms for height and width 
    potentialism from Scambler CITE exhibit a tight form of equivalence with 
    second order arithmetic extended by topological regularity principles. 
    The results are significant because they 
    helps to round out the foundational 
    picture underlying height and width potentialism, and to bring out how different it 
    is from the standard ZFC-conception of set.
}
\section{Overview}
I will begin by discussing the sense in which axiom systems for height potentialism
are equivalent to set theories related to ZFC, and the techniques used to attain 
the equivalence theorems. I then present the relevant axioms fo height and width 
potentialism, and prove they share the same form of equivalence with certain extensions 
of second order arithmetic. Finally, I discuss the relevance of these results for our understanding 
of the height/width potentialist point of view.
\section{Warm Up: Height Potentialism}
We begin with a review of some familiar territory. Though the correctness of 
the results here will be 
obvious to specialists, and though there are similar arguments on offer in the literature 
(cf Button, Linnebo), I'll present some of the details here, partly because the 
results have never been given for the precise set of axioms to be discussed, and 
partly because seeing how the arguments go in the simpler case of plain height potentialism
will help to give intuitions that will be handy for 
the more complicated case of height and width potentialism.
\subsection{Axioms}
We start with a discussion of the `natural' axiom systems for height potentialism 
that are at issue.

The systems will be directed at axiomatizing the following 
intuitive conception. Imagine there is a being who is capable of taking things 
and {\bf collecting} them together into a set, and of repeatedly performing this act 
as many times as is conceivable. Say that a \emph{set} is something that, given enough 
executions of this basic act, could be produced by such a being. 

We want to axiomatize this conception in modal logic. The modal operator, $\Diamond$, 
is to be interpreted so that $\Diamond \varphi$ means
`by repeated acts of collection, $\varphi$ can be made to hold'. The dual 
operator, $\Box \varphi$, means `no matter how many acts of collection are performed, 
$\varphi$ will hold'. The axioms governing this modal logic can be taken to be 
$\mathsf{S4.2}$ with the converse Barcan formula. More details are given in Appendix A.\footnote{(We do not want $.3$, 
since given $a \not= b$ we may form either $\{a\}$ or $\{b\}$, but we do want 
$.2$,
so from either of these we can get $\{a\}$ and $\{b\}$.) The rule of necessitation 
is assumed, along with the standardly valid inference rules:
\begin{prooftree}
    \AxiomC{$\Phi_1 \rightarrow \Box (\Phi_2 \rightarrow ... \Box ( \Phi_n \rightarrow \Box \Psi) ... )$}
    \UnaryInfC{$\Phi_1 \rightarrow \Box (\Phi_2 \rightarrow ... \Box ( \Phi_n \rightarrow \Box \forall x \Psi) ... )$}
\end{prooftree}}
 
Our conception says that the being can take any \emph{things} whatsoever and collect 
\emph{them} into a set. These are plural notions, and accordingly we shall want a plural 
langauge as well. We thus take a stock of plural variables $X$ and singular variables $x$.
The circumstance of something $x$ being one of some things $X$ will be expressed by 
the concatenation $Xx$. The precise details, once more,  are consigned to Appendix A. 

The set-theoretic axioms are more interesting. We assume all the usual 
background stuff: the axiom of extensionality for sets, rigidity for the 
membership relation, and the axiom of foundation. But in addition, we have 
distinctively potentialist existence principles, the most fundamental of which 
being:

\begin{description} 
    \item[HP] $\Box \forall X \Diamond \exists x [Set(x, X)]$
\end{description}
Here and throughout, $Set(x, X)$ is an abbreviation for 
$\forall z[z \in x \equiv Xz]$; {\bf HP} thus says: \emph{necessarily, any 
things can be the elements of a set}. Under the given 
intuitive interpretation, this corresponds to the idea that the being 
can always collect together any given things into a set in any circumstances.  
(Hence, in contrast to standard set theory, there are no 
special things that somehow can't be made to form a set.)

{\bf HP} allows us (given plural comprehension) to prove the possible existence 
of each specific hereditarily finite set. But it does not allow for a proof that 
it is possible to produce any infinite sets. 

In the spirit of transfinite 
set theory, we will want to secure this possibility axiomatically. 
One way to to this uses the following {\bf Completeability} principle. 
Let $\mathbb{N}(x)$ be the assertion that $x$ is a transitive set well-ordered 
by $\in$ which, in addition, is Dedekind finite. Then we have the following 
so-called completeability principle for the nautral numbers:
\begin{description}
    \item[Comp$_\mathbb{N}$]
    $\Diamond \exists X \Box \forall y[ Xy \leftrightarrow \mathbb{N}(x)]$
\end{description}
This says that it is possible, eventually, for the being to create \emph{all possible}
natural numbers. It can be used to prove the possibility 
e.g. of $\omega$'s existence, given the other axioms.

Indeed, in conjunction with {\bf Comp$_\mathbb{N}$}, 
{\bf HP} proves the possible existence of infinitely many
infinite sets. But there is no guarantee from the axioms stated up to now that 
the modal analogue of the powerset axiom should old: that it should be possible, 
given any set, infinite or no, eventually to produce \emph{all} its possible subsets. 
We also have no guarantee that any uncountable sets are possible.

Axiomatically, these things can be imposed by appeal to another completeability axiom:
\begin{description}
    \item[Comp$_\subseteq$]
    $\Box \forall x \Diamond \exists X \Box \forall y[ Xy \leftrightarrow y \subseteq x]$
\end{description}
This time, we assert that the subsets of any given set are always `completeable': 
under our intuitive interpretation, 
it says that one can always eventually form all possible subsets of any given set. 
The axiom can easily be shown is readily seen to have the desired consequences, as 
for example proving the possibility of uncountable sets in the strong sense that 
it is necessary that there is no function defined on the natural numbers with the set 
a subset of its range.

As a final axiom, or rather axiom scheme, we throw in a modal 
translation of replacement (the $\Diamond$-translation in the sense defined below).

These are all fairly natural axioms to adopt if one wishes to explicate the idea 
of a \emph{never ending set-construction process} in modal terms. One has an axiom 
asserting that any things possibly form a set, along with various other axioms 
that implicitly concern the lengths of iterations of set formation that are possible.

For ease of reference the axioms 
for this system are compiled in Appendix A. I will refer to them as the system $\mathsf{L}$,
for their creator, \O ystein Linnebo.

We now turn to a discussion of the proof-theoretic relations between $\mathsf{L}$
and ZFC. 
\subsection{Key Results}
There are three key results of interest. I will state them and indicate the methods of proof. 
Full proofs are provided in Appendix B. 

TODO: YOU NEED TO REDO THIS SO BOTH THEORIES ARE SET IN 
SECOND ORDER LOGIC. THEN YOU CAN BRING OUT THE DIFFERENCES IN 
TERMS OF PLURAL COMPREHENSION AND MORE EASILY PROVE THE FIRST ORDER 
EQUIVALENCE. YOU CAN ALSO DISCUSS STRATGIES FOR INTERPRETING 
THE SECOND ORDER CASE ALONG THE LINES OF BUTTON, WHICH YOU REALLY 
MUST READ AND UNDERSTAND.

The {\bf first key result} is that there is a recursive mapping 
$\varphi \mapsto \varphi^\exists$ from the potentialist 
language to the language of first order set theory that preserves theoremhood 
from $\mathsf{L}$ into ZFC.

The {\bf second key result} is that there is a recursive mapping 
$\varphi \mapsto \varphi^\diamond$ from the language of first 
order set theory to the modal language that preserves theoremhood 
from ZFC into $\mathsf{L}$.

The {\bf third key result} is that the first order fragments of these theories 
exhibit what Button (cite) has called a \emph{near-synonymy} relation.

The latter takes a little unpacking. What does that mean? 
The relation of \emph{full synonymy} or \emph{definitional equivalence} between 
the first order fragments would require, in addition to the first two key results, 
that ZFC should prove $\varphi \leftrightarrow (\varphi^\Diamond)^\exists$, 
and that $\mathsf{L}$ should prove
$\varphi \leftrightarrow (\varphi^\exists)^\Diamond$, where $\varphi$ in each case 
is restricted to the first order fragment of the language (but where it may, if a 
$\mathsf{L}$ formula, contain modal operators).

It would be nice if this were so. But unfortunately it cannot \emph{quite} be. 
The translation $\varphi^\exists$ 
involves interpreting unrestricted quantification in the modal language as 
tacitly restricted 
to sets (``possible worlds''), and catering for the extra parameters they 
induce means that there
are extra assumptions we need to make to ensure such an equivalence is 
provable. For 
instance, we have to assume that the thing the actualist calls 
`the actual world' (for the potentialist)
contains exactly all those things the potentialist recognizes. 
Hence the maps 
are only inverses if we throw in some extra assumptions on the parameters.  
Thus we are 
left only with what Button has called \emph{near synonymy} 
rather than full definitional 
equivalence.

The difference between the two is not great. 
Indeed to me it looks like a trivial book-keeping detail. 
All the model-theoretic consequences one gets 
from definitional equivalence, for example, carry over fairly straightforwardly to the form of 
near-definitional equivalence we acquire. 
Thus there is a very natural sense in which the first order 
picture of the sets painted by ZFC is equivalent to the 
first order modal account axiomatized by theories like L. 
What more to make of this, and the relevance of the restriction to the 
first order fragment, will be discussed below.

Let me go into the methods of proof for these results. 

First, let's discuss the translation $\varphi \mapsto \varphi^\exists$ from the first key result.
It has to translate the whole potentialist idiolect into the austere idiom of the 
first order language. How is that to be done? An important part of the idea, as I said, is to introduce a
notion of a possible world and take the potentialist to really be using quantifiers 
restricted to such worlds. But the speaker of a first order language must also 
find some interpretation for the potentialist's plural variables. To do that, 
they will just interpret plural talk as unusual talk about sets (whose elements 
are the `plurals'). 

To implement this, we take all the 
first order variables $x_k$ of the potentialist, and 
map them to the even numbered variables $x_{2k} := (x_k)_\exists$ 
in our first order langauge; 
and we take the pural variables $X_k$ of the 
potentialist and map them to the odd numbered variables $x_{2k+1} := (X_k)_\exists$ 
of our first order logic. Clearly, this is a way of interpreting talk of 
plurals as just talk of some special objects.

Now that we have an intepretation for the potentialist's variables we can proceed to specify 
the interpretation more generally. We set $x \in y^\exists$ and $Xx^\exists$ as $x_\exists \in y_\exists$ 
and $x_\exists \in X_\exists$ respectively. The connectives commute with the translation. For the quantifiers,
we interpret the potentialist's assertion that any thing(s) has(have) the property $P$ as saying 
that any thing(s) \emph{in the domain} has(have) the property $P$ \emph{relative to that domain}. More explicitly, 
the translation introduces a free (set) variable $d$ (to be thought of as ``the domain'') and sets:
\[(\forall x \varphi)^\exists := \forall x \in d [\varphi^\exists(d)]\]
Modal notions are defined in terms of quantification over extensions of the domain:
\[ (\Diamond \varphi)^\exists := \exists e \supseteq d [\varphi^\exists(e)]\]
It is now tedious but routine to prove that theoremhood is preserved 
in the transition $\varphi \mapsto \varphi^\exists$ (as long as we require $d$ and $e$ 
to range over \emph{transitive} sets).

The key idea behind the other translation, $\varphi \mapsto \varphi^\diamond$, 
is to simply put a $\Diamond$ in front of every existential quantifier (and a $\Box$ in front of 
every universal). For variables, we set $(x_{2k})_\diamond := x_k$, and $(x_{2k+1})_\Diamond = X_k$.
It is easy to show theoremhood is preserved.

The third key result is not hard, with the only tricky part being coming up with precisely 
the right statement. Here it is:

Part one. ZFC proves that for any transitive $d$, 
$\varphi \equiv (\varphi^\Diamond)^\exists(d)$.

Part two. $\mathsf{L}$ proves that if $d$ is a universal set,
then $\varphi \equiv (\varphi^\exists d)^\Diamond$.


\subsection{Discussion}
\section{Axioms}
In this section I will give the system of axioms for height and width potentialism 
that will be the subject of the rest of the paper.

The system is directed at axiomatizing the following intuitive (if fanciful) conception.
As before, we have a being who is able to take things and {\bf Collect} them together into a set.
But now, we will also take them to be able to perform another kind of basic act: namely, 
that of taking some things, and {\bf Counting} them, that is, correlating them one-for-one
with the natural numbers. Our new theory will concern the possibilities for arbitrary iterations 
of {\bf Collect} and {\bf Count}. This time, we say that a \emph{set} is something that can be 
gotten eventually by arbitrary iterations of these operations. {\bf Collect} is a height potentialist 
principle, in that the set collected together has higher rank than the collected sets, 
while {\bf Count} is naturally understood as a width potentialist principle, 
as the introduction of an enumerating function for a given set may corresponds to the introduction 
of new sets of the same or lower rank as the sets counted in many cases--
indeed, up to isomorphism, it will correspond to introducing a subset of the naturals.

There are many natural modal operators that one can read off from this conception. 
There is for example a modal, which I will write $\du \varphi$, which corresponds to 
possibility only taking acts of {\bf Collect} into consideration. Since this is the 
height potentialist sense of possibility I shall often refer to this as {\bf vertical}
possibility. There is also a natural modal, $\dl \varphi$, which corresponds to possibility 
just by iterated enumeration. This might naturally be read as {\bf horizontal} possibility.
 And there is a general modal operator, $\Diamond \varphi$, which 
corresponds to $\varphi$s being possible by some arbitrary number of iterations and alternations 
of the two available operations. I will refer to this as {\bf general} possibility. There are 
many others besides, for example that induced by $\du\dl$, but these will play little direct role.
In fact, for our purposes, even $\dl$ will be superfluous. We will focus our attentions on 
the two operators $\du$ and $\Diamond$, where the former corresponds to possibility just by 
{\bf Collection}, and the latter to possibility by combinations of {\bf Collect} and {\bf Count}.

So we take a multi-modal language with the two modal operators $\Diamond$ and $\du$. We still 
want the capacity to talk about pluralities and so we keep the two types of variable. 
And we want to talk about sets so we include the membership relation symbol $\in$. 

What axioms should we impose? Well, we want all the same plural and first order stuff.
Both of the modal operators should have $\mathsf{S4.2}$, and
we have $\du \varphi \rightarrow \Diamond \varphi$. We take over 
all the axioms from $\mathsf{L}$ for the $\du$ modality: the idea being, nothing 
has changed from before, and if we could get something without forcing/enumeration before 
we can get it by ignoring forcing/enumeration now.
\section{Key Results}
\section{Discussion}
\section{Stuff}
Modal Set Theory:
Absolutness of various kinds.

Set Theory: 
Pi 1-1 PSP equivalent to only countably many reals in L[r].

Logic: 
Define a formula to be \emph{pseudo typed} (PT) iff 
odd numbered variables only ever occur
to the right of $\in$. 
Then in ZFC, any fomula $\varphi$ has a (weak) PT equivalent
$PT(\varphi)$ with $ZFC \vdash PT(\varphi)$ iff $ZFC \vdash \varphi$, 
and any sentence 
$\varphi$ has a (strong) PT equivalent $PT(\varphi)$ with 
$ZFC \vdash \varphi \leftrightarrow PT(\varphi)$.

Proof. Let $n(\varphi)$ be a number greater than the indices of variables in 
$\varphi$. For each $n$ let $E_n$ be an enumeration of 
the evens greater than $n$, so e.g. $E_2(0)=3$. 

We define $PT(\varphi)$ recursively. Moving from left to right in the formula 
keep looking until you find a variable out of place. Then check to see 
if it is free or bound. If it is free, replace all other free occurrences of the 
variable by $E_{n(\varphi)}(0)$; If 
it is bound, do the same for all its occurrences in the scope 
of the binding quantifier. In either case, a formula $PT_0(\varphi)$
results. Then given $PT_n(\varphi)$, do the same thing to get $PT_{n+1}(\varphi)$.
Whenever $PT_n(\varphi) = PT_{n+1}(\varphi)$, $PT_n(\varphi)$ is $PT$. We set 
such $PT_n(\varphi) := PT(\varphi)$.

$PT(\varphi)$ is provable iff $\varphi$ is. Induction on the number of out-of-place 
variables. Suppose it holds for $n$. 

\end{document}