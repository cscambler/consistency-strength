\documentclass{article}

\renewcommand*\familydefault{\sfdefault} 
\newcommand{\du}{\Diamond_\uparrow}
\newcommand{\dl}{\Diamond_\leftarrow}
\newcommand{\bu}{\Box_\uparrow}
\newcommand{\bl}{\Box_\leftarrow}

\usepackage[T1]{fontenc}
\usepackage{amsmath, amssymb}
\usepackage{bussproofs}

\title{On the Consistency Strength of Axioms for Height and Width Potentialism}
\author{Chris Scambler}
\date{\today}
\begin{document}
\maketitle
\abstract{
    There are two parameters along which the universe of sets might 
    conceivably differ in size: it may have more or fewer ordinals, 
    and respectively may naturally be construed as taller or shorter;
    or at any given ordinal height, the initial segment of the 
    universe up to that ordinal may have more or fewer subsets, 
    and the universe may in each case naturally be constured, respectively, 
    as fatter or thinner.
    A potentialist about the height of the universe believes that, no matter what,
    there could always be further ordinals; 
    a potentialist about the width of the universe believes that there could 
    always be more subsets of any given infinite set.
    
    
    It is well known that natural axiom systems for height potentialism 
    exhibit tight forms of proof-theoretic equivalence with standard set theories 
    associated with ZFC. The recent literature has seen some discussion of 
    height and width potentialism combined, but similar results precisely correlating 
    these axiom systems with familiar, purely quantificational theories do not yet exist.
    
    
    This paper provides such results. I show that the axioms for height and width 
    potentialism from Scambler CITE exhibit a tight form of equivalence with 
    second order arithmetic together with the $\Pi_1^1$-Pefect Set Property. 
    This result is significant, because it (and its strengthenings) 
    helps to round out the foundational 
    picture underlying height and width potentialism.
}
\section{Overview}
I will begin by discussing the sense in which axiom systems for height potentialism
are equivalent to set theories related to ZFC, and the techniques used to attain 
the equivalence theorems. I then present the relevant axioms fo height and width 
potentialism, and prove they share the same form of equivalence with theories related 
to second order arithmetic (augmented by `good behavior' or `determinacy' axioms for 
classes of reals). Finally, I discuss the relevance of these results for our understanding 
of the height/width potentialist point of view.
\section{Warm Up: Height Potentialism}
We begin with a review of some familiar territory. Though the correctness of 
the results here will be 
obvious to specialists, and though there are similar arguments on offer in the literature 
(cf Button, Linnebo), we present some of the details here, partly because the 
results have never been given for the precise set of axioms to be discussed, and 
partly because seeing how the arguments go in the simpler case of plain height potentialism
will help to give intuitions when the more complicated case of height and width potentialism 
comes to the fore.
\subsection{Axioms}
We start with a discussion of the `natural' axiom systems for height potentialism 
that are at issue.

Our potentialist theories will be directed at axiomatizing the following 
intuitive conception. Imagine there is a being who is capable of taking things 
and {\bf collecting} them together into a set, and of repeatedly performing this act 
as many times as is conceivable. Say that a \emph{set} is something that, given enough 
executions of this basic act, can be produced by the being. 

We want to axiomatize this conception in modal logic. The modal operator, $\Diamond$, 
is to be interpreted so that $\Diamond \varphi$ means
`by repeated acts of collection, $\varphi$ can be made to hold'. The dual 
operator, $\Box \varphi$, means `no matter how many acts of collection are performed, 
$\varphi$ will hold'. The axioms governing this modal logic can be taken to be 
$\mathsf{S4.2}$ with the converse Barcan formula. (We do not want $.3$, 
since given $a \not= b$ we may form either $\{a\}$ or $\{b\}$, but we do want 
$.2$,
so from either of these we can get $\{a\}$ and $\{b\}$.) The rule of necessitation 
is assumed, along with the standardly valid inference rules:
\begin{prooftree}
    \AxiomC{$\Phi_1 \rightarrow \Box (\Phi_2 \rightarrow ... \Box ( \Phi_n \rightarrow \Box \Psi) ... )$}
    \UnaryInfC{$\Phi_1 \rightarrow \Box (\Phi_2 \rightarrow ... \Box ( \Phi_n \rightarrow \Box \forall x \Psi) ... )$}
\end{prooftree}
 
Our conception says that the being can take any \emph{things} whatsoever and collect 
\emph{them} into a set. These are plural notions, and accordingly we shall want a plural 
langauge as well. We thus take a stock of plural variables $X$ and singular variables $x$.
The circumstance of something $x$ being one of some things $X$ will be expressed by 
the concatenation $Xx$. Identity is type-restricted. The plural logic includes full 
comprehension (in closed form), the law of extensionality, rigidity for plurals, a 
version of the axiom of chocie, and 
the following restricted form of the Barcan formula:
\[\Diamond \exists x[Xx \wedge x = y] \rightarrow \exists x[Xx \wedge x = y]\]
Quantification is free for both kinds of variable.

We can now move on to the discussion of the properly set-theoretic axioms. 
We assume the axiom of extensionality for sets, as well as rigidity for the 
membership relation, and the axiom of foundation. 

The first distinctively potentialist axiom of the system is 
\begin{description} 
    \item[HP] $\Box \forall X \Diamond \exists x [Set(x, X)]$
\end{description}
(Here and throughout, $Set(x, X)$ is an abbreviation for 
$\forall z[z \in x \equiv Xz]$.) This says: \emph{necessarily, any 
things can be collected together into a set.} 

The axiom allows us (given plural comprehension) to prove the possible existence 
of each specific hereditarily finite set. But it does not allow for a proof that 
there are any infinite sets. This is secured by adoption of an axiom of 
`completeability' for the natural numbers:
\begin{description}
    \item[Comp$_\mathbb{N}$]
    $\Diamond \exists X \Box \forall y[ Xy \leftrightarrow \mathbb{N}(x)]$
\end{description}
This says that it is possible for there to be some things that comprise \emph{all possible}
natural numbers. Since each partiular natural number possibly exists using iterated 
{\bf HP}, this guarantees the possibility of an infinite set. (Here $\mathbb{N}$ is any formula 
defining the notion of a natural number, say as a Dedekind-finite 
Von Neumann ordinal.) One can think of the axiom as saying that, in addition to 
each particular natural number being possible to produce, one can in fact go on long 
enough to get them all.

In conjunction with {\bf Comp$_\mathbb{N}$} {\bf HP} proves the possibility of a lot of
infinite sets. But there is no guarantee from the axioms stated up to now that 
the modal analogue of the powerset axiom should old: that it should be possible, 
given any set, infinite or no, eventually to produce \emph{all} its possible subsets. 
We also have no guarantee that any uncountable sets are possible.
Axiomatically, these things can be imposed by appeal to another completeability axiom:
\begin{description}
    \item[Comp$_\subseteq$]
    $\Box \forall x \Diamond \exists X \Box \forall y[ Xy \leftrightarrow y \subseteq x]$
\end{description}
This says that one can always eventually form all possible subsets of any given set.

As a final axiom, or rather axiom scheme, we throw in a certain modal translation of replacement,
whose exact nature will be specified below for the sake of convenience.

These are all fairly natural axioms to adopt if one wishes to explicate the idea 
of a \emph{never ending set-construction process} in modal terms. One has an axiom 
asserting that any things possibly form a set, along with various other axioms 
that implicitly concern the lengths of iterations of set formation that are possible.

For ease of reference the axioms 
for this system are compiled in Appendix A. 

We now turn to a discussion of the proof-theoretic relations between this `A axiom system'
and ZFC.
\subsection{Lemmas}
\subsection{Results}
\subsection{Discussion}
\section{Lemmas}
Modal Set Theory:
Absolutness of various kinds.

Set Theory: 
Pi 1-1 PSP equivalent to only countably many reals in L[r].

Logic: 
Define a formula to be \emph{pseudo typed} (PT) iff 
odd numbered variables only ever occur
to the right of $\in$. 
Then in ZFC, any fomula $\varphi$ has a (weak) PT equivalent
$PT(\varphi)$ with $ZFC \vdash PT(\varphi)$ iff $ZFC \vdash \varphi$, 
and any sentence 
$\varphi$ has a (strong) PT equivalent $PT(\varphi)$ with 
$ZFC \vdash \varphi \leftrightarrow PT(\varphi)$.

Proof. Let $n(\varphi)$ be a number greater than the indices of variables in 
$\varphi$. For each $n$ let $E_n$ be an enumeration of 
the evens greater than $n$, so e.g. $E_2(0)=3$. 

We define $PT(\varphi)$ recursively. Moving from left to right in the formula 
keep looking until you find a variable out of place. Then check to see 
if it is free or bound. If it is free, replace all other free occurrences of the 
variable by $E_{n(\varphi)}(0)$; If 
it is bound, do the same for all its occurrences in the scope 
of the binding quantifier. In either case, a formula $PT_0(\varphi)$
results. Then given $PT_n(\varphi)$, do the same thing to get $PT_{n+1}(\varphi)$.
Whenever $PT_n(\varphi) = PT_{n+1}(\varphi)$, $PT_n(\varphi)$ is $PT$. We set 
such $PT_n(\varphi) := PT(\varphi)$.

$PT(\varphi)$ is provable iff $\varphi$ is. Induction on the number of out-of-place 
variables. Suppose it holds for $n$. 
\section{Results}
\section{Conclusion}

\end{document}