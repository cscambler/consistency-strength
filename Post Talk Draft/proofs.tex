\documentclass{article}
\renewcommand*\familydefault{\sfdefault} 
\newcommand\D[1]{\langle #1 \rangle}
\newcommand\B[1]{[ #1 ]}
\newcommand\I[1]{ #1^* }
\newcommand\Ix[2]{ #1^#2 }
\usepackage[T1]{fontenc}
\usepackage{amsmath, amssymb, amsthm, stmaryrd}
\usepackage{mdframed}
\usepackage{bussproofs}
\newtheorem{lemma}{Lemma}
\newtheorem{corollary}{Corollary}
\newtheorem{definition}{Definition}
\newtheorem{theorem}{Theorem}

\title{Proofs}
\author{CS}
\date{\today}
\begin{document}
\maketitle

\section{ZFC and L}

\subsection{Formulation of ZFC}
\subsubsection{Language $\mathcal{L}_\in^2$}
\paragraph{Signature}
\begin{itemize}
    \item countable infinity of first order variables $x_i$
    \item countable infinity of monadic second order variables $X_i$
    \item propositional connectives $\vee, \neg$
    \item variable binding quantifier $\exists$
    \item relation $=$
    \item relation $\in$
\end{itemize}

\paragraph{wffs}
    $$Xx | x \in y | x = y | X = Y$$
    $$\varphi \wedge \psi | \neg \varphi | \forall x \varphi | \forall X \varphi $$

\paragraph{defs}
\begin{itemize}
    \item usual defs for connectives quantifiers
    \item symbol for $\emptyset$, $\{x : \psi \}$ notation
\end{itemize}

\subsubsection{Axioms of ZFC}

Standard second order logic with full comprehension and extensional second order 
identity. For set theoretic axioms:

\begin{description}
    \item[Ext$_\forall$] $\forall x[x \in y \equiv x \in z] \supset y = z$
    \item[Fun$_\forall$] $ x \not = \emptyset \supset \exists y[y \in x \wedge y \cap x = \emptyset]$
    \item[Pair] $\exists z[z = \{x, y\}]$
    \item[Un] $\exists y [y = \bigcup{U}(x)]$
    \item[Rep] $Fun (F) \supset \forall x\exists y(y = F|x)$
\end{description}



\subsection{Formulation of L}

\subsubsection{Language $\mathcal{L}_0$}
\paragraph{Signature}
\begin{itemize}
    \item countable infinity of first order variables $x_i$
    \item countable infinity of monadic second order variables $X_i$
    \item propositional connectives $\vee, \neg$
    \item variable binding quantifier $\exists$
    \item relation $=$
    \item operator $\Diamond$
    \item relation $\in$
\end{itemize}

\paragraph{wffs}
    $$Xx | x \in y | x = y | X = Y$$
    $$\varphi \wedge \psi | \neg \varphi | \forall x \varphi | \forall X \varphi | \Diamond \varphi$$

\paragraph{defs}
\begin{itemize}
    \item usual defs for connectives quantifiers and modals
    \item $Ex$ is an abbreviation for $\exists y[y = x]$ or a more suitable alphabetic variant
    \item $Set(x, X)$ abbreviates $\forall y[Xy \equiv y \in x]$
    \item Previous abbreviations from set theory
\end{itemize}
\subsubsection{Axioms of L}
We assume all propositional tautologies, along with any standard axioms for positive free quantifier logic.

The modal logic is S4.2 with necessitation (and CBF). The rule of inference ... is also assumed.

As to the plural logic, we assume the following.
\begin{description}
    \item[pExt] $\forall X \forall Y [(\forall x[Xx \equiv Yx]) \supset X = Y]$
    \item[pR] $ \Diamond Xx \supset \Box Xx$
    \item[pBF] $\forall X[ \Diamond (\exists x[Xx \wedge x = y]) \supset \exists x [Xx \wedge x = y]]$
\end{description}
Finally, the set theoretic axioms.
\begin{description}
    \item[Ext] $\forall x \forall y [(\forall z[z \in x \equiv z \in y]) \supset x = y]$
    \item[Ele] $\Box \forall x \exists X \Box \forall y[Xy \equiv y \in x]$
    \item[Fun] $\forall x[ x \not = \emptyset \supset \exists y[y \in x \wedge y \cap x = \emptyset]]$
    \item[Set] $\Box \forall X \Diamond \exists y [Set(y, X)]$
    \item[Inf] $\Diamond X\Box \forall y[Xy \equiv \mathbb{N}(y)]$
    \item[Pow] $\Box \forall x \Diamond \exists X \Box \forall y [Xy \equiv y \subseteq x]$
    \item[Rep] $Rep^\Diamond$
    \item[Min] $\forall y[ y \in x] \supset \Diamond(\forall y[y \in x] \wedge \Diamond (z \in x) \supset Ez)$ 
\end{description}

\subsection{Key Result One}
\begin{theorem} 
There is an interpretation $\exists : \mathcal{L}_0 \to \mathcal{L}_\in$ that preserves 
theoremhood from L to ZFC.
\end{theorem}
\begin{definition}[$\varphi_\exists$]
    The translation $\varphi \mapsto \varphi_\exists$ is defined by the following clauses.
    In each case, $d$ is the least variable not occuring in $\varphi$.
    \begin{itemize}
        \item $x \in y_\exists := x \in y \wedge d = d$
        \item $Xx_\exists := Xx \wedge d = d$
        \item $(\neg \varphi)_\exists := \neg \varphi_\exists(d)$
        \item $(\varphi \vee \psi)_\exists := \varphi_\exists(d) \vee \psi_\exists(d)$
        \item $(\exists x \varphi)_\exists := \exists x \in d[ \varphi_\exists(d)]$
        \item $(\exists X \varphi)_\exists := \exists X \subseteq d [ \varphi_\exists(d)]$
        \item $(\Diamond \varphi)_\exists := \exists e[e \supseteq d \wedge Tran(e) \wedge \varphi_\exists(e)]$
    \end{itemize}
    In these, $\varphi_\exists(e)$ represents the result of substituting $e$ for $d$ in
    $\varphi_\exists$.

    We then set $\exists(\varphi) := Tran(d) \supset \varphi_\exists$.
\end{definition}
\begin{proof}
    The propositional tautologies and modus ponens are obvious. For free universal 
    instantiation, we must show (under the assumptin $Tran(d)$) that
    $$\forall x \in d [ \forall y \in d (\varphi(y))_\exists (d) \supset (\varphi(x))_\exists(d)]]$$
    But this is immediate. (Note however that the unfree quantifier rule,
    which removes the initial quantifier, is not provable 
    under this interpretation.)

    As to the laws of $S4.2$ modal logic, it is completely clear that S4 will hold 
    in light of the reflexivity and transitivity of $\subseteq$. For $.2$, suppose 
    $(\Diamond \Box \varphi)_\exists$. Then there is a transitive extension $e_0$ of $d$ 
    such that every extension $f$ of $e_0$ has $\varphi_\exists(f)$. Suppose given 
    transitive extension $e$ of $d$. Then $f_1 := e \cup e_0$ is a transitive extension 
    of $e_0$ that (therefore) has $\varphi_\exists(f_1)$. 
    Hence $(\Box \Diamond \varphi)_\exists$.\footnote{
        In fact (by transitivity) we have the stronger 
        $\Diamond \Box \varphi \supset \Box \Diamond \Box \varphi$.)
    } For necessitation, we must show that if $Tran(d) \supset \varphi_\exists(d)$
    is a theorem, then so is 
    $Tran(d) \supset \forall e [e \supseteq d \wedge Tran(e) \supset \varphi_\exists(e)]$,
    which is obviously correct.

    All the plural axioms are straightforward, although it is worth 
    remarking that the interpretations become unprovable 
    (in fact demonstrably false) if the initial second order 
    quantifiers are removed. 

    On to the set-theoretic axioms. The case of extensionality reduces to the claim 
    that extensionality holds in transitive sets. Ele and Fun are equally straightforward.
    For set: suppose given a transitive extension $e$ of $d$ and a subset $X$ of $e$. Since 
    every set is an element of a transitive set (ZFC) we can extend $e$ to a transitive 
    set that contains that subset as an element, and the result follows. Inf just comes down 
    to the fact that there is a transitive set that contains the natural numbers. 
    Similarly for pow: given any set there is a transtive set that contains all its subsets.
    Each instance of $\exists(Rep^\Diamond)$ just is an instance of replacement.
\end{proof}

\subsection{Key Result Two}
\begin{theorem}
There is an interpretation $\Diamond: \mathcal{L}_\in \to \mathcal{L}_0$ that preserves theoremhood from ZFC to L \emph{on first order formulas}. 
\end{theorem}
\begin{proof}
    This is just the translation $\varphi \mapsto \varphi^\Diamond$. That the result holds 
    is a theorem of Linnebo.
\end{proof}

\subsection{Key Result Three}
\begin{theorem}
These translations yield a definitional equivalence 
between the first order fragments of ZFC and L in the following sense: for all 
$\varphi$ without second order variables, we have
\begin{enumerate}
    \item $L \vdash Univ(d) \supset (\varphi_\exists)^\Diamond \equiv \varphi$
    \item $ZFC \vdash Tran(d) \supset (\varphi^\Diamond)_\exists \equiv \varphi.$
\end{enumerate}
Here, $Univ(d)$ abbreviates $\forall x[x \in d] \wedge \Diamond(y \in d) \supset Ey$ for 
a suitable choice of (free) $y$.
\end{theorem}
\begin{lemma}
    Let $\varphi^e$ be the restriction of $\varphi$ to the set $e$.
    Then we have (for first order $\varphi$) 
    $$L \vdash Tran(e) \supset \varphi_\exists(e)^\Diamond \equiv \varphi^e $$ 
\end{lemma}
\begin{proof}
    An induction on the complexity of $\varphi$. the base cases are all immediate, as are the propositional connectives. 
    For the quantifier, we need 
    $$L \vdash Tran(e) \supset 
    \Diamond \exists x \in e(\varphi_\exists(e)^\Diamond(x)) 
    \equiv 
    \exists x \in e [\varphi^e(x)]$$
    So suppose $Tran(e)$. Left to right follows by rigidity 
    for membership and the induction hypothesis, and the right 
    to left is similar. 

    For the modal operator, we need 
    $$L \vdash Tran(e) \supset 
    \Diamond \exists f[f \supseteq e \wedge Tran(f) \wedge \varphi_\exists(f)^\Diamond] 
    \equiv 
    \Diamond \varphi^e$$
    so suppose $Tran(e)$. Going left to right, the induction 
    hypothesis gets $$\Diamond \exists f[f \supseteq e \wedge Tran(f) \wedge \varphi^f],$$
    which...
\end{proof}
\begin{proof}
    In each case we proceed by induction on the complexity of $\varphi$.

    For 1., the base cases are all immediate, as are the propositional connectives. 
    For the quantifier, we need 
    $$L \vdash Univ(d) \supset 
    \Diamond \exists x \in d(\varphi_\exists(d)^\Diamond(x)) 
    \equiv 
    \exists x \varphi$$
    So suppose $Univ(d)$. Going right to left, our induction hypothesis 
    yields $\exists x (\varphi_\exists(d)^\Diamond(x))$, and then $Univ(d)$ 
    implies $\Diamond \exists x \in d (\varphi_\exists(d)^\Diamond(x))$ as 
    required. 

    For the converse, suppose 
    $\Diamond Ex \wedge x \in d \wedge \varphi_\exists(d)^\Diamond(x)$.
    Then by $Univ(d)$ we get $Ex$. We also have $\Diamond \varphi_\exists(d)^\Diamond(x)$.
    But by a result of Linnebo this implies $\varphi_\exists(d)^\Diamond(x)$.
    By the IH we have $\varphi(x)$ and the result follows.

    Finally, for $\Box \varphi$, we must show that 
    $$L \vdash Univ(d) \supset 
    \Box \forall e [e \supseteq d \wedge Tran(e) \supset (\varphi_\exists(e))^\Diamond]
    \equiv 
    \Box \varphi$$
    Going left to right, suppose $Univ(d)$ and 
    $\Box \forall e [e \supseteq d \wedge Tran(e) \supset (\varphi_\exists(e))^\Diamond]$.
    Suppose $Univ(e)$. Then $Tran(e)$ and $d \subseteq e$, and hence 
    $\varphi_\exists(e)^\Diamond$. 
\end{proof}

\end{document}